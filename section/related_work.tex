\section{Related Work}\label{sec:related_work}

%\rt{Shall we add some subsections here, since each paragraph has a different topic?}
\subsection{Link Traversal Query Processing}
LTQP is a SPARQL querying paradigm that answers queries by exploring the Web using the follow-your-nose principle~\cite{hartig2016walking}.
It belongs to the family of decentralized SPARQL querying paradigms.
LTQP fundamentally differs from federated querying because the ``federation'' is formed during querying and is expands dynamically as the query is processed.
Thus, many optimization techniques used in federated querying either do not work in the context of LTQP or remain unexplored.
LTQP also differs from querying the TPF family \rt{The formal name here is Linked Data Fragments} of interfaces~\cite{azzam2020smart,azzam2021wisekg, DBLP:journals/corr/HartigA16}, as TPF involves querying fragments of a known local finite dataset, whereas LTQP dynamically discovers new remote datasets during query processing.\
The main challenge of LTQP is the Web's open-ended nature leading to large search spaces.
Completeness in LTQP is defined by the traversal of a well-defined set of links~\cite{Hartig2012}.
The first method employed to define this set was the reachability criteria~\cite{Hartig2012}, which are boolean functions that determine whether a link should be dereferenced.
Building on this, the theoretical query language LDQL~\cite{hartigLDQL} was introduced, separating the traversal definition from the query definition.
Further advancements include the Subweb Specifications Language~\cite{Bogaerts2021LinkTW}, which allows data providers to define how their DKG should be traversed.
These contributions are centered on guiding the engine in selecting links to follow in a discovery process.
However, they do not explicitly address mechanisms for restricting links after the discovery process, based on information acquired during traversal; a process we could refer to as pruning.
Link pruning can significantly reduce the search domain of queries when information about the DKG data model is available.
For example, the structural properties of DESPs could inform the query engine that certain web sections follow a specific data model. 
As a result, a set of IRIs could be pruned from the links selected during the discovery process.
%In contrast, in Linked Open Data, the web is not divided into subsections with structural properties; therefore, data model information cannot be discovered.

\subsection{RDF Data Shape}
RDF data shapes (in this paper, we also refer to them as shapes) are used for validating, describing, and communicating data structures, as well as generating data and driving user interfaces~\cite{Gayo2018a,Gayo2018}.
The two most well known RDF data shape formalisms are SHACL~\cite{Gayo2018b} and ShEx~\cite{Gayo2018}.
%Both describe RDF data but differ in focus: ShEx emphasizes graph structure, while SHACL targets constraints.
For common use cases, they are equally expressive and interchangeable~\cite{Gayo2018c}.
RDF data shapes have already been used in the literature for querying centralized KGs~\cite{kashif2021}.
Shape Trees~\cite{shapetreesShapeTrees} are an index structure for validating and organizing decentralized knowledge graphs (DKGs).
However, to the best of our knowledge Shape Trees have not been used for query optimization. 
Due to their \emph{virtual hierarchy}~\cite{shapetreesShapeTrees}, it can be challenging for a query engine to efficiently capture the relationship between a resource IRI and its corresponding shape.
Furthermore, Shape Trees are not widely adopted.
Thus, proposing a different formalization focus on query optimization by simplifying the relation between shapes and a set of resources can help determine whether shape-based optimization for LTQP is effective.\rt{This sentence feels a bit hypothetical. Let's make it more concrete. "We do X because Y."}
%Automatic generation of RDF data shapes based on KGs~\cite{fernandez2023extracting} and shape-based integration of data~\cite{LabraGayo2023} are also topics that have been studied and can support shape-based summary approaches for DKGs. 

\iffalse
\rt{
    The following paragraph needs to be better structured.
    There's a lot of stuff in there, but it's not very coherent.
    It's introduced as source selection, but that's not accurate, as you mention SERVICE (which is an operator in SPARQL), summarization techniques for query optimization, ... Y
    ou also mix sparql federation and link traversal. 
    And then you end with some centralized techniques. 
    You may even want to have separate subsections for some of these topics. 
    Also think about what point you want to make for each subsection, and how it relates to your work (and then also make this clear to the reader).
}
\fi

\subsection{Source Selection}
\rt{Can you mention here how source selection relates to pruning in LTQP?}
Source selection is a crucial challenge in decentralized querying~\cite{hose2012towards, Harth2010}.
Methods such as basic statistics on triple counts, VoID descriptions, and histogram methods, have been studied in the context of federated querying~\cite{hose2012towards, Harth2010, Montoya2017}.
However, most of those source selection methods face the limitation of assuming a small number of data sources~\cite{Harth2010}, leaving their suitability for LTQP uncertain.
Bloom filters~\cite{dia2018fast} is also a mechanism that has shows success for federated DKGs, however in the context of LTQP it has been show that bloom filters have little effect on performance~\cite{Hanski2024}.
Schemas-based indexing using ontologies~\cite{Stuckenschmidt2004} has also been explored for source selection of SPARQL queries,
however, the proposed approach is sensitive to the high reuse of vocabulary terms in RDF~\cite{Harth2010}, which is exacerbated in the context of LTQP.
The use of implicit RDF schemas for query optimization has been explored with the concept of characteristic sets~\cite{Neumann2011CharacteristicSA, Meimaris2017ExtendedCS, Montoya2017}.
However, their use has not been investigated in LTQP.