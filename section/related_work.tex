\section{Related Work}

LTQP is a SPARQL querying paradigm that answers queries by exploring the web using the follow-your-nose principle~\cite{hartig2016walking}.
The main challenge of LTQP is the web's open-ended nature leading to large search space.
Completeness in LTQP is defined by the traversal of a well-defined set of links~\cite{Hartig2012}.
The first method employed to define this set was the reachability criteria~\cite{Hartig2012}, which are boolean functions that determine whether a link should be dereferenced.
Building on this, the theoretical query language LDQL~\cite{hartigLDQL} was introduced, separating the traversal definition from the query definition.
Further advancements include the subweb specifications language~\cite{Bogaerts2021LinkTW}, which allows data providers to define how their DKG should be traversed.
These contributions are centered on guiding the engine in selecting links to follow in a discovery process.
However, they do not explicitly address mechanisms for restricting certain links in what we could call a pruning process.
Link pruning can be very useful to reduce the search domain of queries when information about the data model of DKG is found.

RDF data shapes are used for validating, describing, and communicating data structures, as well as generating data and driving user interfaces~\cite{Gayo2018a,Gayo2018}.
The two main RDF data shape formalisms are SHACL and ShEx.
%Both describe RDF data but differ in focus: ShEx emphasizes graph structure, while SHACL targets constraints.
For common use cases, they are equally expressive and interchangeable~\cite{Gayo2018c}.
Shape Trees~\cite{shapetreesShapeTrees} are an index structure for validating and organizing decentralized knowledge graphs (DKGs).
However, Shape Trees have not been used for query optimization. 
Due to their \emph{virtual hierarchy}~\cite{shapetreesShapeTrees}, it can be challenging for a query engine to efficiently capture the relationship between a resource IRI and its corresponding shapes.
Furthermore, because Shape Trees are not widely adopted, proposing an alternative formalization for query optimization addresses a gap in the literature.
RDF data shapes have also been used in the litterature for querying of centralized KG~\cite{kashif2021}.
Automatic generation of RDF data shapes based on KG~\cite{fernandez2023extracting} and shape-based integration of data~\cite{LabraGayo2023} are also topics that have been studied and can support shape-based summary approaches for DKGs.

Source selection is a crucial challenge in decentralized querying.
Approaches like SPARQL \texttt{SERVICE} clauses, service descriptions, basic statistics on triple counts, and histogram methods have been studied~\cite{hose2012towards, Harth2010}.
However, most of those source selection methods face the limitation of assuming a small number of data sources~\cite{Harth2010}, leaving their suitability for LTQP uncertain.
Bloom filters~\cite{dia2018fast} is also a mecanism that has with success for federated DKGs, however in the context of LTQP it has been show that bloom filters have little effect on performances~\cite{Hanski2024}
Schemas-based indexing~\cite{Stuckenschmidt2004} using ontologies has also been explored for source selection of SPARQL queries,
however the proposed approach is sensible to the high reuse of vocabulary terms in RDF~\cite{Harth2010}.