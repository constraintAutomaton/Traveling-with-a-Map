\section{Approach}

In this section, we are going to describe our approach starting with a formal definition of our query environments,
the shape index, how to express RDF data shapes into relational algebra, how to solve the query containment problem
and how to do online source selection in environments with a shape index.

\subsection{Structured linked data environments}
The web from a holistic perspective does not have a structure exploitable by the engine for query optimization.
In the web any document can be published at any location and there is no trust mechanism to guide source selection.
From a closer perspective data publisher can implicitly or explicitly organize its data publication in a way that a query engine could exploit it.
With that intuition \citeauthor{Taelman2023} propose the notion of \emph{structural assumptions}~\cite{Taelman2023} to guide query engines towards 
relevant data source given that the query engine decides to trust the data from publishers following selected web specifications.
They have shown that using this web model the execution time of queries targetting principally data concerned with the web specifications can 
be reduced to the extent where query planning becomes the bottleneck of some queries as opposed to the execution of HTTP requests~\cite{Taelman2023}.
We propose to formalize an abstract definition of \emph{structural assumptions} and to create a notion of \emph{Structured linked data environments}
to define bounded web subdomain where multiple \emph{structural assumptions} are enforced and result completeness withing this domain is defined.
A notion of results completeness is important for the comparison of the execution of queries with different reachability criteria.
Reachability criteria already defines completeness but it is from the perspective of the traversal of links in the internal data sources, hence
it cannot properly capture if a criteria is better than another to get the same results as if we traversed the whole Structured linked data environments.
For instance, using reachability criteria such as $CNone$~\cite{Hartig2012} from the perspective of traversal completeness would produce the same completeness of
any criteria able to finish the query execution, because the completeness definition is internal hence they cannot easily be compared with each other.
In the open web is it not practically possible to find an external objective notion to based on completeness~\cite{Hartig2012} however we propose that in Structured linked data environments
with the engine trusting those publishers and queries related to them it is possible.

\iffalse
While BQPs are syntactic objects, we shall use them as a represen-
tation of Linked Data queries which have a certain semantics. In the
remainder of this section we define this semantics. Due to the open-
ness and distributed nature of Webs such as the WWW we cannot
guarantee query results that are complete w.r.t. all Linked Data on
a Web. Nonetheless, we aim to provide a well-defined semantics.
Consequently, we have to limit our understanding of completeness.
However, instead of restricting ourselves to data from a fixed set
of sources selected or discovered beforehand, we introduce an ap-
proach that allows a query to make use of previously unknown data
and sources. Our definition of query semantics is based on a two-
phase approach: First, we define the part of a Web of Linked Data
that is reached by traversing links using the identifiers in a query
as a starting point. Then, we formalize the result of such a query
as the set of all valuations that map the query to a subset of all
data in the reachable part of the Web. Notice, while this two-phase
approach provides for a straightforward definition of the query se-
mantics in our model, it does not correspond to the actual query
execution strategy of integrating the traversal of data links into the
query execution process as illustrated in Section 2.
\fi

A Structured linked data environments 
\begin{equation}
E_{ld} = (SA=\{ sa_1, sa_2...sa_n\}, D)
\end{equation}
is a set of structural assumption $SA$ where structural assumption are defined at Definition~\ref{def:structuralAssumption},
with an associated set of URL $D$, call domain defining the region of the web where the structural assumptions are valid and the space where the data publisher as published its data in relation to $E_{ld}$.


\begin{definition}[Structural assumption]\label{def:structuralAssumption}
   A structural assumption $sa$ is a function such as provided a query $Q$ and an set of IRI $I_{i} \in D$ known by the engine such as $SA(Q, I_{i}) \rightarrow (I_{sa}, I_{nsa})$
   given that $I_{sa}$ is a set of IRI inside $D$ respecting a specific structure of $E_{ld}$ with regards to the known topology of the $E_{ld}$ and $I_{nsa}$ are the IRI.
\end{definition}

\subsection{Shape index}
The shape index was already defined in one of our publication~\cite{tam2024opportunitiesshapebasedoptimizationlink}, however its definition was not formalized.
A shape index $SI$ as three components, a mapping $M$ between RDF data shapes $s_i$ and set of ressources $R_i$ such as $M = \{s_0 \mapsto R_0, s_1 \mapsto R_1..., s_n \mapsto R_n\}$ given $n$ entries,
a domain of actions $D$ composed by a set of URL $ID$ and
a flag $C\rightarrow \{\mathrm{true}, \mathrm{false}\}$ indicating if the shape index has has a relation $m \in M$ for every resource $r_j$ associated with an URL $id_j \in D$. 
The shape index can be succintly defined by the following tuple 
\begin{equation}
SI = (M, D, C)
\end{equation}.

\subsection{Expressing RDF Data shapes into relational algebra}

\subsection{Query shape containment}

\subsection{Online source selection in environments with a shape index}