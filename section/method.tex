\section{Approach}

\sepfootnotecontent{sf:shapeIndexURL}{
   \ifanonymous
      \url{https://anonymous.4open.science/w/shape-index-specification-707E}
   \else
      \url{https://constraintautomaton.github.io/shape-index-specification/}
   \fi
}

\sepfootnotecontent{sf:subwebsep}{
   We assume an implicit conversion between the subweb specification and the formalization of reachability criteria.
   Space constraints prevent us from detailing the conversion.
}

\sepfootnotecontent{sf:ssf_project}{
   The paper \citetitle{delva2023} additional material also proposes a script to convert SCHACL shapes into SPARQL queries.
   \url{https://github.com/MaximeJakubowski/ssf_project}
}

\subsection{Structured Linked Data Environments and Completeness}\label{sec:slde}
From a holistic perspective, the web does not have a structure exploitable by query engines for optimization.
On the web, any document can be published at any location, and there is no index or trust mechanism to guide source selection.
From a perspective where the web is divided into small subsections controlled by data providers, data publishers can implicitly or explicitly organize their data  
in a way that query engines could exploit its organization for optimization.
The concept of \emph{structural assumptions}~\cite{Taelman2023} has been theorized to optimize queries by exploiting the local structure of decentralized RDF datasets.
These assumptions rely on data providers guaranteeing the completeness of results for their statements within a specific set of resources in an IRI domain~\cite{Bogaerts2021LinkTW}.
Structural assumptions have been shown to reduce the query execution time of multiple queries to the extent that the bottleneck of specific queries is not the number of HTTP requests but the suboptimal query plan~\cite{Taelman2023, eschauzier_quweda_linkqueue_2023}.
Structural assumptions define completeness of traversal of SLDE in the same paradigm as reachability criteria~\cite{Hartig2012}.
We propose to put our focus on the completeness of results when considering that the completeness of traversal has been fixed.
Placing our focus on results allows us to investigate ways to optimize the search space of link traversal queries.
Our approach consists of defining more restricting reachability criteria~\cite{Hartig2012} while ensuring the same completeness of results as with less restricting criteria.


We formalize our approach as follows.
A query is executed over a DKG $G$ formed by the union of all the $g \in r$ in a network $R$.
The query engine has to build a KG $G^{\prime}$ in its internal data store from the  $g \in r$ by dereferencing resources in $R$ such that
$G^{\prime} \subseteq G$ holds to answer its query.
We are trying to solve an optimization problem where another query engine builds a KG
$G^{\prime\prime} \subseteq G^{\prime}$
potentially smaller than the one produced with a specific traversal completeness policy
by defining a reachability criterion $C_{mr}$.
We focus on maintaining the same results completeness, so when using $C_{mr}$ the following equation must hold

\begin{equation}\label{eq:evalQueryStructuralAssumption}
   [\![ Q ]\!]^{G^{\prime\prime}} = [\![ Q ]\!]^{G^{\prime}}
\end{equation}
for any network $R$.

Because the $g \in G^{\prime}$ can only be acquired from the dereferencing of resources $r \in R$, a smaller $G^\prime$ implies that a lesser number of HTTP requests has been performed to answer a query.
Naturally, query execution is faster with a smaller KG instance.
Additionally, HTTP requests can be slow and unpredictable~\cite{hartig2016walking}, making them a significant factor in overall query execution time. 
Thus, the benefit of reducing the number of HTTP request is twofold.

To define less selective reachabilities to produce $G^{\prime\prime}$, we propose extending the reachability criteria by formalizing a chain of criteria in a concept called \emph{composite reachability criteria}.
In this form, a reachability criterion $cp_i$ is said to \emph{prune} links, and $cd_i$ is said to \emph{discover} links.
A reachability $cp_i \in C$ should act upon all the links that have yet to be dereferenced as well as on the incoming links.
Equation~\ref{eq:cReachabilityCriteria} formalizes a composite reachability criterion $C$ with $nd$ discovery criteria and $np$ pruning criteria.

\begin{equation}\label{eq:cReachabilityCriteria}
   C(t, iri, B) = \bigvee_{cd \in Cd} cd(t, iri, B) \mathrel{\land} \bigwedge_{cp \in Cp} \, cp(t, iri, B)
\end{equation}
where $Cd$ is the set of every $cd_i(t, iri, B)$ and $Cp$ the set of every $cp_i(t, iri, B)$ used by the engine.
\iffalse
Maybe refer to the link queue here in a footnote.
\fi

\iffalse
In the context of the World Wide Web (WWW), defining completeness based on results is not possible because it necessitate the acquisition of the whole web.
However, in the context of a decentralized dataset with structural properties and associated structural assumptions,
the traversal completeness of those datasets is defined outside of the query engine and thus can serve as a 
stable comparison concept regardless of the traversal policy of the engine.
Hence, for each dataset with structural properties, it is possible to compute a theoretical result completeness.
If we suppose a finite network of connected datasets that are discoverable with a specified reachability criterion such as \texttt{cmatch},
Then, it is possible to establish the completeness of the subset of the query targeting information in the network.



Furthermore, having a stable traversal completeness we can create a theorical results completeness metric, by considering the descentralized dataset as a centralized one that is the union 
of all the knowledge graph inside the ressources of the dataset $GD = \bigcup\limits_{i}^{n} g\in r_i$ given $n$ ressource $r$.
This, metric 

\begin{equation}\label{eq:metricResult}
   [\![ Q ]\!]^{G^{\prime}} = [\![ Q ]\!]^{G}
\end{equation}

\fi 



\iffalse
While BQPs are syntactic objects, we shall use them as a represen-
tation of Linked Data queries which have a certain semantics. In the
remainder of this section we define this semantics. Due to the open-
ness and distributed nature of Webs such as the WWW we cannot
guarantee query results that are complete w.r.t. all Linked Data on
a Web. Nonetheless, we aim to provide a well-defined semantics.
Consequently, we have to limit our understanding of completeness.
However, instead of restricting ourselves to data from a fixed set
of sources selected or discovered beforehand, we introduce an ap-
proach that allows a query to make use of previously unknown data
and sources. Our definition of query semantics is based on a two-
phase approach: First, we define the part of a Web of Linked Data
that is reached by traversing links using the identifiers in a query
as a starting point. Then, we formalize the result of such a query
as the set of all valuations that map the query to a subset of all
data in the reachable part of the Web. Notice, while this two-phase
approach provides for a straightforward definition of the query se-
mantics in our model, it does not correspond to the actual query
execution strategy of integrating the traversal of data links into the
query execution process as illustrated in Section 2.

The analysis
of this implementation approach is particularly interesting because
this approach trades completeness of query results for the guarantee
that all query executions terminate as we shall see.


Hartig2012
\fi


\subsection{Shape Index}

We propose using a decentralized dataset summary method, called the shape index~\cite{tam2024opportunitiesshapebasedoptimizationlink}, as the support mechanism for our pruning reachability criterion.
This method involves mapping the content of a decentralized dataset using RDF data shapes.
The intuition behind this approach is that rendering explicit RDF schemas is relatively inexpensive for data providers when publishing decentralized datasets.
Although RDF does not enforce schemas on its data, the nature of the usually modeled objects and the formalism often results in implicit schemas~\cite{Neumann2011CharacteristicSA}.
Additionally, the cost of making these schemas explicit should be evaluated in the context of the data's intended usage.
First, mapping and documenting decentralized data improves its discoverability, which is a common objective when publishing RDF data.
Second, embedding this information within datasets facilitates client-side query engine processing, potentially reducing reliance on costly SPARQL endpoints~\cite{aranda2013}.
Since the web is a highly dynamic environment where data providers require flexibility, shape indexes support open shapes as mapping keys.
For example, a data provider might expose personal data such as movie recommendations, bookmark metadata, and friend lists.
The provider might use open shapes for movie recommendation data because they are less structured and employ closed shapes for bookmark and friend list data.

A shape index 
\begin{equation}\label{eq:shapeIndex}
   SI = \{s_1 \mapsto IRI_1, s_2 \mapsto IRI_2 \cdots, s_n \mapsto IRI_n\}
\end{equation}
, is a mapping between RDF data shapes and a set of IRI given $n$ entries.
The domain of a shape index is defined by $D = \bigcup_{i=1}^{n} \bigcup_{iri \in IRI_i} iri$.
A shape index \emph{must} map every resource in $D$.
We denote a shape index as \emph{complete} when every shape $s_i \in \text{DOM}(SI)$ is closed and \emph{incomplete} otherwise.
A mapping between a shape and a set of IRI has implications in the distribution of the data in $D$.
When a shape $s$ is mapped to an $IRI$, then the KG targeted by the mapping $G = \{g | g \in r, iri \mapsto r \land iri \in IRI\}$ satisfy $s$.
Given that the shape is close, then every set of triple (see the end of section \ref{sec:shape}) in the resource associated with $D$ respecting the shape must be in a resource mapped to an $iri \in IRI$.
A draft web specification of the shape index is also available online.~\sepfootnote{sf:shapeIndexURL}

\subsection{Online Source Selection in Shape Index Documented Datasets}\label{sec:sourceSelection}

A decentralized dataset exposing a shape index provides a query engine with the opportunity to reduce its search domain by evaluating which resources are query-relevant.
The shape descriptions and constraints on the distribution of the data allow the query engine, 
within a domain $D$ of a shape index $SI$ to determine a priori before traversing $D$ which subdomain $d \subseteq D$ contains data that is query-relevant.
To formalize our approach, we use the concept of composite reachability criteria described in section~\ref{sec:slde} and of the source selectors of the subweb specification formalization~\cite{Bogaerts2021LinkTW, Taelman2023}.

We propose that a query engine with composite reachability $C_t$ at any time $t$ can query an SLDE with completeness of results.
$C_t$ is defined as such that if the engine were not to change its reachability
until $t_f$, the end of the query execution, the completeness of results,
as defined in Equation~\ref{eq:evalQueryStructuralAssumption},
will remain consistent with the results obtained using the reachability at the beginning of execution, $C_{0}$.
We propose that $C_0$ must have a discovery criteria $cd_{\text{shape index}}$ that follow links leading to shape indexes.
Let us consider that the predicate \texttt{si:shapeIndexLocation} indicates that the object of the triple is the location of the shape index 
of the dataset, thus we can define the criterion
$cd_{\text{shape index}} = \{o|\forall s \langle \text{si:shapeIndexLocation} \rangle \}$.~\sepfootnote{sf:subwebsep}
A reachability $C_{tsi_i}$ is created after dereferencing a shape index.
This reachability must accept the links $IRI_d$ that are query-relevant and prune the links $IRI_p$.
To determine $IRI_d$ and $IRI_p$ a query-shape containement ($\sqsubseteq_{qs}$) is performed.
The problem is defined in section \ref{sec:containment}.
We define 
$IRI_d = \{ \bigcup_{iri \in SI_i(s_j)} iri |  B \sqsubseteq_{qs}  s_j = \mathrm{true}, s_j \in \text{DOM}(SI_i) \}$ and 
$IRI_p = \{ \bigcup_{iri \in SI_i(s_j)} iri | B \sqsubseteq_{qs}  s_j = \mathrm{false}, s_j \in \text{DOM}(SI_i) \}$

given a shape index $SI_i$.
From those sets of links we define two new reachability criteria;
\begin{equation}
   \begin{aligned}
       cd_{si}(t, iri, B) &= iri \in IRI_d \\
       cp_{si}(t, iri, B) &= iri \notin IRI_p
   \end{aligned}
\end{equation}
the new reachability $C_{tsi_i}$ is created by taking the $Cd$ and $Cp$ of $C_{tsi - 1}$ and adding
$cd_{si}$ and $cp_{si}$ to the corresponding set. 

\subsection{Expressing RDF Data shapes into SPARQL algebra}\label{sec:shape2SPARQL}

Before tackling the query shape containment problem, we must define how to express RDF data shapes into queries.
A shape expression $e_i$ can be intuitively treated as a segment of a query~\cite{delva2023}.
A common approach for shape validation over an RDF graph is to convert shapes into SPARQL queries~\cite{labragayo2017validatingdescribinglinkeddata, Corman2019,Prestamo2023, spapeExpressionConvert}.~\sepfootnote{sf:ssf_project}
We define in the following lines the transformation of a shape $S$ into a query $Q_s$ using the transformation $T(S,?x)$ and $t(e_i, ?x) = q_i$ where every 
symbol with a $?$ prefix is a variable in $\mathcal{V}$.
\iffalse
For our query-shape containment problem we are looking into transforming our shapes $S$ into queries $Q_s =  q_i \bowtie q_{i+1} ... \bowtie q_n$ by transforming each $e_i$ of $s$ in a $q_i$.
We are looking into creating a query $Q_s$ with the following property
\begin{equation}\label{eq:shapeSPARQL}
   G \models S \iff [\![ Q_s ]\!]^{G} =  G
\end{equation}
For every $G$.

Our definition is aimed at close shapes because we are not looking into validating data but into containment since an open shape can validate 
For any $G$ that contains at least the constraint specified by $S$, we simplify the fact that if we neglect negative statements, any query can be contained in such a shape.
The intuition behind equation~\ref{eq:shapeSPARQL} is that we are looking to produce a query that simulates the behavior of a shape.
It is natural from the definition of shape languages. It has been shown in the literature that shape constraint can be transformed into a combination of triple patterns, filter expressions, and other SPARQL fragments.
Thus, if we can establish a mapping between the constraints of $S$ in terms of query fragments, a query targeting $G$ should return $G$ if every triple of the 
$G$ respects the constraint of the shape.
\fi


\iffalse
INSPIRE YOURSELF BY Corman2019 style

We might want to reference that

https://labra.weso.es/publication/2023_kg_subsets_pregel/
https://labra.weso.es/publication/2024_extracting_shapes_consolidator/

https://labra.weso.es/publication/2023_rdf_data_integration/

ssf_project Maxime

https://github.com/MaximeJakubowski/ssf_project/blob/master/ssf/sparql_conformance.py

\fi

\begin{prop}\label{prop:triplePattern}
   A shape expression $e_i = st$ given $rs^a_b = st_3$ and $p$ is the predicate from $pc_p = st_1$
   can be transformed into a query consisting of one triple pattern with an associated \texttt{GROUP BY ?x HAVING(COUNT(?o))} statement on the object of triple pattern,
   such that
   $Q_{e_i} = \beta_{?x, (|?o| \in rs^a_b)}(?x, p, ?o)$,
   where $\beta$ is a \texttt{GROUP BY ?x HAVING} clause and $|?o|$ is a \texttt{COUNT} clause. 
\end{prop}

\begin{prop}
   In a shape expression $e_i$ , $oc_f = st_2$ can be transformed into a filter expression targeting $?o$ if the constraint validates an RDF type or a literal comparison.
   This transformation takes the form $q_{i} = \sigma_{oc_f((?x, p, ?o))}(?x, p, ?o)$ where $\sigma$ represent the filter expression applid on $?o$.
   If $oc_f$ validates a shape on $?o$, then $q_i$ must be formed by joining a query derived from $e_i$ with $oc_f = \mathrm{true}$ and a query derived from the shape targeted by $oc_f$ 
   where $?o$ is the subject such that $q_{i}= (?x, p, ?o) \bowtie T(S_{oc_f},?o)$ given $S_{oc_f}$ is the shape related to $oc_f$.
\end{prop}

\begin{prop}
   In a shape expression $e_i = e_j|e_k$ the expression can be transform into the union between two queries
   $q_{i} = t(e_j, ?x) \cup t(e_k, ?x)$.
\end{prop}

\begin{prop}
   In a shape expression $e_i = st$ when $st_4 = n = \mathrm{true}$ then the query $q_i$ must be negated such that $\neg q_i$ and $\beta$ statements ignored.
\end{prop}

\begin{prop}
   A closed shape ($op = \mathrm{false}$) $S$ can be transform into $Q = t(e_1, ?x) \bowtie t(e_2, ?x) ... \bowtie t(e_{n_{ec}}, ?x)$ given $n_{ec}$ shape expressions.
\end{prop}

\begin{prop}
   A open shape $S$ ($op = \mathrm{true}$) is transform into $Q = (?x, ?p, ?o) -  (t(e_1, ?x) \bowtie  t(e_2, ?x) ... \bowtie t(e_{n_{eo}}, ?x)) $
   given $n_{eo}$ negative $e_i$ (where $n = \mathrm{true}$).
\end{prop}

\subsection{Query Shape Containment}\label{sec:containment}

We propose using query containment as a source selection mechanism.
While query containment is intuitively unsuitable for online source selection due to its NP-complete complexity~\cite{Spasi2023}, this complexity depends on the size of the queries, which are typically small.
Moreover, practical use cases can leverage polynomial time complexity algorithms~\cite{Doan2012}.
In our context, the container query (the shape) follows a tree star pattern template structure where predicates are always IRIs, and subjects and objects are always variables.
This structure can be leverage to create an algorithm with polynomial time complexity algorithms.

We consider that a query $Q$ is contained into a shape $S$ denoted $Q \sqsubseteq_{qs} S$, if a tree star pattern in $Q$ is contained in $Q_s = T(S, ?x)$ where $?x$ is the variable of the subject of the body of the shape.
For a tree star pattern to be contained, we need to consider the two parts of $Q = Q_{body} \cup Q_{\text{unions}}$.
$Q_{body}$ is the Basic Graph Pattern (BGP) of the query and $Q_{\text{unions}} = \bigcup Q_u$ are the Union Graph Patterns (UGP) where $Q_u = q_0 \cup q_1 \cup q_2 ... \cup q_n$.
A tree star pattern $Q_{starT_i}$ is contained in $Q_s$ if its segment in $Q_{body}$ is contained in $Q_s$ and if its segment in at least one $q_i$ in each $Q_u$ is contained in $Q_s$.
If $Q_{starT_i}$ is not a part of an $Q_u$, then the $Q_u$ is ignored.
In our containment problem we ignore "cardinality constraint" which are translated in 
fragments of the form \texttt{GROUP BY ?x HAVING(COUNT(?o))}.
We ignore these segments because we are not attempting to identify sources that can fully answer a particular segment of a query.
Instead, we attempt to disregarding data sources that are irrelevant to the query, given the constraints imposed by the shape index.
Thus, discriminating based on cardinalities might impact query results.
Additionally, in this paper, we do not consider filter expressions.

\iffalse
While evaluating filter expressions would increase the algorithm's time complexity, it would also improve the precision of source selection.
However, ignoring them does not affect the completeness of results in our context.
Therefore, the tradeoff of including them in the resolution warrants an analysis.
\fi

% https://en.wikipedia.org/wiki/Master_theorem_(analysis_of_algorithms)
\begin{algorithm}[h]
   \caption{Determine if a tree star pattern is contained ($isContain_{T}$)}\label{alg:containmentTree}
   \begin{algorithmic}
      \REQUIRE  $Q_{star}$, $Q_{starT_i}$, $Q_s = Q_{s\text{body}} \cup Q_{s\text{unions}}$ and $MS$
      \ENSURE \TRUE $ $ or \FALSE $ $ whether the tree star is contained in the shape

      \IF{$S_{star}(Q_{star}) \in MS$}
         \RETURN \TRUE
      \ENDIF 

      \FORALL{$tp \in Q_{star}$} % O(n_star)
         \IF{\NOT $match(tp, Q_{s\text{body}})$}
            \STATE $hasOnePath \gets $ \FALSE
            \FORALL{$q_{us} \in Q_{s\text{unions}}$}
               \IF{$isContain_{T}(Q_{star}, Q_{starT_i}, q_{us}, MS)$}
                  \STATE $hasOnePath \gets $ \TRUE
               \ENDIF
            \ENDFOR
            \IF{\NOT $hasOnePath$}
               \RETURN \FALSE
            \ENDIF
         \ELSE
            %eval nested
            \IF{$O(tp) \in  S_{star}(Q_{starT_i})$}
               \IF{\NOT $isContain_{T}(Q_{star_{O(tp)}} \in Q_{starT_i}, Q_{starT_i}, Q_s, MS)$}
                  \RETURN \FALSE
               \ENDIF
            \ENDIF
            %
         \ENDIF
      \ENDFOR

      \STATE $MS \gets MS \cup S_{star}(Q_{star})$
      \RETURN \TRUE
   \end{algorithmic}
\end{algorithm}


We define the function $isContain_{T}$ defined in Algorithm~\ref{alg:containmentTree} to evaluate if a $Q_{star_i}$ is contained in $Q_s$.
The $match$ function evaluates if a triple pattern independent of the variable's names is in a query.
The algorithm is recursive, and on its first call, $Q_{star}$ must be the root of the $Q_{starT_i}$, and $MS$ must be an empty set, 
this set is used not to duplicate containment calculations.
We notice that the complexity of Algorithm~\ref{alg:containmentTree} is $O( n_o \times n_{tp}^2 \times n_{sunion}^2)$
where $n_{tp}$ is the number of triple patterns of $Q_{star_i}$, $n_{sunion}$ the number of UGP in $Q_s$ and $n_o$ the number of object term in $Q_{starT_i}$.
To solve $Q \sqsubseteq_{qs} S$, we need to consider the number of tree stars from the BGP with their number of segments in the UGP and the number of BGPs in the UGPs.
This operation results again in a polynomial time complexity algorithm.
%To solve $Q \sqsubseteq_{qs} S$, we need to consider the $n_{starT}$ tree star from the BGP with their $n_{starTu}$ segment in the UGP and $n_{starTui}$ BGP in the UGP.
%This operation results in a polynomial time complexity of $O(n_o \times n_{tp}^2 \times n_{union}^2 \times n_{starT} \times n_{starTu} \times n_{starTui})$.
In the \nameref{sec:appendix} Algorithm~\ref{alg:containment} present the full resolution.
