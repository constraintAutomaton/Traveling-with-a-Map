\section{Approach}

In this section, we are going to describe our approach starting with a formal definition of our query environments,
the shape index, how to express RDF data shapes into relational algebra, how to solve the query containment problem
and how to do online source selection in environments with a shape index.

\subsection{Structured linked data environments}

\subsection{Shape index}
The shape shape index was already defined in one of our publication~\cite{tam2024opportunitiesshapebasedoptimizationlink}, however its definition was not formalized.
A shape index $SI$ as three components, a mapping $M$ between RDF data shapes $s_i$ and set of ressources $R_i$ such as $M = \{s_0 \mapsto R_0, s_1 \mapsto R_1..., s_n \mapsto R_n\}$ given $n$ entries,
a domain of actions $D$ composed by a set of URL $ID$ and
a flag $C\rightarrow \{\mathrm{true}, \mathrm{false}\}$ indicating if the shape index has has a relation $m \in M$ for every resource $r_j$ associated with an URL $id_j \in D$. 
The shape index can be succintly defined by the following tuple $SI = (M, D, C)$.

\subsection{Expressing RDF Data shapes into relational algebra}

\subsection{Query shape containment}

\subsection{Online source selection in environments with a shape index}