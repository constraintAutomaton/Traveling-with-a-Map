\section{Approach}

\sepfootnotecontent{sf:shapeTarget}{
   It has to be noted that the concept of target and map of SHACL and ShEx does not apply, since it is always the whole knowledge graph that has to be valid.
}

\sepfootnotecontent{sf:shapeIndexURL}{
   \url{https://constraintautomaton.github.io/shape-index-specification/}
}

\subsection{Structured Linked Data Environments and Completeness}
From a holistic perspective, the web does not have a structure exploitable by query engines for optimization.
On the web, any document can be published at any location, and there is no index or trust mechanism to guide source selection.
Furthermore, there is no official mechanism for providing summary information about the content of data sources before their acquisition.
From a local perspective, a perspective where the web is divided into small subsections controlled by data providers, data publishers can implicitly or explicitly organize their data  
in a way that a query engine could exploit its organization for query optimization.
The concept of \emph{structural assumptions}~\cite{Taelman2023} has been theorized to optimize queries by exploiting the local structure of decentralized RDF datasets where the data providers 
guarantee the completeness of results related to their statement in a specific set of resources in an IRI domain~\cite{Bogaerts2021LinkTW}.

Structural assumptions have been shown to reduce the query execution time of multiple queries to the extent that the bottleneck of specific queries is not the number of HTTP requests but the suboptimal query plan~\cite{Taelman2023, eschauzier_quweda_linkqueue_2023}.
We propose to reduce the number of HTTP request furthermore by using more restricting reachability criteria~\cite{Hartig2012} while keeping the same completeness of results.
We formalize our general approach as follow.
A query is executed over a DKG $G$ formed by the union of all the $g \in r$ in the network $R$.
The query engine has to build a KG $G^{\prime}$ in its internal data store from the  $g \in r$ by dereferencing resources in $R$ such that

\begin{equation}\label{eq:subsetDKG}
G^{\prime} \subseteq G
\end{equation}

and 

\begin{equation}\label{eq:evalQueryStructuralAssumption}
   [\![ Q ]\!]^{G^{\prime}} = [\![ Q ]\!]^{G}
\end{equation}
.
Because the $g \in G^{\prime}$ can only be acquired from the dereferencing of resources $r \in R$, a smaller $G^\prime$ implies that a lesser number of HTTP requests has been performed to answer a query
returning the same results.
HTTP requests can be slow and unpredictable~\cite{hartig2016walking}; thus, executing fewer requests to answer a query can lead to a reduction of query execution time.
Additionally, a smaller internal data store leads to fewer triples to process during the joining operations, which can further contribute to the reduction of query execution time.

To define our less selective reachability, we propose extending the reachability criteria to chain them in a concept called \emph{Composite reachability criteria} as they are boolean functions.
In this form, a reachability criterion ($cp_i$) is said to \emph{prune} links if it is chained with an \emph{and} operator, and ($cd_i$) is said to \emph{discover} links if it chain with an \emph{or} operator.
Equation~\ref{eq:cReachabilityCriteria} formalize a composite reachability criterion $C$ with $nd$ discovery criteria and $np$ prunning criteria.

\begin{equation}\label{eq:cReachabilityCriteria}
    C(t, i_d, B)  = \bigvee_{i=0}^{nd}cd_i(t, i_d, B) \land \bigvee_{j=0}^{np}cp_j(t, i_d, B)
\end{equation}

We aim to define less restrictive criteria by appending a composite reachability criterion with a pruning criterion, taking advantage of data summaries captured during the traversal.

\iffalse
In the context of the World Wide Web (WWW), defining completeness based on results is not possible because it necessitate the acquisition of the whole web.
However, in the context of a decentralized dataset with structural properties and associated structural assumptions,
the traversal completeness of those datasets is defined outside of the query engine and thus can serve as a 
stable comparison concept regardless of the traversal policy of the engine.
Hence, for each dataset with structural properties, it is possible to compute a theoretical result completeness.
If we suppose a finite network of connected datasets that are discoverable with a specified reachability criterion such as \texttt{cmatch},
Then, it is possible to establish the completeness of the subset of the query targeting information in the network.



Furthermore, having a stable traversal completeness we can create a theorical results completeness metric, by considering the descentralized dataset as a centralized one that is the union 
of all the knowledge graph inside the ressources of the dataset $GD = \bigcup\limits_{i}^{n} g\in r_i$ given $n$ ressource $r$.
This, metric 

\begin{equation}\label{eq:metricResult}
   [\![ Q ]\!]^{G^{\prime}} = [\![ Q ]\!]^{G}
\end{equation}

\fi 



\iffalse
While BQPs are syntactic objects, we shall use them as a represen-
tation of Linked Data queries which have a certain semantics. In the
remainder of this section we define this semantics. Due to the open-
ness and distributed nature of Webs such as the WWW we cannot
guarantee query results that are complete w.r.t. all Linked Data on
a Web. Nonetheless, we aim to provide a well-defined semantics.
Consequently, we have to limit our understanding of completeness.
However, instead of restricting ourselves to data from a fixed set
of sources selected or discovered beforehand, we introduce an ap-
proach that allows a query to make use of previously unknown data
and sources. Our definition of query semantics is based on a two-
phase approach: First, we define the part of a Web of Linked Data
that is reached by traversing links using the identifiers in a query
as a starting point. Then, we formalize the result of such a query
as the set of all valuations that map the query to a subset of all
data in the reachable part of the Web. Notice, while this two-phase
approach provides for a straightforward definition of the query se-
mantics in our model, it does not correspond to the actual query
execution strategy of integrating the traversal of data links into the
query execution process as illustrated in Section 2.

The analysis
of this implementation approach is particularly interesting because
this approach trades completeness of query results for the guarantee
that all query executions terminate as we shall see.


Hartig2012
\fi


\subsection{Shape Index}
The summary that we propose is a Shape Index.
It consists of mapping the content of a decentralized dataset with RDF data shapes.
Because the web is a highly dynamic environment and data providers might need a high degree of flexibility, it is possible to use open shapes or to map part of the dataset with shapes.
In practice, a shape index could look something like this: A data provider exposes personal data containing movie recommendations, bookmark metadata, friendliest data, and various personal data.
The data provider might have a more or less structured movie recommendation sub-dataset; thus, they use an open shape; they might have structured bookmarks and friendliest, thus they 
use close shapes, and their data might change drastically over time, and part might have a non-trivial privacy policy; thus, they decide not to map it with a shape. 

The concept of a shape index was already described in the publication \citetitle{tam2024opportunitiesshapebasedoptimizationlink}. 
However, it was not formalized.
A shape index 
\begin{equation}\label{eq:shapeIndex}
   SI = (M, D)
\end{equation}
, is a tuple where $M = \{s_1 \mapsto IRI_1, s_2 \mapsto IRI_2 \cdots, s_n \mapsto IRI_n\}$ is a mapping between RDF shapes and a set of IRI 
and where $D = IRI_0 \cup (\bigcup\limits_{i=1}^{n} IRI_i)$ is the domain of the index.
$IRI_0$ is the set of IRI that is inside of the domain of the index but where the IRI is not mapped in $M$
such that $iri \in IRI_0 \iff \nexists iri \in M(s_i)$.
When $IRI_0 = \emptyset$, then the shape index describes each resource of its domain and is denoted as a \emph{complete shape index}. 

A mapping between a shape and a set of IRI has implications in the distribution of the data in $D$.
when a shape is mapped to a set of $IRI$ then every triple of the knowledge graph $g^{\prime}$ such as $iri \mapsto g^{\prime} \in r \land iri \in IRI$
are validated by the shape such that $SE(s, g^{\prime}) = \text{true}$.~\sepfootnote{sf:shapeTarget}
Given that the shape mapping a set of IRI $IRI_i$ is close, then every set of triple in the resource associated with $D$ respecting the shape must be in a resource mapped to an IRI in $IRI_i$, 
such that given $G = \bigcup\limits_{j=1}^{n} g_j$ where $g_j$ is the knowledge graph containing every triple inside of the document with the IRI in the mapping $M$
then $SE(s, g) = true \iff g \mapsto iri \in IRI$.
A web specification of the shape index is also available online.~\sepfootnote{sf:shapeIndexURL}

\subsection{Online source selection in environments with a shape index}
To make use of the shape index...

\subsection{Expressing RDF Data shapes into SPARQL algebra}

A shape expression $e_i$ can be intuitively treated as a segment of a query~\cite{delva2023} more precisely of a star pattern with dependencies query.
A common approach for shape validation over an RDF graph is to convert shapes into SPARQL queries~\cite{labragayo2017validatingdescribinglinkeddata, Corman2019, spapeExpressionConvert}.
In this paper we ignore recursive shapes because they cannot be expressed with a single SPARQL query~\cite{Corman2019}. 

We are looking into transforming every $e_i$ into a $q_i$ such that an RDF data shape $S$ would be transform into a $Q_s = q_i \bowtie q_{i+1} ... \bowtie q_n$.
$Q$ must have the property that given a KG $G$ then 

\begin{equation}\label{eq:shapeSPARQL}
   SE(S,G) = \mathrm{true} \iff [\![ Q_s ]\!]^{G} =  G
\end{equation}
For every $G$.
Our definition is aimed at close shapes because we are not looking into validating data but into containment, since an open shape can validate 
for any KG that contains at least the constraint specified by $S$, we consider that if we neglect negative statements, any query can be contained in such a shape.
The intuition behind equation~\ref{eq:shapeSPARQL} is that we are looking to produce a query that simulates the behavior of a shape.
A shape can only return $\mathrm{true}$ or $\mathrm{false}$ based on whether or not $G$ is validated by it.
It is natural to consider that the constraints of the shapes combine triple patterns and filter expression (and having clauses) segments.
Thus, if we can establish a mapping between those formulations, a query targeting a KG $G$ should return $G$ if every triple of the 
KG respects the constraint of the shape.

We define in the following lines the transformation of a shape $S$ into a query $Q_s$ using the transformation $T(S,?x)$ and $q = t(e, ?x)$ where every 
symbol with a $?$ prefix is a variable in $\mathcal{V}$. 


\iffalse
INSPIRE YOURSELF BY Corman2019 style

!!add having count in the query definition for the cardinalities!!
https://www.w3.org/TR/sparql11-query\#sparqlHavingClause
\fi

\begin{prop}\label{prop:triplePattern}
   In a shape expression $e_i$, $p$ and $rs$ can be transformed into a query consisting of one triple pattern with an associated \texttt{HAVING(COUNT(?o))} statement on the object of triple pattern,
   such that $Q_{e_i} = \beta_{(|?o| \geq rs[0] \land |?o| > rs[1])}(?x, p, ?o)$, where $\beta$ is a \texttt{HAVING} clause and $|?o|$ is a \texttt{COUNT} clause. 
\end{prop}

\begin{prop}
   In a shape expression $e_i$ , $c$ can be transformed into a filter expression targeting $?o$ if the constraint is a comparison or enforce a type on an object such that
   $q_{i} = \sigma_{c(?o)}(?x, p, ?o)$. 
   If $c$ enforces a shape on $?o$, then $q_i$ must be formed by the join of a query derived by $e_i$ with $c = \mathrm{true}$ and a query derived from the shape targeted in $c$ where $?o$ is the subject 
   such that $q_{i}= (?x, p, ?o) \bowtie T(S_c,?o)$ given $S_c$ is the shape related to $c$.
\end{prop}

\begin{prop}
   In a shape expression $e_i = e_j|e_k$ the expression can be transform into the union between two queries
   $q_{i} = t(e_j, ?x) \cup t(e_k, ?x)$.
\end{prop}

\begin{prop}
   In a shape expression $e$ when $n= \mathrm{true}$ then the query $q_i$ must be negated such that $\neg q_i$.
\end{prop}

\begin{prop}
   A shape closed shape ($op = \mathrm{false}$) $S$ can be transform into $Q = t(e_1, ?x) \bowtie t(e_2, ?x) ... \bowtie t(e_{n_e}, ?x)$ given $n_e$ shape expressions.
\end{prop}

\begin{prop}
   A open shape $S$ ($op = \mathrm{true}$) is transform into $Q = (?x, ?p, ?o) -  (t(e_1, ?x) \bowtie  t(e_2, ?x) ... \bowtie t(e_{n_e}, ?x)| n\in e_i = \mathrm{true}) $ where $Q_{e_i}$.
\end{prop}

\subsection{Query shape containment}

To perform our source selection, we compute a query shape containment problem. 
This problem involves determining if a star pattern with dependencies $q_i$ from a query $Q = q_1 \bowtie q_2 ... \bowtie q_n$ is contained inside a shape transformed into a query $Q_s$. 
The previous section shows that $Q_s$ follows a query template of a star pattern with dependencies, where the subjects and objects are always variables, and the predicate is always an IRI.
In this paper, we will ignore the filter expressions and clauses and use this query template to solve the query containment problem.
We are presenting two variances of the query shape containment problem, one for general source discovery that we denote \emph{loose} containment $\sqsubseteq_l$ and 
another denote \emph{strict} $\sqsubseteq$ that could be used for determining information similar to exclusive group~\cite{Schwarte2011}. 
Intuitively, the query containment problem does not seem suited for online source selection because its complexity is NP-complete~\cite{Spasi2023}.
However, the complexity is in terms of the size of both queries, which tend to be small.
Additionally, in practical use cases, polynomial algorithms exist~\cite{Doan2012}; 
thus, it is a valid mechanism for source selection, and our experiment results will further demonstrate this proposition.

\begin{definition}[Loose containment]\label{def:looseContainment}
We define a losse containment between a query $q_i$ and a shape $Q_s$ as determining if one path $q_{pi} \subseteq q_i$ of the query $q_i = \bigcup_{i=0}^{n} q_{pi}$ given $n$ path is contained in 
$Q_s$ such has $\exists (q_{pi} \sqsubseteq  Q_s) = q_{i} \sqsubseteq_l  Q_s$.
\end{definition}

The loose containment problem is used to determine if part of a query is contained in shape,
it is necessary to determine if at least one path is contained inside a shape to avoid losing potential partial results.
Due to the exclusivity constraint of the shape index, the engine has a guarantee that a path $q_{pi} \sqsubseteq Q_s$ will be located at a specific location in the domain of a shape index.
A simpler possibility for source selection without avoiding the loss of partial results is to use property matching;
however, this method is less selective because RDF vocabulary tends to be reused by multiple entities~\cite{Stuckenschmidt2004, Harth2010},
which can lead to selecting multiple sources that would be query-relevant.
There are some particularities that property paths introduce in the query containment problem~\cite{Kostylev2015}.
As presented in the preliminaries, most property paths can be rewritten into SPARQL algebra; with the use of an alternative path "$/$", multiple unions can be added to the query.
The property that changes the most the query containment problem is the negative path ("$!$")~\cite{Kostylev2015},
however, in our context, negative paths have little effect because we are looking for one path that is contained inside $Q_s$.

\iffalse
LET MAKE AN ALGORITHM and analyse the complexity
\fi


\begin{definition}[Strict containment]\label{def:strictContainment}
We define a strict containment between a query $q_i$ and a shape $Q_s$ when $q_i$ is contain in $Q_s$ such has $q_{i} \sqsubseteq Q_s$.
\end{definition}

The version of the query containment is algorith have the same definition than a classic query containment problem, however due to the 
particular query template of $Q_s$, the complexity of the problem is simpler.

This problem could be used in the context of federated query to determine exclusive groups, given that the federation exposes a shape index, 
also, in the context of LTQP, where some data source would have a SPARQL interface, this concept of containment could be used to send a large segment of the query to the endpoint 
minimizing communication and joining operation costs of the client engine.

With this version, the negative path also has a lesser impact than the noted typical query containment problem because the container query 
always have a single IRI as its predicate; hence, a $q_i$ using a negated property path will be contained in $Q_s$.

\iffalse
The query shape containment problem is similar to the query containment problem as it 
try to determine if the possible answers of the star pattern with dependencies query are contained in any instance of data respecting an RDF data shape.
We can divide the query shape contaiment problem into two category the \emph{strict} and \emph{loose} containment.
The difference between the problems lies into their handling of the union of queries, optional triple patterns and cardinality of property path.


We consider an RDF data shapes as start pattern with dependencies such as 
\begin{equation}
   Q_{shape} = Q_{body} \bigvee Q_{nested body}
\end{equation}
because the subject of a shape is always a single variable 

CITE FOR PROPERTY PATH PARTICULARLY THE NEGATION \cite{Kostylev2015}


The shape index by solving the query containment problem can be expressed as a structural assumption.
... define the algorithm
\fi