\section{Approach}

In this section, we are going to describe our approach starting with a formal definition of our query environments,
the shape index, how to express RDF data shapes into relational algebra, how to solve the query containment problem
and how to do online source selection in environments with a shape index.

\subsection{Structured linked data environments}
Our method is intented to optimize environement with structured assumptions~\cite{Taelman2023}.
For that purpose extend and formalize the defintion from \citeauthor{Taelman2023} in a new concept that we call \emph{Structured linked data environments} $SLDE$.
A SLDE is a graph of ressources $R_i$ located at URL $id_i$ such as it can be possible to traverse from $R_i$ to $R_j$ by dereferencing an RDF term in $R_i$ $id_j$.
The SLDE is explicitly delimited by a set of URL $D$.
A SLDE contain a set of structural assumption $SA$ defined in Defintion~\ref{def:structuralAssumption} such as 
it is possible to discover to discover every $R_i \in SLDE$ starting from any $i_j$ using an arbitrary long combination of $sa$.


\begin{definition}[Structural assumption]\label{def:structuralAssumption}
   A structural assumption $sa$ is a function such as provided a query $Q$ (or a fragment of a query) and an IRI such as $id_i$ $SA(Q, id_i) \rightarrow I_{sa}$
   the structural assumption provide a set of IRI leading to the resources withing the SLDE.
\end{definition}

The purpose of this definiton is to exteriorise the completness of the of results outside of the query engine to enable a more coherent comparaison
between composite reachability criterion.

\subsection{Shape index}
The shape index was already defined in one of our publication~\cite{tam2024opportunitiesshapebasedoptimizationlink}, however its definition was not formalized.
A shape index $SI$ as three components, a mapping $M$ between RDF data shapes $s_i$ and set of ressources $R_i$ such as $M = \{s_0 \mapsto R_0, s_1 \mapsto R_1..., s_n \mapsto R_n\}$ given $n$ entries,
a domain of actions $D$ composed by a set of URL $ID$ and
a flag $C\rightarrow \{\mathrm{true}, \mathrm{false}\}$ indicating if the shape index has has a relation $m \in M$ for every resource $r_j$ associated with an URL $id_j \in D$. 
The shape index can be succintly defined by the following tuple $SI = (M, D, C)$.

\subsection{Expressing RDF Data shapes into relational algebra}

\subsection{Query shape containment}

\subsection{Online source selection in environments with a shape index}