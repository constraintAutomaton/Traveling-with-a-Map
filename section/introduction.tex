\section{Introduction}
\sepfootnotecontent{sf:webID}{
    \url{https://www.w3.org/wiki/WebID}
}
\sepfootnotecontent{sf:dataSovereignty}{
    \url{https://digital-strategy.ec.europa.eu/en/policies/strategy-data}
}

Data sovereignty seeks to establish a more just definition of personal data ownership in terms of data usage and storage.
It can be defined as ``the self-determination of individuals and organizations concerning to the use of their data''~\cite{verstraete2022solid},
which in practice can be interpreted as the power to choose where one's data is stored and who has access to it~\cite{verstraete2022solid}.
Multiple studies have denoted problems of ownership, democracy, reinforcement of inequality, and antagonism between users and owners of web social applications~\cite{Terranova2000FreeLP, Curran2016ch1, Sevignani2013, 9663788}.
Several authors consider decentralizing web data an insufficient solution~\cite{9663788, Curran2016ch1}; yet, it is an integral component of initiatives focused on data sovereignty,
which necessitates technical research.
Linked Data enables the creation of Decentralized Knowledge Graphs (DKGs) through the use of dereferencable IRIs.
These IRIs allow access to additional Knowledge Graphs (KGs) containing information relevant to what an IRI identifies.
For example, a WebID~\sepfootnote{sf:webID} represents a user, dereferencing it can provide the name of the user, among other information, without having to store the information locally.
Despite these benefits, most SPARQL query processing occurs in centralized setups, partly because there is a better understanding of how to optimize queries in centralized environments.

The Link Traversal Query Processing (LTQP)~\cite{Hartig2012} is more natural paradigm for querying in decentralized environments,
which leverages the potential descriptive power of IRI dereferencing.
LTQP involves recursively dereferencing IRIs discovered in a query engine's internal triple store during query execution to expand its base of information.
The main difficulty of LTQP is the large domain of exploration, which leads to a high number of HTTP requests as demonstrated by \citeauthor{hartig2016walking}~\cite{hartig2016walking}.
From another perspective, and without contradicting \citeauthor{hartig2016walking}~\cite{hartig2016walking}, \citeauthor{Taelman2023}~\cite{Taelman2023} has demonstrated that in Decentralized Environments with Structural Properties (DESPs), it is possible to attain query completeness for various types of practical queries within acceptable execution times for the context of social media applications~\cite{nielsen1993response}.
Moreover, they showed that query planning could significantly influence the execution time.
Structural properties ensure data discoverability, which in turn helps guarantee result completeness by making \emph{structural assumptions}.
In practice, DESPs emerge in different context, such as personal data, social networks, and more.
Concrete DESPs has been shown to be beneficial for datasets following the Solid protocol~\cite{Taelman2023} and the TREE specification~\cite{tam_iswc_traversalsensortree_2024}.
The work of \citeauthor{Taelman2023}~\cite{Taelman2023} indicates that there are multiple optimizations possible in LTQP in the context of decentralized environments with structural properties as opposed to the
more pessimist conclusion of the work of \citeauthor{hartig2016walking}~\cite{hartig2016walking} in the context of Open Linked Data without structural properties.

In general, the Web does not have a structure exploitable by query engines for optimization.
That is because any document can be published at any location, and there is no standard index or trust mechanism to guide discovery.
However, when we consider subsections of the Web that are controlled by specific data providers as \emph{subwebs},
implicit or explicit data organizations with specific structural properties can emerge
which can be exploited by query engines.
In this work, we build upon a new dataset summarization approach for decentralized environments called a \emph{shape index}~\cite{tam2024opportunitiesshapebasedoptimizationlink},
to enable pruning links within LTQP.
This method involves mapping the content of a decentralized dataset using RDF data shapes.
The intuition behind this approach is that publishing explicit shapes is relatively inexpensive for data providers when publishing decentralized datasets but they could be highly beneficial for clients.
Even if such shapes are not defined explicitly, they often emerge in practise~\cite{Neumann2011CharacteristicSA}.

This paper is organized as follows: first, we discuss the \hyperref[sec:related_work]{related work} and present \hyperref[sec:preliminaries]{preliminaries}
We then describe our \hyperref[sec:approach]{approach}, we then introduce the \hyperref[sec:problem_statement]{problem statement}, followed by the \hyperref[sec:experiment]{experimental setup} and the \hyperref[sec:result]{discussion of results}.
Finally, we conclude the article.
