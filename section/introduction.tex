\section{Introduction}
\sepfootnotecontent{sf:webID}{
    \url{https://www.w3.org/wiki/WebID}
}
\sepfootnotecontent{sf:dataSovereignty}{
    \url{https://digital-strategy.ec.europa.eu/en/policies/strategy-data}
}

Data sovereignty seeks to establish a more just definition of personal data ownership in terms of data usage and storage.
It can be defined as ``the self-determination of individuals and organizations concerning to the use of their data''~\cite{verstraete2022solid},
which in practice can be interpreted as the power to choose where one's data is stored and who has access to it~\cite{verstraete2022solid}.
Multiple studies have denoted problems of ownership, democracy, reinforcement of inequality, and antagonism between users and owners of web social applications~\cite{Terranova2000FreeLP, Curran2016ch1, Sevignani2013, 9663788}.
Several authors consider decentralizing web data an insufficient solution~\cite{9663788, Curran2016ch1}; however, it can still be an integral component of initiatives focused on data sovereignty.
Thus, technical research on that topic is a relevant endeavor.

Linked data has the potential to create seamless decentralized databases through the use of dereferencable IRIs.
These IRIs allow access to additional databases containing information relevant to what an IRI identifies.
For example, dereferencing a term representing a user like a WebID~\sepfootnote{sf:webID} can provide the name of the user, among other information, without having 
to store the information locally.
Despite these potential advantages, most SPARQL query processing is conducted using centralized setups or federations of endpoints, partly due to performance issues with traversal queries.

A query paradigm called Link Traversal Query Processing (LTQP)~\cite{Hartig2012} has been developed to leverage the potential descriptive power of IRI dereferencing.
LTQP consists of recursively dereference IRI contained in a query engine's internal data source during query execution to expand its base of information.
To manage the scope of exploration, a lookup policy can be employed to limit the search domain of the query.
LTQP has multiple difficulties, such as the open endlessness of the web, which can be interpreted as a pseudo-infinite domain of exploration,
and the challenge of creating an efficient query plan with the lack of information about the data sources. 
Research by \citeauthor{hartig2016walking} has demonstrated that in the open web, the primary performance bottleneck and obstacle to query completeness and rapid execution is the large number of HTTP requests required to fulfill a query.

From another perspective, and without contradicting \citeauthor{hartig2016walking}, \citeauthor{Taelman2023} has demonstrated that in Structured Linked Data Environments (SLDE),
it is possible to attain query completeness for various types of practical queries.
Moreover, they showed that query planning could significantly influence the execution time.
An SLDE is defined as an RDF environment where specifications in addition to the RDF principles guarantee the completeness of results.
This guarantee of completeness has the positive side effect of providing information
that can be used to create a lookup policy to reduce the number of HTTP requests necessary to attain completeness.
In practice, SLDE can be used in the context of personal data and social networks, among others,
examples of environments that adhere to these constraints are datasets following the Solid protocol~\cite{Taelman2023} and the TREE specification~\cite{tam_iswc_traversalsensortree_2024}.
The work of ~\citeauthor{Taelman2023} indicates that there are multiple optimizations possible in LTQP in the context of SLDE as opposed to the
more pessimist conclusion of the work of ~\citeauthor{hartig2016walking}.
In line with this research direction, this paper formalizes the concepts of the \emph{Shape Index} and \emph{query-shape containment}~\cite{tam2024opportunitiesshapebasedoptimizationlink} to enable a query and data-aware mechanism for reducing the search domain of link traversal queries.
\iffalse
The shape index concept relies on RDF data shapes; the conceptual idea of RDF data shapes is to describe the properties of an entity.
We propose to use them in an index to describe decentralized datasets.
Because shapes and queries share similarities, we propose transforming shapes into queries to perform a query containment problem
and assist our source selection.
\fi

To guide our study, we formulated the following research question:
Can a link traversal query engine use shape indexes in networks of decentralized datasets to reduce the number of HTTP requests while maintaining the same completeness of results, and does this reduction of HTTP requests lead to a decrease in query execution time?
With hypotheses that follow:
Using shape indexes will reduce the number of non-contributing data sources acquired;
there is a linear relationship between the reduction of the number of HTTP requests and the reduction of query execution time;
executing a query-shape containment is negligible in the context of social media applications; 
more detailed shapes will provide a higher reduction in the number of HTTP requests;
a \emph{complete} shape index will reduce the number of HTTP requests and the query execution time than an \emph{incomplete} shape index;
a shape index approach can be adaptative, so not every dataset in the network needs to have an index to see a performance improvement.
\iffalse
\pc{Started changing these to be more concrete by changing how to by how much or what. What do you think about that? More concrete or worse?}
By how much can this method reduce the ratio of dereferenced non-contributing data sources?
By how much do the fewer HTTP request affect the query execution time?
By how much does the level of detail of the shapes impact the performances?
What is the difference in performance between a \emph{complete} and an \emph{incomplete} shape index?
How does the ratio of subdomains containing an index influence global performances?
How does the fragmentation of the subdomain impact the performance?
How does the quantity of non-query contributing resources impact the performances?
What is the ideal query execution time if we only dereferenced contributing data sources?
\fi
