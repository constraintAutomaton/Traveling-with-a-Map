\section{Introduction}
\sepfootnotecontent{sf:webID}{
    \url{https://www.w3.org/wiki/WebID}
}
\sepfootnotecontent{sf:dataSovereignty}{
    \url{https://digital-strategy.ec.europa.eu/en/policies/strategy-data}
}

Data sovereignty seeks to establish a more just definition of personal data ownership in terms of data usage and storage.
It can be defined as ``the self-determination of individuals and organizations concerning to the use of their data''~\cite{verstraete2022solid},
which in practice can be interpreted as the power to choose where one's data is stored and who has access to it~\cite{verstraete2022solid}.
Multiple studies have denoted problems of ownership, democracy, reinforcement of inequality, and antagonism between users and owners of web social applications~\cite{Terranova2000FreeLP, Curran2016ch1, Sevignani2013, 9663788}.
Several authors consider decentralizing web data an insufficient solution~\cite{9663788, Curran2016ch1}; however, it can still be an integral component of initiatives focused on data sovereignty.
Thus, technical research on that topic is a relevant endeavor.
Linked data has the potential to create seamless decentralized knowledge graphs through the use of dereferencable IRIs.
These IRIs allow access to additional knowledge graph (KG) containing information relevant to what an IRI identifies.
For example, a WebID~\sepfootnote{sf:webID} represents a user, dereferencing it can provide the name of the user, among other information, without having to store the information locally.
Despite these benefits, most SPARQL query processing occurs in centralized setups, partly because optimizing queries is easier in centralized environments.

A query paradigm called Link Traversal Query Processing (LTQP)~\cite{Hartig2012} has been developed to leverage the potential descriptive power of IRI dereferencing.
LTQP consists of recursively dereferencing IRIs discovered in a query engine's internal triple store during query execution to expand its base of information.
The main difficulty of LTQP is the large domain of exploration, which results in the performance of a high number of HTTP requests as demonstrated by \citeauthor{hartig2016walking}~\cite{hartig2016walking}.
From another perspective, and without contradicting \citeauthor{hartig2016walking}~\cite{hartig2016walking}, \citeauthor{Taelman2023}~\cite{Taelman2023} has demonstrated that in descentralized environments with structural properties (DESP),
it is possible to attain query completeness for various types of practical queries within acceptable execution times.
Moreover, they showed that query planning could significantly influence the execution time.
Structural properties ensure data discoverability, which in turn helps guarantee result completeness through a concept called structural assumptions.
In practice, DESP can be used in the context of personal data and social networks, among others.
Examples of environments that adhere to these constraints are datasets following the Solid protocol~\cite{Taelman2023} and the TREE specification~\cite{tam_iswc_traversalsensortree_2024}.
The work of \citeauthor{Taelman2023}~\cite{Taelman2023} indicates that there are multiple optimizations possible in LTQP in the context of descentralized environments with structural properties as opposed to the
more pessimist conclusion of the work of \citeauthor{hartig2016walking}~\cite{hartig2016walking} for Open Linked Data without structural properties.

From a holistic perspective, the web does not have a structure exploitable by query engines for optimization.
On the web, any document can be published at any location, and there is no index or trust mechanism to guide source selection.
From a perspective where the web is divided into small subsections controlled by data providers, data publishers can implicitly or explicitly organize their data  
in a way that query engines could exploit its organization for optimization.
We propose to put our focus on the completeness of results when considering that the completeness of traversal has been fixed.
Placing our focus on results allows us to investigate ways to optimize the search space of link traversal queries by pruning irrelevant sources.
We propose using a decentralized dataset summary method, called the shape index~\cite{tam2024opportunitiesshapebasedoptimizationlink}, as the support mechanism for our pruning approach.
This method involves mapping the content of a decentralized dataset using RDF data shapes.
The intuition behind this approach is that publishing explicit RDF schemas is relatively inexpensive for data providers when publishing decentralized datasets but they could be highly beneficial for clients.
Although RDF does not enforce schemas on its data, the nature of the usually modeled objects and the formalism often results in implicit schemas~\cite{Neumann2011CharacteristicSA}.

This paper is organized as follows: first, we introduce the problem statement; next, we discuss related work and present preliminary concepts.
We then describe our approach, followed by the experimental setup and results.
Finally, we conclude the article.

\section{Problem Statement}
To guide our study we formulated the following research question.
\textbf{Can a LTQP use shape-based pruning in networks of decentralized Knowledge Graphs to reduce the number of HTTP requests while maintaining the same completeness of results, and does this reduction of HTTP requests lead to a decrease in query execution time?}
We formulated the following hypotheses:
\newcounter{hypothesisCounter}
\setcounter{hypothesisCounter}{1}

\begin{itemize}[label=\textbf{H\arabic{hypothesisCounter}}\,\stepcounter{hypothesisCounter}]
    \item Using shape indexes will reduce the number of non-contributing data sources acquired
    \item There is a linear relationship between the reduction of the number of HTTP requests and the reduction of query execution time
    \item Executing a query-shape containment is negligible in the context of social media applications
    \item More detailed shapes will provide a higher reduction in the number of HTTP requests
    \item A network with more \emph{complete} shape index will reduce more the number of HTTP requests and the query execution time than one with less
    \item The shape index approach can be adaptative, so not every dataset in the network needs to have an index to see a performance improvement
\end{itemize}
