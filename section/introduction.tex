\section{Introduction}
\sepfootnotecontent{sf:webID}{
    \url{https://www.w3.org/wiki/WebID}
}
\sepfootnotecontent{sf:dataSovereignty}{
    \url{https://digital-strategy.ec.europa.eu/en/policies/strategy-data}
}



Multiple studies have highlighted that social web applications relying on highly centralized data management systems give rise to problems of ownership, threats to democracy, reinforcement of inequality, and antagonism between users and owners~\cite{Terranova2000FreeLP, Curran2016ch1, Sevignani2013, 9663788}.
Yet, several authors consider decentralizing data over the web an insufficient solution~\cite{9663788, Curran2016ch1}, although it is an integral component of initiatives focused on data sovereignty.
Linked Data and knowledge graphs (KG)~\cite{heath2011} can be considered technical contributions toward the development of a decentralized web of data.
However, SPARQL, the standard query language for RDF knowledge graphs, is predominantly performed in centralized environments, partly due to the more mature understanding of query optimization in such settings.

Link Traversal Query Processing (LTQP)~\cite{Hartig2012} is a query paradigm designed for querying non-indexed, Decentralized Knowledge Graphs (DKGs) on the web, by leveraging the descriptive power of IRI dereferencing.
LTQP involves recursively dereferencing IRIs, dynamically discovering and storing triples from the documents associated with those IRIs in an internal triple store, thereby expanding the engine’s underlying knowledge base during query execution.
The main difficulty of LTQP is the large domain of exploration, which leads to a high number of HTTP requests as demonstrated by \citeauthor{hartig2016walking}~\cite{hartig2016walking}.
From another perspective, it has been shown that in Decentralized Environments with Structural Properties (DESPs), it is possible to attain query completeness for various types of practical queries with acceptable execution times for the context of social media applications~\cite{Taelman2023, nielsen1993response}.
%Moreover, they showed that query planning could significantly influence the execution time.
Structural properties ensure data discoverability, which in turn helps guarantee result completeness.

In practice, DESPs emerge in various contexts, such as social networks~\cite{Taelman2023}, the publication of sensor data~\cite{tam_iswc_traversalsensortree_2024}, among others.
The work on the topic of DESPs for LTQP~\cite{Taelman2023, Bogaerts2021LinkTW} suggests that various optimizations are feasible for LTQP in decentralized environments with structural properties, in contrast to the more pessimistic conclusion of \citeauthor{hartig2016walking}~\cite{hartig2016walking}.

\begin{figure}
    \centering
    \includesvg[width=1.0\linewidth]{figure/network.svg}
    \caption{
    The traversal query engine dereferences a large proportion of query-irrelevant resources. 
    In the figure, green circles represent dereferenced resources, with those marked by a check indicating query-relevant ones. 
    White circles represent resources that are not dereferenced by the engine.
    }
    \label{fig:situation}
\end{figure}

In general, data on the World Wide Web lack a structure that query engines can exploit for optimization.  
Any document can be published anywhere, with no standard index or trust mechanism to guide discovery.
Thus, for the vast majority of queries, traversal query engines dereferences a large proportion of resources that are not query-relevant, as depicted in Figure~\ref{fig:situation}.
However, within specific \emph{subwebs}, defined as subsections of the web controlled by particular data providers, implicit or explicit data structures may emerge, which query engines can leverage~\cite{Bogaerts2021LinkTW}.
In this work, we extend a dataset summarization approach for decentralized environments known as the \emph{shape index}~\cite{tam2024opportunitiesshapebasedoptimizationlink}.
We apply this approach to enable link pruning within LTQP, removing links that are not relevant to the query, based on an analysis of RDF data shapes and the user's query.
The analysis is performed by conducting a query-shape subsumption check to determine whether a resource conforming to a given shape is relevant to a query.
Our approach assumes a DKG composed of subwebs, each hosted by data providers and containing shape indexes.
A subweb that exposes a shape index enables the query engine to narrow its search domain by identifying resources that are certainly non-query-relevant.
\textbf{This is particularly useful when only a subset of a subwebs is relevant to a given query, for example, in social media applications where it is rare to query all information about a user.}
An example of such a query is one that requests messages within a user's subweb over a specified date range, while also incorporating contextual information from outside the subweb.
We describe informally this category of network-tied queries as \emph{data-model selective}.
In the context of this paper, we define a \emph{data-model selective query} as a query that requests less than 50\% of the data-model objects.
%This designation goes beyond simple query categorization, since a subweb may organize its data such that multiple data model objects are materialized within the same resources.  
%In such cases, even if a query retrieves a small percentage of triples, we do not regard it as data-model selective.  
In this work, we do not aim to optimize queries that retrieve a large proportion of a subweb's data model objects.  
For instance, queries that request all user information together with all content published after a given date are outside the scope of our optimization; however, we attempt to ensure that the performance of such queries is not degraded.  

Our contributions are as follows:
\begin{enumerate}[label=(\roman*)]
    \item An introduction and formalization of \emph{link pruning} for LTQP 
    \item A web specification for the \emph{shape index} data summary
    \item A novel shape-based pruning approach for LTQP using RDF data shapes
    \item A novel network abstractions of \emph{subwebs} and \emph{decentralized knowledge graphs} (DKGs) for LTQP,
    \item A novel \emph{query-shape subsumption algorithm} for assessing data source relevance
    \item An open-source implementation in the Comunica SPARQL query engine~\cite{taelman_iswc_resources_comunica_2018}
    \item An extensive experimental evaluation using the SolidBench benchmark.
\end{enumerate}

In this work, we ask the following research question: 
\textbf{Can LTQP use shape-based pruning in DKG networks to reduce query execution time while preserving result completeness?}
To address this question, we propose the following hypotheses:  

\begin{itemize}
    \item[\textbf{H1}] The shape indexes approach reduce the number of non-contributing data sources retrieved and the query execution time of \emph{data-model selective} queries, while preserving result completeness.
    \item[\textbf{H2}] The shape indexes approach does not affect the query execution time of \emph{non-data-model selective} queries.
    \item[\textbf{H3}] The execution time of a query-shape subsumption algorithm is negligible compare to the query execution.
    \item[\textbf{H4}] Stricter shape constraints lead to a greater reduction in HTTP requests.
    \item[\textbf{H5}] Querying a network with more \emph{complete} shape indexes results in faster query execution.
    \item[\textbf{H6}] Performance gains can be achieved even in networks with partial shape index coverage, demonstrating that the method provides significant improvements without requiring all network participants to adopt shape indexes.
    \item[\textbf{H7}] There is a linear relationship between the reduction in the number of HTTP requests and the decrease in query execution time.
\end{itemize}

The remainder of this paper is organized as follows: 
we first review the \hyperref[sec:related_work]{related work} and introduce the \hyperref[sec:preliminaries]{preliminaries}. 
Next, we present our \hyperref[sec:approach]{approach}, followed by a description of the \hyperref[sec:experiment]{experimental setup} and a \hyperref[sec:result]{discussion of the results}. 
Finally, we conclude with the \hyperref[sec:conclusion]{conclusion}.
