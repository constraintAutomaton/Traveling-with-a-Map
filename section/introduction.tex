\section{Introduction}
\sepfootnotecontent{sf:webID}{
    \url{https://www.w3.org/wiki/WebID}
}
\sepfootnotecontent{sf:dataSovereignty}{
    \url{https://digital-strategy.ec.europa.eu/en/policies/strategy-data}
}

Data sovereignty attempts to define a more just definition of ownership and usage of personal data.
It can be defined as ``the self-determination of individuals and organizations concerning to the use of their data''~\cite{verstraete2022solid},
which in practice can be interpreted as the power to choose where one's data is stored and who has access to it~\cite{verstraete2022solid}.
Multiple studies have denoted problems of ownership, democracy, reinforcement of inequality, and antagonism between users and owners of web social applications~\cite{Terranova2000FreeLP, Curran2016ch1, Sevignani2013, 9663788}.
Decentralizing web data is probably an insufficient propostion~\cite{9663788, Curran2016ch1}, however, it is an integral part of a project like data sovereignty.

Linked data has the potential to create seamless decentralized databases through the use of dereferenciable IRIs.
These IRIs allow access to additional databases containing information relevant to the IRI.
For example, dereferencing a term representing a user like a WebID~\sepfootnote{sf:webID} can provide the name of the user, among other information, without having 
to store this information locally.
Such data publication paradigm gives the opportunity to potentially break down data silos~\cite{verstraete2022solid},
and empowers users by reducing the need for indexing strategies to find relevant information about a term. 
Furthermore, it can help address legal and ethical concerns around data centralization, a topic of actuality in multiple world regions.
Despite of those potentials, most SPARQL query processing is carried out with centralized setups or federations of endpoints.

To take advantage of the potential descriptive power of IRI dereferencing a query paradigm called Link Traversal Query Processing (LTQP)~\cite{Hartig2012} has been developed.
LTQP consists of recursively dereferencing IRI contained into the internal data source of a query engine during its query execution to expand its base of information.
A lookup policy can be used to limit the search domain of the query.
LTQP has multiple difficulties such as the open endlessness of the web, which can be interpreted as a pseudo-infinite domain of exploration,
and the challenge of creating an efficient query plan due to the lack of information about the data sources. 
It has been demonstrated by \citeauthor{hartig2016walking} that in the open web the main performance bottleneck and obstacle for query completeness and fast query execution time is not the query plan but the large number of the HTTP request necessary to fulfill a query.

From another perspective, and not disproving \citeauthor{hartig2016walking}, \citeauthor{Taelman2023} has demonstrated that in a Structured Linked Data Environment (SLDE),
it is possible to attain query completeness.
Furthermore, query planning could significantly influence the speed of execution.
A SLDE is defined has an RDF environment where in addition to the RDF principles, specifications
guarantee the completeness of results.
This guarantee of completeness has the positive side effect of providing information
that can be used to create a lookup policy to reduce the number of HTTP requests necessary to attain completeness.

In practice SLDE can be use in the context of personal data and social network among others,
exemples of environments respecting those constraints are dataset following the Solid protocol~\cite{Taelman2023} and TREE specification~\cite{tam_iswc_traversalsensortree_2024}.
The work of Taelman~\cite{Taelman2023} indicate that there are multiple optimizations possible in LTQP in the context SLDE as opposed to the
more pessimist conclusion of the work of ~\citeauthor{hartig2016walking}.
This paper formalize the concept of \emph{Shape Index} and \emph{query-shape contaiment}~\cite{tam2024opportunitiesshapebasedoptimizationlink} for online source selection in LTQP.
The shape index concept relly on RDF data shapes.
The conceptual idea of RDF data shape is to describe the properties of an entity;
for instance the type of the property, their cardinalities and constraints.
RDF data Shapes have already been used for optimizing the query planning~\cite{kashif2021}
in the context of a single endpoint query, but as far as our knowledge goes, none of the studies have used them in the context of LTQP of SLDE.
We propose to transform them into queries to perform a query containment problem and determine when a source is not query relevant.

We formulated this research question to guide our study:
Can a link traversal query engine use shape indexes in networks of decentralized datasets to reduce the number of HTTP requests while maintaining the same completeness of results, and does this reduction of HTTP requests lead to a decrease in query execution time?
We also made the following hypotheses:
Using shape indexes will reduce the number of non-contributing data sources acquired;
a more detailed shape will provide a higher reduction in the number of HTTP requests;
a \emph{complete} shape index will have a higher positive impact on performance with most queries;
a shape index approach can be adaptative, so not every dataset in the network needs to have an index to see a performance improvement.
\iffalse
Can our method reduce the ratio of non-contributing data source dereferenced?
How does the diminution of HTTP request affect the query execution time?
How does the level of detail of the shapes impact the performances?
What is the difference in performance between a \emph{complete} and an \emph{incomplete} shape index?
How does the ratio of subdomains containing an index influence global performances?
How does the fragmentation of the subdomain impact the performance?
How does the quantity of non-query contributing resources impact the performances?
What is the ideal query execution time if we only dereferenced contributing data sources?
\fi