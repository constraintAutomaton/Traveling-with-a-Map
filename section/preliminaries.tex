\section{Preliminaries}

\sepfootnotecontent{sf:sparqlSpecification}{
    The SPARQL specification is available at the following link \newline\url{https://www.w3.org/TR/sparql11-query/}
}

\sepfootnotecontent{sf:blankNode}{
 A blank node is a unique idenfier with no IRI \url{https://www.w3.org/wiki/BlankNodes}.
}

\sepfootnotecontent{sf:propertyPathAlgebra}{
    \url{https://www.w3.org/TR/sparql11-query##PropertyPathPatterns}
}

\sepfootnotecontent{sf:recursiveShape}{
    In this formulation we ignore "inverse constraints" such as
    inverse triple constraint in ShEx \url{https://shex.io/shex-primer/index.html##inverse-properties}
    and SCHACL Inverse Paths \url{https://www.w3.org/TR/shacl##property-path-inverse} to avoid recurse 
    shape schemas~\cite{Corman2019}.
}


\subsection{RDF knowledge graph and SPARQL queries}
Our work focuses on conjunctive and disjunctive queries $Q$ of RDF knowledge graph.
An RDF knowledge graph is composed of triples $t$ defined in Definition~\ref{def:triple} where the subjects and objects are nodes of the graph and the predicates are the edges.

\begin{definition}[RDF Triple]\label{def:triple}
    An RDF triple $t = (s,p,o)$ is formed by a subject term $s \in \mathcal{I} \cup \mathcal{B}$, a predicate term  $p \in \mathcal{I}$ and an object term $o \in \mathcal{I} \cup \mathcal{B} \cup \mathcal{L}$
    where $\mathcal{I}$, $\mathcal{B}$, $\mathcal{L}$ are respectiverly the set of every possible IRI, blank node~\sepfootnote{sf:blankNode} and litteral.
\end{definition}

We define the basic graph pattern (BGP) of $Q$~\sepfootnote{sf:sparqlSpecification} in Definition~\ref{def:bgp}.
The atomic statement of a BGP is a triple pattern $tp$ defined at Definition~\ref{def:triplePattern}.

\begin{definition}[Triple pattern]\label{def:triplePattern}
    A triple pattern $tp = (s_{tp},p_{tp},o_{tp})$ is formed by a subject term $s_{tp} \in \mathcal{I} \cup \mathcal{B} \cup \mathcal{V}$, 
    a property path  $p_{tp}$ defined in Definition~\ref{def:propertyPath} and an object term  $o_{tp} \in \mathcal{I} \cup \mathcal{L} \cup \mathcal{V}$ 
    where $\mathcal{V}$ is the set of every possible variable. 
\end{definition}

\begin{definition}[Property path]\label{def:propertyPath}
   A property path $p_{tp}$~\cite{Kostylev2015} is an expression that describe the route of predicate $p$ from $s_{tp}$ to $o_{tp}$.
   \iffalse
   A property path in $tp$ is defined as follow:
   \begin{equation}
    p_{tp} ::= p \in \mathcal{I} | (p_{tpi}/p_{tpj}) | (p_{tpi}|p_{tpj}) | p_{tpi}* | p_{tpi}+ | p_{tpi}? | !p_{tpi}| p_{tpi}^{-}
   \end{equation}.
   The "$/$" operator chain two property path, the alternative operator "$|$" define a possibility between $p_{tpi}$ and $p_{tpj}$.
   The "$^-$" operator inverse the path from $s_{tp}$ to $o_{tp}$.
   The "$!$" represent the negation of a path $p_{tpi}$ such as $p_{tpi} = \mathcal{I} \setminus p_{tpi}$.
   The "$*$", "$+$" and "$?$", refered to in this paper as \emph{cardinality property paths}, are respectively; the repetition of 0 and more of $p_{tpi}$, 
   the repetition of 1 and more of $p_{tpi}$ and the presence or absence of $p_{tpi}$.
   \fi
\end{definition}

Property paths can be represented in SPARQL algebra as defined by \citeauthor{Kostylev2015} (Definition 2) and the SPARQL specification (section 18.2).~\sepfootnote{sf:propertyPathAlgebra}
Notably, the alternative operator can be interpreted as a union, and the sequence operator as a chain of triple pattern.

\begin{definition}[BGP]\label{def:bgp}
 A BGP $B$ is a set of $tp$, \texttt{OPTIONAL} and \texttt{UNION} (both clause containing a sets of $tp$).
 $tp$ sets can be divided into star pattern $Q_{star}$ defined in Definition~\ref{def:starPattern} in a way 
 that $Q = Q_{star_{i-1}} \bowtie Q_{star_i} \forall i \in S$ given that $S$ is the set of every subject of the $tp$ in the $Q$.
\end{definition}


\begin{definition}[Star pattern with dependencies]\label{def:starPattern}
We define a star pattern $Q_{star}$ as a set of $tp \in Q$ with the same subject such has $Q_{star} = \{ tp\in Q| s_{tp_1} = s_{tp_2} ... = s_{tp_n}\}$~\cite{Karim2020}.
We define a star pattern with 1 deep dependency $Q_{star1D}$ as the union between a star pattern $Q_{star} \in Q$ and other star patterns having as subjects
the objects of the initial star pattern such as
\begin{equation}\label{eq:starPattern1D}
    Q_{star1D} = Q_{star} \cup \{tp \in Q| o_{0} \in tp_{star} = s_{tp1} \land s_{tp_1} = s_{tp_2} ... = s_{tp_n}\}
\end{equation}
.
A star pattern with dependencies has $n$ dependencies and apply recursively equation \label{eq:starPattern1D}

We also use the same terminology when dealing with triples instead of triple patterns.
\end{definition}

\subsection{Reachability criteria}

\emph{Reachability criteria} are boolean functions ($c_i$) restricting the dereferencing of links from the internal data source of the query engine.
They take as parameters an RDF triple $t$ from the internal triple store, a dereferenceable IRI $iri$ from $t$, and the basic query pattern $B$ from the query~\cite{Hartig2012}.
If $c_i$ returns $true$, the query engine must try to dereference $iri$.
A formalization is presented in equation~\ref{eq:reachabilityCriteria}.


\begin{equation}\label{eq:reachabilityCriteria}
c_i(t, iri, B) \rightarrow \{\mathrm{true}, \mathrm{false}\}
\end{equation}

The engine's dereferencing operation stops when, during an iteration of the new triples in the internal triple store, no new $iri$ has to be dereferenced.

\subsection{RDF data shapes}
RDF data shapes similar to data schema are constraints on RDF graphs.
They define the predicate, and the object a term with the same subject must respected.
We can interpret them as defining a schema on star patterns with dependencies.
We propose an abstract shape formalization. 
Thus, we make no distinction between the shapes languages, however, it is strongly inspired by ShEx~\cite{Gayo2018}.~\sepfootnote{sf:recursiveShape}

We define a shape
\begin{equation}
S = (E, op \in \{\mathrm{true},\mathrm{false}\})
\end{equation}

with some similarities to \citeauthor{Abbas2018} definition.
$E = \{e_1, e_2 ..., e_{n}\}$ is a set of shape expressions given $n$ shape expression $e_i$, $op$ describing whether the shape is opened or closed.
An open shape allowed the presence of non-contradicting properties (with consideration of negative properties) not defined in $E$ in a knowledge graph (KG) $G$
A shape expression is defined by
\begin{align}
 e_i ::= (p, c, rs, n) | (e_j|e_k)
\end{align}
.
$e_i$ is a tuple of constraints on the predicate and the object of one triple of a star pattern with dependencies in a KG. 
A shape expressions define a predicate $p$ constraint that the star pattern with dependencies in $G$ must possess.
It also define a constraint on the object $c(o) \rightarrow \{\mathrm{true}, \mathrm{false}\}$.
The $c$ constraint can enforce an RDF type on an object, that the object must respects a shape or a literal comparison.
$rs = \{x \in \mathbb{R}_+ | a < x < b \}$ is a range constraint on the number of object.
The shape expression can be a negation $n \in \{\mathrm{true}, \mathrm{false}\}$ meaning that the constraint $e_i$ must not be satisfied.
A shape expression can also be a disjunction, between two other shape expressions.

We also define an evaluation function returning $\mathrm{true}$ if a knowledge graph (KG) $G$ is validated by a shape $S$, 
meaning that no triple in $G$ makes a star pattern with dependencies in $G$, invalidates the constraint of $S$.

\begin{equation}
    SE(S, G) \rightarrow \{\mathrm{true}, \mathrm{false}\}
\end{equation}


\subsection{Decentralized Knowledge Graph}
We define a decentralized knowledge graph (DKG) as a knowledge graph (KG) $G$ materialized in a network of resources $R$.
A resource $r_i \in R$ contains a KG $g_i \subseteq G$ and is mapped to an IRI $iri_i$.
The network forms a graph where the resources $r_i$ are the nodes and the $iri_j \in t \in g$ are directed edges starting from $r_i$ to $r_j$.
$G$ is formed by the union of all the $g \in r$, such that $G = \bigcup_{i=0}^{n}g_i$ given $n$ resources.