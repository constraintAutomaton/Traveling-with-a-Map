\section{Preliminaries}\label{sec:preliminaries}

\subsection{RDF Knowledge Graphs and SPARQL Queries}

Our work focuses on the union of conjunctive queries over RDF knowledge graphs~(KG) using the SPARQL query language~\cite{w3SPARQLQuery}.
The fundamental building blocks of KGs and SPARQL queries are triples and triple patterns, respectively, as defined in Definition~\ref{def:triple} and Definition~\ref{def:triplePattern}.

\begin{definition}[Triple]\label{def:triple}
    RDF triples $t = (s,p,o)$ are tuples formed with three terms. A subject where $s \in \mathcal{I} \cup \mathcal{B}$, a predicate $p \in \mathcal{I}$ and an object $o \in \mathcal{I} \cup \mathcal{B} \cup \mathcal{L}$.
    Where $\mathcal{I}$, $\mathcal{B}$, $\mathcal{L}$,  are respectively the set of every possible IRI, blank node, literal.
    For simplicity, we denote the union of these sets by concatenating their symbols, so that $ \mathcal{I} \cup \mathcal{B}$ is written as $ \mathcal{I} \mathcal{B}$.
\end{definition}

\begin{definition}[Triple pattern]\label{def:triplePattern}
    Triple patterns $tp = (s_{tp}, p_{tp}, o_{tp})$ are similar to triples, where $s_{tp} \in \mathcal{I} \mathcal{B} \mathcal{V}$,
    $p_{tp} \in \mathcal{I} \mathcal{V}$ and an object term  $o_{tp} \in \mathcal{I} \mathcal{B} \mathcal{V} \mathcal{L}$.
    Where $\mathcal{V}$ is the set of every possible variable.
    %A triple pattern returns a solution sequence with solution mappings a single or multiple $tp$ forms a Basic Graph Pattern (BGP).
\end{definition}

We also define two access functions to respectively get the subject and object term of a triple or a triple pattern while ignoring literals,
$ S:  (\mathcal{I} \mathcal{B} \mathcal{V}, \mathcal{I} \mathcal{V}, \mathcal{I} \mathcal{B} \mathcal{V} \mathcal{L}) \rightarrow \mathcal{I} \mathcal{B} \mathcal{V}$ and $O: (\mathcal{I} \mathcal{B} \mathcal{V}, \mathcal{I} \mathcal{V}, \mathcal{I} \mathcal{B} \mathcal{V} \mathcal{L}) \rightarrow \mathcal{I} \mathcal{B} \mathcal{V}$.
We denote $[\![ Q ]\!]^{G}$ as the evaluation of a query $Q$ over a KG $G$~\cite{Angles2008}.

\subsection{Reachability Criteria}

LTQP defines completeness on the traversal of links instead of the query results~\cite{Hartig2012}.
To formalize the completeness of queries, \emph{Reachability criteria}~\cite{Hartig2012} have been formalized.
Reachability criteria are boolean functions ($c_i$) restricting the dereferencing of links from the internal data source of the query engine.
They take as parameters an RDF triple $t$ from an internal triple store, a dereferenceable IRI $iri$ from $t$, and a union of conjunctive queries $Q$.
If $c_i$ returns $true$, the query engine must dereference $iri$.
More formally
\begin{equation}\label{eq:reachabilityCriteria}
c_i(t, iri, Q) \rightarrow \left\{ \mathrm{true}, \mathrm{false} \right\}
\end{equation}

\subsection{Decentralized Knowledge Graphs and Subweb}\label{sec:dkg}

We define a DKG as a KG $G$ materialized in a network of resources $R$.
A resource $r_i \in R$ is mapped to a KG $g_i \subseteq G$, which is a set of triples~\cite{w3ConceptsAbstract}.
We denote this mapping $r_i \mapsto_{\mathcal{G}} g_i$.
A resource is mapped and exposed by an IRI $iri_i$ denoted by $iri_i \mapsto_{\mathcal{R}} r_i$.
The network forms a graph where the resources $r_i$ are the nodes and the $iri_j$ are directed edges starting from $r_i$ to $r_j$.
The $iri_j$ are RDF terms in the triples in $g_i$.
$G$ is formed by the union of all the $g_i$ mapped to a resource in the network.
A subweb is a (sub)DKG defined by the KG derived from a set of IRIs controlled by a data provider.

\subsection{Data-model Selectiveness}
Data-model selectiveness is an ordering of queries over networks. 
In this work, we assume that data-model objects are represented by shapes, however, they could also be represented by vocabularies or other formalisms.
We define $DM$ as the set of the data model present in the networks.
We define a function to determine all the relevant data-model objects for a given query:
\begin{equation}
   P(Q) = \{ D \mid (D \in DM) \text{ \textbf{is query relevant for} } Q \}.
\end{equation}
A query $Q_i$ is more data-model selective than another query $Q_j$ if and only if 
\begin{equation}
    \left| P(Q_i) \right| > \left| P(Q_j) \right|.
\end{equation}

For practical use cases, we may consider only the data models present in subwebs of a network and replace $DM$ by $DM^s$,
with the associated function becoming $P^s(Q)$.
Furthermore, for practical designation, we say that a query is \emph{data-model selective} (in absolute terms) if 
\begin{equation}
    \frac{\left| P^s(Q) \right|}{\left| DM^s \right|} \leq 0.5,
\end{equation}
that is, if fewer than 50\% of the data-model objects are potentially relevant to the query.