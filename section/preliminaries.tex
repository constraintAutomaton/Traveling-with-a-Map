\section{Preliminaries}

\sepfootnotecontent{sf:sparqlSpecification}{
    The SPARQL specification is available at the following link \newline\href{https://www.w3.org/TR/sparql11-query/}{https://www.w3.org/TR/sparql11-query/}
}

\sepfootnotecontent{sf:propertyPath}{
 We purposely ignored the inverse operator because we want to define the property path as a route of predicates and the inverse operator acts on the subject and object.
 Since its deletion does not change the expressivity of property we believe that their deletion does not affect the validity of our formalism.
}



\subsection{RDF knowledge graph and SPARQL queries}
Our works focus on conjunctive queries of RDF knowledge graph.
A knowledge is a graph of RDF triples $t$ defined in Definition~\ref{def:triple}.

\begin{definition}[RDF Triple]\label{def:triple}
    An RDF triple $t = (s,p,o)$ is formed by a  subject term $s \in \mathcal{I} \cup \mathcal{B}$, a predicate term  $p \in \mathcal{I}$ and an object term $o \in \mathcal{I} \cup \mathcal{B} \cup \mathcal{L}$
    where $\mathcal{I}$, $\mathcal{B}$, $\mathcal{L}$ are respectiverly the set of every possible IRI, blank node and litterals.
\end{definition}

In our work we we consider SPARQL conjunctive query $Q$.
We consider two parts of the query the basic graph pattern (BGP) and the filter expressions~\sepfootnote{sf:sparqlSpecification}.
The most atomic statement of a BGP is a triple pattern $tp$ which shares some similarities with $t$.

\begin{definition}[Triple pattern]\label{def:triplePattern}
    An triple pattern $tp = (s_{tp},p_{tp},o_{tp})$ is formed by a subject term $s_{tp} \in \mathcal{I} \cup \mathcal{B} \cup \cup \mathcal{V}$, 
    a property path  $p_{tp}$ (defined in the Definition~\ref{def:propertyPath}) and an object term  $o \in \mathcal{I} \cup \mathcal{B} \cup \mathcal{L} \cup \mathcal{V}$ 
    where $\mathcal{V}$ is the set of every possible variable. 
\end{definition}

\begin{definition}[Property path]\label{def:propertyPath}
   A property is an expression that describe the route of predicate $p$ from $s_{tp}$ to $o_{tp}$.
   A property path in $tp$ is defined as follow~\sepfootnote{sf:propertyPath}:
   $$p_{tp} ::= p | (p_{tpi}/p_{tpj}) | (p_{tpi}|p_{tpj}) | p_{tpi}* | p_{tpi}+ | p_{tpi}? | !p_{tpi}$$.

   The "$/$" operator chain two property path, the operate "$|$" define a possibility between $p_{tpi}$ and $p_{tpj}$.
   The "$!$", "$*$", "$+$" and "$?$" respectively represent the negation of $p_{tpi}$, the a repetition of 0 and more of $p_{tpi}$, 
   the repeition of 1 and more of $p_{tpi}$ and the presence of absence of $p_{tpi}$.
\end{definition}

\begin{definition}[BGP]\label{def:propertyPath}
 A BGP is a set of $tp$, \texttt{OPTIONAL} and \texttt{UNION} (both clause containing a set of $tp$).
\end{definition}


\subsection{Reachability criteria}

\emph{Reachability criteria} are boolean function ($c_i$) restricting the dereferencing of operation links inside the internal data source of the query engine.
They are take as parameters an RDF triple $t$ from the internal triple store, an dereferenciable IRI $i_d$ from $t$ and the a basic query pattern $B$ from the query~\cite{Hartig2012}.
If $c_i$ return $true$ then the query engine try to dererefence $i_d$.
A formalization is presented in equation~\ref{eq:reachabilityCriteria}.


\begin{equation}\label{eq:reachabilityCriteria}
c_i(t, i_d, B) \rightarrow \{\mathrm{true}, \mathrm{false}\}
\end{equation}

As they are boolean function reachability criteria can be chained together to form \emph{Composite reachability criteria}.
In this form a reachability criterion ($cp_i$) is said to \emph{prune} links if it is chained with an \emph{and} operator with the other criterion ($cd_i$) and it is said to \emph{discover} links if it chain with an \emph{or} operator.
Equation~\ref{eq:cReachabilityCriteria} express a Composite reachability criterion $C$ with $nd$ discovery criteria and $np$ prunning criteria.

\begin{equation}\label{eq:cReachabilityCriteria}
    C(t, i_d, B)  = \bigvee_{i=0}^{nd}cd_i(t, i_d, B) \land \bigvee_{j=0}^{np}cp_j(t, i_d, B)
\end{equation}

The derefencing operation of the query stop when during an interation of the new element of the internal triple store no new $i_d$ has to be derefenced.

\subsection{RDF data shapes}