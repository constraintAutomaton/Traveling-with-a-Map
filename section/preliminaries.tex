\section{Preliminaries}\label{sec:preliminaries}

\subsection{RDF knowledge graphs and SPARQL queries}

Our work focuses on the union of conjunctive queries over RDF knowledge graphs~(KG) using the SPARQL query language~\cite{w3SPARQLQuery}.
The core components of KGs and SPARQL queries are respectively the triples and the triple patterns as deined in Definition~\ref{def:triple} and ~\ref{def:triplePattern}.

\begin{definition}[Triple]\label{def:triple}
    RDF triple $t = (s,p,o)$ are tuples formed with three terms. A subject where $s \in\mathcal{I} \cup \mathcal{B}$, a predicate $p \in \mathcal{I}$ and an object $o \in \mathcal{I} \cup \mathcal{L} \cup \mathcal{B}$.
    Where $\mathcal{I}$, $\mathcal{B}$, $\mathcal{L}$,  are respectively the set of every possible IRI, blank node, literal.
\end{definition}

\begin{definition}[Triple pattern]\label{def:triplePattern}
    Triple patterns $tp = (s_{tp}, p_{tp}, o_{tp})$ are similar to triples, where $s_{tp} \in \mathcal{I} \cup \mathcal{B} \cup \mathcal{V}$,
    $p_{tp} \in \mathcal{I} \cup \mathcal{V}$ and an object term  $o_{tp} \in \mathcal{I} \cup \mathcal{L} \cup \mathcal{V} \cup \mathcal{B}$.
    Where $\mathcal{V}$ is the set of every possible variable.
    %A triple pattern returns a solution sequence with solution mappings a single or multiple $tp$ forms a Basic Graph Pattern (BGP).
\end{definition}

We also define two access functions to get respectively the subject and object term of a triple pattern or a triple while ignoring literals,
$ S: tp \rightarrow \mathcal{I} \cup \mathcal{B} \cup \mathcal{V}$ and $O: tp \rightarrow \mathcal{I} \cup \mathcal{B} \cup \mathcal{V}$.
We denote $[\![ Q ]\!]^{G}$ as the evaluation of a query $Q$ over a KG $G$~\cite{Angles2008}.

\subsection{Reachability Criteria}

LTQP defines completeness on the traversal of links instead of the query results~\cite{Hartig2012}.
To formalize the completeness of queries, a link discrimination formalism has been developed called \emph{Reachability criteria}~\cite{Hartig2012}.
Reachability criteria are boolean functions ($c_i$) restricting the dereferencing of links from the internal data source of the query engine.
They take as parameters an RDF triple $t$ from an internal triple store, a dereferenceable IRI $iri$ from $t$, and a union of conjective queries $Q$~\cite{Hartig2012}.
If $c_i$ returns $true$, the query engine must dereference $iri$.
More formally
\begin{equation}\label{eq:reachabilityCriteria}
c_i(t, iri, Q) \rightarrow \{\mathrm{true}, \mathrm{false}\}
\end{equation}

\subsection{Decentralized Knowledge Graphs and subweb}\label{sec:dkg}

We define a DKG as a KG $G$ materialized in a network of resources $R$.
A resource $r_i \in R$ is map to a KG $g_i \subseteq G$, which is a set of triples~\cite{w3ConceptsAbstract}.
We denote this mapping $r_i \mapsto_{g} g_i$.
A resource is mapped and exposed by an IRI $iri_i$ denoted by $iri_i \mapsto_{iri} r_i$.
The network forms a graph where the resources $r_i$ are the nodes and the $iri_j \in g_i$ are directed edges starting from $r_i$ to $r_j$.
$G$ is formed by the union of all the $g \in \text{dom}(R)$.%, such that $G = \bigcup_{i=1}^{n}g_i$ given $n$ resources in the network.
A subweb is a (sub)DKG defined by a set of IRIs controlled by a data provider, which can offer a structure that aids query engine optimization.