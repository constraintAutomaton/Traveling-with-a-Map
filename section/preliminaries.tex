\section{Preliminaries}

\sepfootnotecontent{sf:sparqlSpecification}{
    The SPARQL specification is available at the following link \newline\href{https://www.w3.org/TR/sparql11-query/}{https://www.w3.org/TR/sparql11-query/}
}

\sepfootnotecontent{sf:blankNode}{
 A blank node is a unique idenfier with no IRI \href{https://www.w3.org/wiki/BlankNodes}{https://www.w3.org/wiki/BlankNodes}.
}

\sepfootnotecontent{sf:propertyPathAlgebra}{
    \href{https://www.w3.org/TR/sparql11-query/\#PropertyPathPatterns}{https://www.w3.org/TR/sparql11-query/\#PropertyPathPatterns}
}

\subsection{RDF knowledge graph and SPARQL queries}
Our work focuses on conjunctive queries $Q$ of RDF knowledge graph.
An RDF knowledge graph is composed of triples $t$ defined in Definition~\ref{def:triple} where the subjects and objects are nodes of the graph and the predicates are the edges.

\begin{definition}[RDF Triple]\label{def:triple}
    An RDF triple $t = (s,p,o)$ is formed by a subject term $s \in \mathcal{I} \cup \mathcal{B}$, a predicate term  $p \in \mathcal{I}$ and an object term $o \in \mathcal{I} \cup \mathcal{B} \cup \mathcal{L}$
    where $\mathcal{I}$, $\mathcal{B}$, $\mathcal{L}$ are respectiverly the set of every possible IRI, blank node~\sepfootnote{sf:blankNode} and litteral.
\end{definition}

We consider two parts of $Q$~\sepfootnote{sf:sparqlSpecification} the basic graph pattern (BGP) defined in Definition~\ref{def:bgp} and the filter expressions as defined in Defintion~\ref{def:filterExpression}.
The atomic statement of a BGP is a triple pattern $tp$ defined at Definition~\ref{def:triplePattern}.

\begin{definition}[Triple pattern]\label{def:triplePattern}
    A triple pattern $tp = (s_{tp},p_{tp},o_{tp})$ is formed by a subject term $s_{tp} \in \mathcal{I} \cup \mathcal{B} \cup \mathcal{V}$, 
    a property path  $p_{tp}$ defined in Definition~\ref{def:propertyPath} and an object term  $o_{tp} \in \mathcal{I} \cup \mathcal{L} \cup \mathcal{V}$ 
    where $\mathcal{V}$ is the set of every possible variable. 
\end{definition}

\begin{definition}[Property path]\label{def:propertyPath}
   A property path $p_{tp}$~\cite{Kostylev2015} is an expression that describe the route of predicate $p$ from $s_{tp}$ to $o_{tp}$.
   A property path in $tp$ is defined as follow:
   \begin{equation}
    p_{tp} ::= p \in \mathcal{I} | (p_{tpi}/p_{tpj}) | (p_{tpi}|p_{tpj}) | p_{tpi}* | p_{tpi}+ | p_{tpi}? | !p_{tpi}| p_{tpi}^{-}
   \end{equation}.
   The "$/$" operator chain two property path, the alternative operator "$|$" define a possibility between $p_{tpi}$ and $p_{tpj}$.
   The "$^-$" operator inverse the path from $s_{tp}$ to $o_{tp}$.
   The "$!$" represent the negation of a path $p_{tpi}$ such as $p_{tpi} = \mathcal{I} \setminus p_{tpi}$.
   The "$*$", "$+$" and "$?$" are respectively, the repetition of 0 and more of $p_{tpi}$, 
   the repetition of 1 and more of $p_{tpi}$ and the presence of absence of $p_{tpi}$.
\end{definition}

Property path can be interpreted in term of relational algebra for SPARQL evaluation as define by \citeauthor{Kostylev2015} (Definition 2) and the SPARQL specification (section 18.2)~\sepfootnote{sf:propertyPathAlgebra}.
Notably, the alternative operator can be interpreted as a union and the sequence as another triple pattern.

\begin{definition}[BGP]\label{def:bgp}
 A BGP $B$ is a set of $tp$, \texttt{OPTIONAL} and \texttt{UNION} (both clause containing a sets of $tp$).
 $tp$ sets can be divided into star pattern $Q_{star}$ defined in Definition~\ref{def:starPattern} in a way 
 that $Q = Q_{star_{i-1}} \bowtie Q_{star_i} \forall i \in S$ given that $S$ is the set of every subject of the $tp$ in the $Q$.
\end{definition}

\begin{definition}[Filter expression]\label{def:filterExpression}
    A filter expression $F \rightarrow \{\mathrm{true}, \mathrm{false}\}$ is chain of boolean expressions $f_i(p_1, p_2...,p_n) \rightarrow \{\mathrm{true}, \mathrm{false}\}$ such that 
    $p_i \in \mathcal{I} \cup \mathcal{L} \cup (\mathcal{V}\in Q)$.
\end{definition}

\begin{definition}[Star pattern with dependencies]\label{def:starPattern}
We define a star pattern $Q_{star}$ as a set of $tp \in Q$ with the same subject such has $Q_{star} = \{ tp\in Q| s_{tp_1} = s_{tp_2} ... = s_{tp_n}\}$.
We define a star pattern with dependencies $Q_{starD}$ as the union between a star pattern $Q_{star} \in Q$ and other star patterns having as subjects
the objects of the initial star pattern such as $Q_{starD} = Q_{star} \cup \{tp \in Q| o_{i} = s_{tp} \land o_{i}\in tp_i \land tp_i  \in Q_{star}\}$.
\end{definition}

\subsection{Reachability criteria}

\emph{Reachability criteria} are boolean functions ($c_i$) restricting the dereferencing of links from the internal data source of the query engine.
They are take as parameters an RDF triple $t$ from the internal triple store, an dereferenciable IRI $i_d$ from $t$ and the a basic query pattern $B$ from the query~\cite{Hartig2012}.
If $c_i$ return $true$ then the query engine try to dererefence $i_d$.
A formalization is presented in equation~\ref{eq:reachabilityCriteria}.


\begin{equation}\label{eq:reachabilityCriteria}
c_i(t, i_d, B) \rightarrow \{\mathrm{true}, \mathrm{false}\}
\end{equation}

As they are boolean function reachability criteria can be chained together to form \emph{Composite reachability criteria}.
In this form a reachability criterion ($cp_i$) is said to \emph{prune} links if it is chained with an \emph{and} operator with the other criterion ($cd_i$) and it is said to \emph{discover} links if it chain with an \emph{or} operator.
Equation~\ref{eq:cReachabilityCriteria} formalize a composite reachability criterion $C$ with $nd$ discovery criteria and $np$ prunning criteria.

\begin{equation}\label{eq:cReachabilityCriteria}
    C(t, i_d, B)  = \bigvee_{i=0}^{nd}cd_i(t, i_d, B) \land \bigvee_{j=0}^{np}cp_j(t, i_d, B)
\end{equation}

The derefencing operation of the query stop when during an interation of the new element of the internal triple store no new $i_d$ has to be derefenced.

\subsection{RDF data shapes}
RDF data shapes are schema on a set of RDF data (RDF graph) defining constraint in the presence of predicated linked
to the same subject.
In this formalization, we make no distinction between the shapes languages, but it was strongly inspired by ShEx~\cite{Gayo2018}.

We define a shape
\begin{equation}
S = (E, op \in \{\mathrm{true},\mathrm{false}\})
\end{equation}

similarly to \citeauthor{Abbas2018}.
$E$ is a set of shape expression 
\begin{align}
        e_i = (p, c, r, n) \\
        e_i = e_j|e_k
\end{align}
.
A shape expression is a tuple of constraints on the predicate and the object of one triple of a set of triples must respect. 
A shape expressions define a direct predicate $p$ constraint,
a constraint on the object $c(o) \rightarrow \{\mathrm{true}, \mathrm{false}\}$,
the constraint can be based on the type of the object or for it to respect another shape,
a constraint on the cardinality range of the predicate similar to a property path, with a minimum and a maximum $r = [\mathbb{R}_+, \mathbb{R}_+^*]$.
The property path can also be a negation $n \in \{\mathrm{true}, \mathrm{false}\}$ meaning that the constraint $e_i$ must not apply.
A shape expression can also be a disjunction, between two other shape expressions.
$op$ describes whether the shape is opened or closed.
An open shape allowed the presence of non-contradicting properties not defined in $E$.
