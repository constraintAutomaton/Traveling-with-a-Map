\begin{abstract}
    % Context
    Centralization of web information causes \rt{can cause} legal and ethical problems, especially in the context of social applications.
    % Need
    Decentralizing this information offers a potential solution, but maintaining query performance remains a challenge.
    Link Traversal Query Processing (LTQP) supports decentralized querying \rt{The querying is not decentralized. We're querying over decentralized data} in large scale networks but is limited by long execution times and high data usage.
    \rt{A sentence inbetween these two is needed to clarify the need a bit better: exec times are high because of too many unnecessary http requests that are being done.}
    % Task
    This paper presents the \emph{shape index}, a lightweight RDF data summarization method using RDF data shapes to reduce the search space for traversal queries.\rt{From this sentence, it's not clear yet that this index is exposed as a REST/hypermedia interface.}
    \rt{I wonder if this is really your main task. I think we can generalize this a bit and say that our task is the introduction of a shape-based link pruning approach. And this approach consists of two parts: a shape index (that exposes important metadata), and a LTQP pruning algorithm (that relies upon this metadata)}
    % Object
    In this work, we formalize and evaluate the shape index data summarisation approach and a \emph{query-shape containment} problem to detect irrelevant sources during LTQP.\rt{And we introduce a pruning algorithm for LTQP!}
    % Findings
    Our findings show that shape-based data summarisation can reduce the query execution time \rt{of LTQP?} and network usage of selective queries \rt{By how much?}, even in networks with missing \rt{What does missing information mean?} or lower-quality information.
    % Conclusion
    This performance gain, without server involvement during querying \rt{But the server is involved right? For exposing the shape index? I suspect you want to say that server effort is increased only minimally? (by how much?)}, makes our approach a good candidate for selective queries in networks of structured, decentralized datasets \rt{Knowledge Graphs?}.

\keywords{Linked data,
Link Traversal Query Processing,
RDF data shapes,
Highly decentralized querying, \rt{"Decentralization"}
Data summarisation \rt{Make sure to be consistent. Sometimes you use "summarisation" and sometimes "summarization"}}

\end{abstract}
