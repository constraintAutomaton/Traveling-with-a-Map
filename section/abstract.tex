\begin{abstract}
% Context  
The centralization of web information raises legal and ethical concerns, particularly in social, healthcare and education applications.
% Need  
Decentralized architectures offer a promising alternative
by keeping data closer to its source,
but efficient query processing remains a challenge.  
Link Traversal Query Processing (LTQP) enables querying across decentralized networks, yet suffers from long execution times and high data transfer volumes due to a~substantial number of HTTP requests.  
% Task  
We propose a shape-based pruning approach that relies on \emph{shape indexes} and a \emph{query-shape subsumption} algorithm to reduce the search space and, consequently, the number of HTTP requests.
% Object  
We formalize this approach as a link pruning mechanism for LTQP and evaluate its effectiveness on social media queries using the SolidBench benchmark across multiple evaluation metrics.
% Findings  
Our results show that shape-based pruning improves query execution time and reduces network usage up to 7 times compared to the state of the art,
\rv{absolute numbers possible here?}
in exchange for a minor increase in the number of triples per shape-index instance.
\rv{be specific here}
\rv{Suggestion to rephrase:
Our results show that,
if sources are willing to … in a~shape index,
query processing time improves up to…
}
% Conclusion  
This work demonstrates that shape-based metadata can significantly optimize LTQP queries in decentralized knowledge graphs.
By exposing such metadata, data providers not only improve their data quality and interoperability but also enhance the processing efficiency of traversal queries.
\rv{But that's not the conclusion, showing potential. The conclusion is what readers can now do as a result. So basically, more along the lines of: decentralized scenarios become more realistic, given that query times… (but also begs the question earlier on, the times 7 reduction, how far still from technical optimum of legally annoying centralization?}
\end{abstract}
