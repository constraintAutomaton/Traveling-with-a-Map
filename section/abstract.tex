\begin{abstract}
% Context  
The centralization of web information raises legal and ethical concerns, particularly in social, healthcare and education applications.
% Need  
Decentralized architectures, offer a promising alternative, but efficient query processing remains a challenge.  
Link Traversal Query Processing (LTQP) enables querying across decentralized networks but suffers from long execution times and high data transfer due to excessive HTTP requests.  
% Task  
We propose a shape-based pruning approach that relies on \emph{shape indexes} and a \emph{query-shape subsumption} algorithm to reduce the search space and consequently, the number of HTTP requests.
% Object  
We formalize this method as a link pruning mechanism for LTQP and evaluate its impact on social media queries using the SolidBench benchmark.  
% Findings  
Our results show that shape-based pruning improves query execution time and reduces network usage up to 7 times compared to the state of the art, in exchange of a minor increase in the number of triples per shape-index instance.
% Conclusion  
This work demonstrates the potential of shape-based metadata for optimizing LTQP queries in decentralized knowledge graphs, going beyond its traditional use on data validation.

\keywords{Linked Data,
Link Traversal Query Processing,
RDF data shapes,
Decentralization,
Data summarization
}

\end{abstract}
