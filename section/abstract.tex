\begin{abstract}
    % Context
    Centralization of web information can raise legal and ethical problems, especially in the context of social applications.
    % Need
    Decentralizing this information offers a potential solution, but achieving acceptable query performance remains a challenge.
    Link Traversal Query Processing (LTQP) enables querying in large-scale networks of decentralized data but suffers from long execution times and high data transfer,
    largely due to the extensive number of HTTP requests required for network exploration.
    % Task
    To solve this problem, we introduce a shape-based pruning approach to minimize the search space of traversal queries.
    The approach utilizes \emph{shape indexes} provided by data providers in networks of decentralized knowledge graphs to reduce the search space using a \emph{query-shape containment} algorithm.  
    % Object
    In this article, we formalize this shape index and query-shape containment approach as a link pruning mechanism for LTQP,
    and evaluate its impact on the performance of queries in a social media context.
    % Findings
    Our findings show that shape-based link pruning can reduce the query execution time and network usage of selective queries by up to 7 times.\rt{Can you also mention if there is an increase in server cost? If not, also mention that.}
    % Conclusion
    Our work shows the benefits of exposing shape-based metadata for handling selective LTQP queries in large networks of structured, decentralized knowledge graphs.

\keywords{Linked Data,
Link Traversal Query Processing,
RDF data shapes,
Decentralization,
Data summarization
}

\end{abstract}
