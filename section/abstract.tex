\begin{abstract}
    % Context
    Centralization of web information can raise legal and ethical problems, especially in the context of social applications.
    % Need
    Decentralizing this information offers a potential solution, but maintaining query performance remains a challenge.
    Link Traversal Query Processing (LTQP) enables querying in large-scale networks of decentralized data but suffers from long execution times and high data usage,
    largely due to the extensive HTTP requests required for network exploration.
    % Task
    This paper introduces a shape-based pruning approach to minimize the search space of traversal queries.
    The approach utilizes \emph{shape indexes} provided by data providers in networks of decentralized knowledge graphs to reduce the search space using a \emph{query-shape containment} algorithm.  
    % Object
    This work introduces link pruning in LTQP by formalizing the shape index and query-shape containment approach, and evaluates its impact on the performance of traversal queries.
    % Findings
    Our findings show that shape-based data summarization can reduce the query execution time and network usage of selective traversal queries by up to 7 times in our benchmark.
    % Conclusion
    This performance gain, achieved without delegating queries to endpoints, makes our approach a strong candidate for handling selective queries in large networks of structured, decentralized knowledge Graphs.

\keywords{Linked data,
Link Traversal Query Processing,
RDF data shapes,
Decentralization,
Data summarization
}

\end{abstract}
