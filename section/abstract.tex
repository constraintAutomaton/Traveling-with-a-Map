\begin{abstract}
    % Context     
    Link traversal queries are notoriously slow.
    % Need
    The main bottleneck of this query paradigm is the pseudo-infinite traversal search space.
    % Task
    Using data publication structure, traversal queries can be made faster by creating a finite search domain.
    Yet, the search domain can remain needlessly large regarding specific queries, hindering query engine performance.    
    % Object      
    In this paper, we propose a lightweight and low-maintenance data summary method called a \emph{Shape Index} leveraging 
    the descriptive power of RDF data shapes with
    an associated query-shape containment algorithm for online source selection during link traversal queries.
    We formalize our approach and evaluate its impact on different query processing metrics considering multiple 
    level of detail of data summarisation in the network.  
    % Findings    
    We show that the use of this simple data sumarization can have a significant positive impact on query execution,
    and that our algorithm as low overhead and is easlily adaptative when part of the network does not expose the index.
    % Conclusion
    Given the performance gain and its low maintenance a shape index can be in the context 
    of the publication of descentralized RDF dataset a good low-cost way to improve query performance.
    % Perspectives
    Given our results, it would be interesting to explore the creation of a more restrictive algorithm by handling filter expression for source selection or a more detailed shape index by formalizing the 
    absence of RDF subgraphs respecting a particular shape and, more interestingly, investigating the use of a
    shape index for adaptative query planning during link traversal queries.
\end{abstract}
