\begin{abstract}
    % Context     
    Link traversal queries are notoriously slow, primarily due to their expansive search space.
    % Need
    % Task
    By leveraging the structure of the publication of data and trust policies, traversal queries can be made faster by imposing a manageable finite domaine.
    Yet, the search space can remain needlessly large when considering that multiple documents may not be query-relevant.
    % Object      
    This paper proposes a lightweight, low-maintenance data summary method for decentralized RDF documents called a \emph{Shape Index}.
    The shape index leverages the descriptive power of RDF data shapes by mapping them to sets of RDF resources to summarise decentralized RDF datasets.
    With this simple information, we propose to use a query-shape containment problem as a mechanism for online source selection during link traversal queries
    to reduce query-irrelevant documents acquisition.
    This paper formalizes our approach and evaluates its impact on different query-processing metrics with multiple scenarios. 
    % Findings    
    We show that this simple data summarization can reduce the query execution time and network usage of queries where a significant number of sources are not query-relevant 
    while not producing a large overhead.    
    Our findings also indicate that the relation between the decrease in the number of HTTP requests correlated none linearly with the reduction in query execution time,
    leading us to the same conclusion as the state of the art about the proposition that better query planning could positively impact the query execution time of link traversal queries.
    % Conclusion
    With low provider maintenance and computational requirements, the Shape Index offers an effective, scalable means of improving query performance for decentralized RDF datasets.
    % Perspectives
    The paper leaves open the questions of performing query-shape containment problems with filter expression for more selective source selection, 
    of using negative entries in the shape index in the source selection process and the shape index for adaptative query planning.

\keywords{Linked data,
Link Traversal Query Processing,
Query containment,
RDF data shapes,
Data summarisation,
Descentralized environments}

\end{abstract}
