\begin{abstract}
% Context  
The centralization of web information raises legal and ethical concerns, particularly in social applications.  
% Need  
Decentralization offers a promising alternative, but efficient query performance remains a challenge.  
Link Traversal Query Processing (LTQP) enables querying in decentralized networks but suffers from long execution times and high data transfer due to excessive HTTP requests.  
% Task  
We propose a shape-based pruning approach that utilizes \emph{shape indexes} and a \emph{query-shape containment} algorithm to reduce the search space and, consequently, the number of HTTP requests.
% Object  
We formalize this method as a link pruning mechanism for LTQP and evaluate its impact on social media queries using the Solidbench benchmark.  
% Findings  
Our results show that shape-based pruning improved query execution time and network usage by up to 7 times compared to the state of the art, with the trade-off of increased server costs for hosting shape indexes.  
% Conclusion  
This work demonstrates the potential of shape-based metadata for optimizing selective LTQP queries in decentralized knowledge graphs.  



\keywords{Linked Data,
Link Traversal Query Processing,
RDF data shapes,
Decentralization,
Data summarization
}

\end{abstract}
