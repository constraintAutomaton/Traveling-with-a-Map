\section{Conclusion}

We proposed optimizing the search space of traversal queries by leveraging a shape index, which is a mapping between RDF shapes and a set of resources, as a mechanism for source selection.
The primary mechanism for source selection that we propose is to execute a query-shape containment problem similar to the classic query containment problem; we also 
propose an extension of the reachability criteria concept to include the sources' pruning.
From our experiment for highly selective queries, we can notice a high improvement in query execution time with a marginal decrease in the performance of nonselective queries 
and an important degradation of performance for queries that do not exploit the structural properties of datasets.
We have also shown that our approach is adaptative and can operate in networks with partial shape index information or shapes with minimally described shapes and still benefit
from the information provided by the network to reduce the execution time.
Shape indexes are small descriptive summaries of decentralized datasets; this work did not discuss their production. However,
there exists work in the literature for automatic generation of RDF datashapes~\cite{fernandez2023extracting}, we also 
did not discuss their maintenance, but work in the literature for shape-based data integration also exists~\cite{LabraGayo2023}.
Given the descriptive power of shapes and their use in centralized setting~\cite{kashif2021} for query planning, interesting future work could be to investigate adaptatively
query planning in LTQP~\cite{taelman2024towards} using shapes.   