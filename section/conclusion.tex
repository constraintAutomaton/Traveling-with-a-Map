\section{Conclusion}

In this article, we minimized the search space for link traversal queries by leveraging a shape index and addressing the query-shape containment problem.
Additionally, we introduced pruning in LTQP by extending the concept of reachability criteria.
Using the Solidbench benchmark, we demonstrated that our approach can improve query execution times by up to 7 times.
Our adaptive method effectively handles scenarios with reduced information in shape indexes or their partial absence in a network.
This study highlights that shape-based pruning can be highly effective for LTQP in decentralized environments with structural properties, especially for selective queries.
These findings are particularly relevant for decentralization initiatives that aim to enable users or third-party clients to perform efficient queries over large, diverse networks.
Future work could explore further advancements in this area, such as enhancing query planning in LTQP~\cite{taelman2024towards} with RDF data shapes.
