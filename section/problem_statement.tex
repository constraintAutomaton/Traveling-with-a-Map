\iffalse
\section{Problem Statement}\label{sec:problem_statement}

We pose the following research question: 
\textbf{Can LTQP use shape-based pruning in DKG networks to reduce query execution time while preserving result completeness?}
We propose the following hypotheses:  
(\textbf{H1}) Shape indexes reduce the number of non-contributing data sources retrieved and the query execution time.  
(\textbf{H2}) The execution time of a query-shape subsumption algorithm is negligible in social media applications.  
(\textbf{H3}) Stricter shape constraints lead to a greater reduction in HTTP requests.  
(\textbf{H4}) Querying a network with more \emph{complete} shape indexes results in faster query execution.  
(\textbf{H5}) Performance gain can be acquired in networks with less shape index information.



\newcounter{hypothesisCounter}
\setcounter{hypothesisCounter}{1}

\begin{itemize}[label=\textbf{H\arabic{hypothesisCounter}}\,\stepcounter{hypothesisCounter}]
    \item Using shape indexes will reduce the number of non-contributing data sources acquired
    \item Query-shape subsumption execution time is negligible in the context of social media applications
    \item Shapes with stricter constraints result in a greater reduction in HTTP requests.
    \item A network with a more \emph{complete} shape index will reduce more the number of HTTP requests and the query execution time than one with less
    \item The shape index approach can be adaptative, so not every dataset in the network needs to have an index to see a performance improvement
\end{itemize}
\fi