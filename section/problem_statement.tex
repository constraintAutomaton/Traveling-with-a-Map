\section{Problem Statement}\label{sec:problem_statement}
To guide our study we formulated the following research question.
\textbf{Can LTQP use shape-based pruning in networks of decentralized Knowledge Graphs to reduce the query execution time while maintaining the same completeness of results?}
We formulated the following hypotheses:
\newcounter{hypothesisCounter}
\setcounter{hypothesisCounter}{1}

\begin{itemize}[label=\textbf{H\arabic{hypothesisCounter}}\,\stepcounter{hypothesisCounter}]
    \item Using shape indexes will reduce the number of non-contributing data sources acquired
    \item There is a linear correlation between the reduction of the number of HTTP requests and the reduction of query execution time
    \item Query-shape containment execution time is negligible in the context of social media applications
    \item More restrictive shapes will provide a higher reduction in the number of HTTP requests
    \item A network with a more \emph{complete} shape index will reduce more the number of HTTP requests and the query execution time than one with less
    \item The shape index approach can be adaptative, so not every dataset in the network needs to have an index to see a performance improvement
\end{itemize}
