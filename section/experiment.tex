\section{Experimental Setup}

\sepfootnotecontent{sf:implementationComunica}{
   \ifanonymous
      \url{https://anonymous.4open.science/r/comunica-feature-link-traversal-AE1C}
   \else
      \url{https://github.com/constraintAutomaton/comunica-feature-link-traversal/tree/feature/shapeIndex}
   \fi
}

\sepfootnotecontent{sf:solidbench}{
   \url{https://github.com/SolidBench/SolidBench.js}
}

\sepfootnotecontent{sf:implementationQueryShapeContainment}{
   \ifanonymous
      \url{https://anonymous.4open.science/r/query-shape-detection-ED87/}
   \else
      \url{https://github.com/constraintAutomaton/query-shape-detection}
   \fi
}

\sepfootnotecontent{sf:shapeIndexGenerator}{
   \ifanonymous
      \url{https://anonymous.4open.science/r/rdf-dataset-fragmenter_js-08B9}
   \else
      \url{https://github.com/constraintAutomaton/rdf-dataset-fragmenter.js/tree/feature/shape-index-fragmentation-strategy}
   \fi
}

\sepfootnotecontent{sf:complementaryMaterial}{
   \ifanonymous
      \url{https://anonymous.4open.science/r/documentation-1A65}
   \else
      \url{https://github.com/shapeIndexComunicaExperiment/documentation}
   \fi
}

We implemented our approach using the LTQP version of the Comunica query engine~\cite{taelman_iswc_resources_comunica_2018}.
We chose Comunica due to its modularity~\cite{taelman_swj_componentsjs_2022} and its established use in several LTQP studies~\cite{Bogaerts2021LinkTW, Taelman2023, eschauzier_quweda_linkqueue_2023, Hanski2024, eschauzier_amw_rcubemetric_2024, tam2024opportunitiesshapebasedoptimizationlink}.
The implementation of our shape index approach is open source~\sepfootnote{sf:implementationComunica}, as well as our query-shape containment solver~\sepfootnote{sf:implementationQueryShapeContainment}.
We use SolidBench~\cite{Taelman2023}, based on the LDBC social network benchmark~\cite{Angles2020}, to evaluate our contribution. 
To facilitate this, we created an open-source module~\sepfootnote{sf:shapeIndexGenerator} to generate shape indexes in SolidBench, based on user-provided mappings between ShEx shapes and data model objects.
The shape annotated portion of the data model includes posts, comments on posts, user profiles, cities, and likes.
The datasets are Solid Pods~\cite{Taelman2023}
Each Solid Pod contains a shape index and separate files for each shape definition.
Some shapes are nested within others. 
For example, profiles are associated with cities, and comment are associated with posts.
Depending on the pod instance, certain data model objects are materialized in a single file, while others are distributed across multiple files.
The entire data model and query templates are available online~\sepfootnote{sf:solidbench}.

To evaluate our approach, we conducted the following experiments, we first  measured the execution time and results of our query-shape containment algorithm using the shapes from the study.
Then, We compared the Solid Pod network optimal traversal algorithm, which uses the LDP specification and type-index~\cite{Taelman2023}, with the LDP traversal algorithm~\cite{Taelman2023} and our shape index approach in a network where each Solid Pods (datasets) provides a complete shape index with the most descriptive shapes.
Additionally, we evaluated the adaptivity of our approach by reducing the shape index information across the network:
\begin{itemize}
   \item We compared the impact of query execution time in a network where 0\%, 20\%, 50\%, and 80\% of Solid Pods expose a shape index.
   \item We compared the impact of using shape indexes with 20\%, 50\%, and 80\% of entries using closed shapes.
   \item We compared the impact of using shapes that incorporate only data from the Solid Pods, and shapes providing a minimal dataset description where the object constraints are always an IRI or a literal.
\end{itemize}
We used query templates from SolidBench, each with five instances varying the starting pod.
Experiments were repeated 50 times with a 2 minute timeout (120,000 ms). 
They were conducted on an Ubuntu 20.04.6 LTS machine with a 2x Hexacore Intel E5645 CPU and 24GB RAM.
All experiments are reproducible, with raw data and complementary materials available online~\sepfootnote{sf:complementaryMaterial}.

%https://doc.ilabt.imec.be/ilabt/virtualwall/hardware.html#virtual-wall-2

