\section{Experimental Setup}

\sepfootnotecontent{sf:implementationComunica}{
   \ifanonymous
      \url{https://anonymous.4open.science/r/comunica-feature-link-traversal-AE1C}
   \else
      \url{https://github.com/constraintAutomaton/comunica-feature-link-traversal/tree/feature/shapeIndex}
   \fi
}

\sepfootnotecontent{sf:implementationQueryShapeContainment}{
   \ifanonymous
      \url{https://anonymous.4open.science/r/query-shape-detection-ED87/}
   \else
      \url{https://github.com/constraintAutomaton/query-shape-detection}
   \fi
}

\sepfootnotecontent{sf:shapeIndexGenerator}{
   \ifanonymous
      \url{https://anonymous.4open.science/r/rdf-dataset-fragmenter_js-08B9}
   \else
      \url{https://github.com/constraintAutomaton/rdf-dataset-fragmenter.js/tree/feature/shape-index-fragmentation-strategy}
   \fi
}

\sepfootnotecontent{sf:complementaryMaterial}{
   \ifanonymous
      \url{https://anonymous.4open.science/r/documentation-1A65}
   \else
      \url{https://github.com/shapeIndexComunicaExperiment/documentation}
   \fi
}

We implemented our approach using the LTQP version of the Comunica query engine~\cite{taelman_iswc_resources_comunica_2018}. 
We chose Comunica for its modularity~\cite{taelman_swj_componentsjs_2022} and its established use in multiple LTQP studies~\cite{Bogaerts2021LinkTW, Taelman2023, eschauzier_quweda_linkqueue_2023, Hanski2024, eschauzier_amw_rcubemetric_2024, tam2024opportunitiesshapebasedoptimizationlink} \rt{I'd expect some of these to be discussed in related work.}.
The implementation of our reachability criteria and the link queue filter is open source~\sepfootnote{sf:implementationComunica}, as well as our query-shape containment solver.~\sepfootnote{sf:implementationQueryShapeContainment}
We use SolidBench~\cite{Taelman2023}, which is based on the LDBC social network benchmark~\cite{Angles2020} to evaluate our contribution.
We created an open source module~\sepfootnote{sf:shapeIndexGenerator} to generate shape indexes in SolidBench based on user-provided mapping between ShEx shapes and data model objects.\rt{This needs to be discussed in detail! What does the data model look like, how does it link to the shapes, examples, ...}

\rt{Big problem: you've never mentioned Solid up until now!}

\rt{The following is lacking many details that are necessary to understand the experimental setup. I recommend expanding this section, and perhaps creating subsections for each experiment (or an enumeration).}
We designed several experiments to analyze and evaluate our approach.
First, we measure the execution time and results of our query-shape containment algorithm using the shapes from the study \rt{Which shapes are this exactly?}.
We also compare the state-of-the-art traversal algorithm \rt{I would not call this the state-of-the-art traversal algorithm. Just use the citation, or call it the optimal algorithm that does LTQP based on structural assumptions}, the LDP specification and type-index, and the LDP specification~\cite{Taelman2023} with our shape index approach in a network where every dataset provides a complete shape index with the most descriptive shapes.
Additionally, we evaluate the adaptivity of our approach by modulating the shape index information across the network.\rt{Unclear what this means.}
In one experiment, we examine the impact of query execution time in a network where 0\%, 20\%, 50\%, and 80\% of datasets \rt{What are datasets? I guess pods?} expose a shape index.
In another experiment, we assess the effect of having shape indexes with 20\%, 50\%, and 80\% entries using closed shapes.
In the final experiment, we evaluate the impact of using shapes that only incorporated data from the dataset itself and shapes that provided a minimal description of the data.\rt{Unclear what this means.}
For all experiments, we performed 50 repetitions and set a timeout of 2 minutes (120,000 ms).
To run our experiments, we used a machine with Ubuntu 20.04.6 LTS, featuring a 2x Hexacore Intel E5645 CPU and 24GB of RAM.
All our experiments are fully reproducible, and all raw data is available online~\sepfootnote{sf:complementaryMaterial}.

\iffalse
PROVIDE LINK FOR EXPERIMENTS AND SHAPES

102x pcgen3 nodes
https://doc.ilabt.imec.be/ilabt/virtualwall/hardware.html#virtual-wall-2
\fi
