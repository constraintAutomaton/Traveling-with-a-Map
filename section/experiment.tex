\section{Experiment}

\sepfootnotecontent{sf:implementationComunica}{
   \ifanonymous
      \url{https://anonymous.4open.science/r/comunica-feature-link-traversal-AE1C}
   \else
      \url{https://github.com/constraintAutomaton/comunica-feature-link-traversal/tree/feature/shapeIndex}
   \fi
}

\sepfootnotecontent{sf:implementationQueryShapeContainment}{
   \ifanonymous
      \url{https://anonymous.4open.science/r/query-shape-detection-ED87/}
   \else
      \url{https://github.com/constraintAutomaton/query-shape-detection}
   \fi
}

\sepfootnotecontent{sf:shapeIndexGenerator}{
   \ifanonymous
      \url{https://anonymous.4open.science/r/rdf-dataset-fragmenter_js-08B9}
   \else
      \url{https://github.com/constraintAutomaton/rdf-dataset-fragmenter.js/tree/feature/shape-index-fragmentation-strategy}
   \fi
}

\sepfootnotecontent{sf:complementaryMaterial}{
   \ifanonymous
      \url{https://anonymous.4open.science/r/documentation-1A65}
   \else
      \url{https://github.com/shapeIndexComunicaExperiment/documentation}
   \fi
}

We implemented our approach using the LTQP version of the Comunica query engine~\cite{taelman_iswc_resources_comunica_2018}. 
We chose Comunica for its modularity~\cite{taelman_swj_componentsjs_2022} and its established use in multiple LTQP studies~\cite{Bogaerts2021LinkTW, Taelman2023, eschauzier_quweda_linkqueue_2023, Hanski2024, eschauzier_amw_rcubemetric_2024, tam2024opportunitiesshapebasedoptimizationlink}.
The implementation of our reachability criteria and the link queue filter is open source~\sepfootnote{sf:implementationComunica}, as well as our query-shape containment solver.~\sepfootnote{sf:implementationQueryShapeContainment}
We use the social benchmark Solidbench~\cite{Taelman2023} which is based on the LDBC benchmark~\cite{Angles2020} to evaluate our contribution.
We created an open source module~\sepfootnote{sf:shapeIndexGenerator} to generate shape indexes in Solidbench based on user-provided mapping between ShEx shapes and data model objects.

We designed several experiments to analyze and evaluate our approach.
First, we assessed the execution time and results of our query-shape containment algorithm using the shapes from the study.
We also compared the state-of-the-art traversal algorithm, the LDP specification and type-index, and the LDP specification~\cite{Taelman2023} with our shape index approach in a network where every dataset provides a complete shape index with the most descriptive shapes.
Additionally, we evaluated the adaptivity of our approach by modulating the shape index information across the network.
In one experiment, we examined the impact of query execution time in a network where 0\%, 20\%, 50\%, and 80\% of datasets exposed a shape index.
In another experiment, we assessed the effect of having shape indexes with 20\%, 50\%, and 80\% entries using closed shapes.
In the final experiment, we evaluated the impact of using shapes that only incorporated data from the dataset itself and shapes that provided a minimal description of the data.
For all experiments, we performed 50 repetitions and set a timeout of 2 minutes (120,000 ms).
To run our experiments, we used a machine with Ubuntu 20.04.6 LTS, featuring a 2x Hexacore Intel E5645 CPU and 24GB of RAM.
We provide complementary material online, including experimental setups, raw data, and additional plots and tables of our results.~\sepfootnote{sf:complementaryMaterial}

\iffalse
PROVIDE LINK FOR EXPERIMENTS AND SHAPES

102x pcgen3 nodes
https://doc.ilabt.imec.be/ilabt/virtualwall/hardware.html#virtual-wall-2
\fi
