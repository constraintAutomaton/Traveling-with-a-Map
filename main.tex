% sage_latex_guidelines.tex V1.20, 14 January 2017

\documentclass[Afour,sageh,times]{sagej}

\usepackage{moreverb,url}

\usepackage[breaklinks=true, bookmarks=true, bookmarksdepth=3, colorlinks,bookmarksopen,bookmarksnumbered,citecolor=red,urlcolor=red]{hyperref}

\usepackage[T1]{fontenc}
% T1 fonts will be used to generate the final print and online PDFs,
% so please use T1 fonts in your manuscript whenever possible.
% Other font encondings may result in incorrect characters.
%
\usepackage{graphicx}

% Used for displaying a sample figure. If possible, figure files should
% be included in EPS format.
%
% If you use the hyperref package, please uncomment the following two lines
% to display URLs in blue roman font according to Springer's eBook style:
%\usepackage{color}
%\renewcommand\UrlFont{\color{blue}\rmfamily}
%\urlstyle{rm}
%
\usepackage[backend=biber]{biblatex}
%% Declare bibliography sources (one \addbibresource command per source)
\addbibresource{references.bib}

%%
\usepackage{listings}
%% auto break lines
\lstset{breaklines=true}
\usepackage{xcolor} % For coloring text
\newcommand{\pc}[1]{\textcolor{red}{Pieter: #1}}
\newcommand{\rt}[1]{\textcolor{blue}{RT: #1}}
\newcommand{\jr}[1]{\textcolor{orange}{JR: #1}}
\usepackage{svg}
\usepackage{sepfootnotes}
\usepackage{amsmath}
\usepackage{stmaryrd}
\usepackage{amsfonts}
\usepackage{amssymb}
\usepackage{algorithm}
\usepackage{algorithmic}
\usepackage{subfig}
\usepackage{placeins}
\usepackage{nameref}
%\usepackage[breaklinks=true, bookmarks=true, bookmarksopen=true, bookmarksdepth=3]{hyperref}
\usepackage{enumitem}
\usepackage{multirow}
\usepackage{amsthm}

\newtheorem{definition}{Definition}[section] % numbered per section

\newif\ifanonymous
\anonymoustrue

\setlength{\textfloatsep}{10pt plus 1.0pt minus 2.0pt}
\setlength{\floatsep}{8pt plus 1.0pt minus 2.0pt}
\setlength{\intextsep}{8pt plus 1.0pt minus 2.0pt}

\renewcommand{\algorithmicrequire}{\textbf{Input:}}
\renewcommand{\algorithmicensure}{\textbf{Output:}}


\newcommand\BibTeX{{\rmfamily B\kern-.05em \textsc{i\kern-.025em b}\kern-.08em
T\kern-.1667em\lower.7ex\hbox{E}\kern-.125emX}}

\def\volumeyear{2016}

\begin{document}

\runninghead{B.-E. Tam et al.}

\title{Traveling with a Map: Reducing the Search Space of Link Traversal Queries Using RDF Shapes}

\author{Bryan-Elliott Tam\affilnum{1}, Ruben Taelman\affilnum{1}, Joachim Van Herwegen\affilnum{1} and Pieter Colpaert\affilnum{1}}

\affiliation{\affilnum{1}Universiteit Gent, Ghent, Belgium}

\corrauth{Bryan-Elliott Tam}

\email{\{firstname.lastname\}@ugent.be}

\begin{abstract}
% Context  
The centralization of web information raises legal and ethical concerns, particularly in social applications\rt{I wouldn't say in particular, maybe for example, and then perhaps give another example.}.  
% Need  
Decentralization offers a promising alternative, but efficient query performance remains a challenge.  
Link Traversal Query Processing (LTQP) enables querying in decentralized networks but suffers from long execution times and high data transfer due to excessive HTTP requests.  
% Task  
We propose a shape-based pruning approach that utilizes \emph{shape indexes} and a \emph{query-shape subsumption} algorithm to reduce the search space and, consequently, the number of HTTP requests.
% Object  
We formalize this method as a link pruning mechanism for LTQP and evaluate its impact on social media queries using the Solidbench benchmark.  
% Findings  
Our results show that shape-based pruning improves query execution time and network usage by up to 7 times compared to the state of the art, at the cost of increased server costs for hosting shape indexes.\rt{Is this cost significant? If yes, can you give a number here?}  
% Conclusion  
This work demonstrates the potential of shape-based metadata for optimizing LTQP queries in decentralized knowledge graphs.  



\keywords{Linked Data,
Link Traversal Query Processing,
RDF data shapes,
Decentralization,
Data summarization
}

\end{abstract}


\keywords{Linked Data,
Link Traversal Query Processing,
RDF data shapes,
Decentralization,
Data summarization
}

\maketitle


\section{Introduction}
\sepfootnotecontent{sf:webID}{
    \url{https://www.w3.org/wiki/WebID}
}
\sepfootnotecontent{sf:dataSovereignty}{
    \url{https://digital-strategy.ec.europa.eu/en/policies/strategy-data}
}

Data sovereignty attempts to define a more just definition of personal data ownership in terms of data usage and storage.
It can be defined as ``the self-determination of individuals and organizations concerning to the use of their data''~\cite{verstraete2022solid},
which in practice can be interpreted as the power to choose where one's data is stored and who has access to it~\cite{verstraete2022solid}.
Multiple studies have denoted problems of ownership, democracy, reinforcement of inequality, and antagonism between users and owners of web social applications~\cite{Terranova2000FreeLP, Curran2016ch1, Sevignani2013, 9663788}.
Several authors consider decentralizing web data an insufficient solution~\cite{9663788, Curran2016ch1}; however, it can still be an integral component of initiatives focused on data sovereignty.

Linked data has the potential to create seamless decentralized databases through the use of dereferenciable IRIs.
These IRIs allow access to additional databases containing information relevant to the IRI.
For example, dereferencing a term representing a user like a WebID~\sepfootnote{sf:webID} can provide the name of the user, among other information, without having 
to store this information locally.
\iffalse
Such data publication paradigm gives the opportunity to potentially break down data silos~\cite{verstraete2022solid},
and empowers users by reducing the need for indexing strategies to find relevant information about a term. 
Furthermore, it can help address legal and ethical concerns around data centralization, a topic of actuality in multiple world regions.
\fi
Despite these potential advantages, most SPARQL query processing is conducted using centralized setups or federations of endpoints, partly due to performance issues with traversal queries.

To take advantage of the potential descriptive power of IRI dereferencing a query paradigm called Link Traversal Query Processing (LTQP)~\cite{Hartig2012} has been developed.
LTQP consists of recursively dereferencing IRI contained into the internal data source of a query engine during its query execution to expand its base of information.
A lookup policy can be used to limit the search domain of the query.
LTQP has multiple difficulties such as the open endlessness of the web, which can be interpreted as a pseudo-infinite domain of exploration,
and the challenge of creating an efficient query plan due to the lack of information about the data sources. 
It has been demonstrated by \citeauthor{hartig2016walking} that in the open web the main performance bottleneck and obstacle for query completeness and fast query execution time is not the query plan but the large number of the HTTP request necessary to fulfill a query.

From another perspective, and not disproving \citeauthor{hartig2016walking}, \citeauthor{Taelman2023} has demonstrated that in a Structured Linked Data Environment (SLDE),
it is possible to attain query completeness.
Furthermore, query planning could significantly influence the speed of execution.
A SLDE is defined has an RDF environment where in addition to the RDF principles, specifications
guarantee the completeness of results.
This guarantee of completeness has the positive side effect of providing information
that can be used to create a lookup policy to reduce the number of HTTP requests necessary to attain completeness.

In practice SLDE can be use in the context of personal data and social network among others,
exemples of environments respecting those constraints are dataset following the Solid protocol~\cite{Taelman2023} and TREE specification~\cite{tam_iswc_traversalsensortree_2024}.
The work of ~\citeauthor{Taelman2023} indicate that there are multiple optimizations possible in LTQP in the context SLDE as opposed to the
more pessimist conclusion of the work of ~\citeauthor{hartig2016walking}.
In line with this research direction, this paper formalizes the concepts of the \emph{Shape Index} and \emph{query-shape containment}~\cite{tam2024opportunitiesshapebasedoptimizationlink} to enable a query and data aware mechanism for reducing the search domain of link traversal queries.
The shape index concept relies on RDF data shapes; the conceptual idea of RDF data shapes is to describe the properties of an entity.
We propose to use them in an index to describe decentralized datasets.
Because shapes and queries share similarities, we propose transforming shapes into queries to perform a query containment problem
and assist our source selection.

To guide our study, we formulated this research question:
Can a link traversal query engine use shape indexes in networks of decentralized datasets to reduce the number of HTTP requests while maintaining the same completeness of results, and does this reduction of HTTP requests lead to a decrease in query execution time?
We formulate the following hypotheses:
Using shape indexes will reduce the number of non-contributing data sources acquired;
more detailed shapes will provide a higher reduction in the number of HTTP requests;
a \emph{complete} shape index will reduce the number of HTTP requests and the query execution time than an \emph{incomplete} shape index;
a shape index approach can be adaptative, so not every dataset in the network needs to have an index to see a performance improvement.
\iffalse
Can our method reduce the ratio of non-contributing data source dereferenced?
How does the diminution of HTTP request affect the query execution time?
How does the level of detail of the shapes impact the performances?
What is the difference in performance between a \emph{complete} and an \emph{incomplete} shape index?
How does the ratio of subdomains containing an index influence global performances?
How does the fragmentation of the subdomain impact the performance?
How does the quantity of non-query contributing resources impact the performances?
What is the ideal query execution time if we only dereferenced contributing data sources?
\fi
\section{Related work}

\rt{This section is very short. I would expect more references on things like using shapes in querying, using summaries (e.g. bloom filters) in SPARQL, the LDF axis, ...}
\rt{For everything you reference, also mention why it is or is not relevant to this work, and how.}

LTQP is a SPARQL querying paradigm that answers queries by exploring the web using the follow-your-nose principle~\cite{hartig2016walking}.
Query execution begins by dereferencing \emph{seed URLs}~\cite{hartig2016walking} and injecting triples from these sources into an internal triple store.
It then recursively dereferences links from the store while answering the query to provide the user with a stream of results.
The main challenge of LTQP is the web's open-ended nature and the lack of information for query planning.

RDF data shapes are used for validating, describing, and communicating data structures, as well as generating data and driving user interfaces~\cite{Gayo2018a,Gayo2018}.
The two main RDF data shape formalisms are SHACL and ShEx.
Both describe RDF data but differ in focus: ShEx emphasizes graph structure, while SHACL targets constraints.
For common use cases, they are equally expressive and interchangeable~\cite{Gayo2018c}.

Source selection is a crucial challenge in decentralized querying.
Approaches like SPARQL \texttt{SERVICE} clauses, service descriptions, basic statistics on triple counts, and histogram methods have been studied~\cite{hose2012towards, Harth2010}.
Schemas-based indexing~\cite{Stuckenschmidt2004} using ontologies has also been explored for source selection of SPARQL queries. 
However, most source selection methods face the limitation of assuming a small number of data sources~\cite{Harth2010}, leaving their suitability for LTQP uncertain.

\iffalse
LTQP has some difficulties.
The open endlessness of the web is the primary one.
During LTQP, it is considered that the search space of the query engine is a pseudo infinite graph~\sepfootnote{sf:graphDomain} domain , where the query engine can discover data sources within a distance of one HTTP request of the already discovered graph.
\emph{Reachability criteria}~\cite{Hartig2012} tries to alleviate this problem by defining completeness on traversal of links respecting conditions.
A difficulty of the approach is the inability to declare the reachability outside of the internals of the query engine.
The theoretical query language LDQL~\cite{hartigLDQL} and the subweb specifications language (SWSL)~\cite{Bogaerts2021LinkTW} are propositions to create a language to express the reachability.
LDQL proposes to let the user along with its query define the reachability using a formalism based on nested regular expression with a formalism close to SPARQL. 
In contrast, SWSL proposes to let the data provider define the traversal within \emph{subweb} in a way that if the engine trusts the data provider it can choose to use the traversal path proposed.
Another approach, to define completeness is to use the structural assumption of web environments~\cite{Taelman2023}.
``Structural assumptions act as contracts between the data provider and the query engines stipulating that within a certain subdomain of the web, information meeting a specific constraint can be found.``~\cite{tam2024opportunitiesshapebasedoptimizationlink}
In this approach,


Previous, LTQP research as focused on the open web however in more recent years it has focused on web environments with structure.
Witing those web en

Those approaches come with some limitations because the criterion have to be chosen carefully by the users not to prune data sources containing relevant results or oppositely
not pruning enough irrelevant sources.
So it is important to clearly define what we meant by completeness in LTQP because identical absolute measurement doesn't necessarily have the same signification. 
Has proposed by Taelman [](cite:cites Taelman2023) in the case of Linked Data Structured Environment (LDSE) the [type index](https://solid.github.io/type-indexes/),
[WedID](https://www.w3.org/wiki/WebID) and [linked data platform](https://www.w3.org/TR/ldp/)
specifications can be used to define a source selector that greatly diminush the domain of exploration and the query execution time.

\subsection{RDF data shapes}

RDF data shapes have been used primarly in validation and description of data~\cite{Gayo2018a}, communicating data strutures, generating data and driving user interfaces~\cite{Gayo2018}.
RDF shapes have the same role as relational and xml schemas~\cite{Boneva2017}.
The two main formalism are SHACL and ShEx.
Both language share the common goal of describing RDF data, but they have different approaches.
ShEx focus on describing RDF graph structure whereas SHACL focus on describing constraints.
For common use cases they share the same expressiveness~\cite{Gayo2018c} thus they can be used interchangely.
The semantic of ShEx, is sound given that we apply some restriction to the syntaxes namely restricting the negations (mainly locally) and the recursion to avoid costy validation and uncoherent facts~\cite{Boneva2017}.
Shex and SCHACL shapes can be closed or open~\cite{Gayo2018, Gayo2018b}, which can have a large impact on their usage.

Shapes have also been used in the context of querying for instance selectivity estimate~\cite{Abbas2018} and cardinality estimate~\cite{kashif2021}.

caracteristic set

Talk about how shapes can become queries
{:.todo}
\fi
\section{Preliminaries}

\subsection{RDF knowledge graphs and SPARQL queries}

Our work focuses on conjunctive and disjunctive queries over RDF knowledge graphs~(KG) using the SPARQL query language~\cite{w3SPARQLQuery}.
The core components of SPARQL queries and of KGs are respectively the triple patterns and the triples defined at Definition~\ref{def:triplePattern} and ~\ref{def:triple}.

\rt{Can you first introduce triple, and then triple pattern. The reverse direction is a bit odd.}

\begin{definition}[Triple pattern]\label{def:triplePattern}
    A triple pattern $tp = (s_{tp}, p_{tp}, o_{tp})$ is formed by a subject term $s_{tp} \in \mathcal{I} \cup \mathcal{B} \cup \mathcal{V}$, 
    a property path \rt{A triple pattern does not contain property paths. Just use predicate here. (I, B, or V)}  $p_{tp}$ as defined by  \citeauthor{Kostylev2015} (Definition 2) and the SPARQL specification (section 18.2)~\cite{w3SPARQLQuery} 
    and an object term  $o_{tp} \in \mathcal{I} \cup \mathcal{L} \cup \mathcal{V} \cup \mathcal{B}$.
    Where $\mathcal{I}$, $\mathcal{B}$, $\mathcal{L}$, $\mathcal{V}$ are respectively the set of every possible IRI, blank node, literal and variable.
\end{definition}

\rt{What is missing here is an explanation of what a triple pattern does (returns a solution sequence with solution mappings).}
\rt{You'll also at least need to hint to its relationship to BGPs up to broader query processing.}

\begin{definition}[Triple]\label{def:triple}
    An RDF triple $t = (s,p,o)$ is similar to a triple pattern, 
    where $s \in\mathcal{I} \cup \mathcal{B}$,
    $p \in \mathcal{I}$ and $o \in \mathcal{I} \cup \mathcal{L} \cup \mathcal{B}$.
\end{definition}

We also define two access functions to get respectively the subject and object term of a triple pattern $tp$ or a triple $t$ while ignoring literals,
$ S: tp \rightarrow \mathcal{I} \cup \mathcal{B} \cup \mathcal{V}$ and $O: tp \rightarrow \mathcal{I} \cup \mathcal{B} \cup \mathcal{V}$.

\subsection{Reachability Criteria}

Relying on results or the traversal of a large network like the Web is not feasible for defining completeness in LTQP.\rt{For defining completeness, this would definitely be feasible. But for practical use cases, it would not be feasible.}
Thus, to formalize the completeness of queries, a link discrimination formalism has been developed called \emph{Reachability criteria}~\cite{Hartig2012}.\rt{I would just introduce this by saying something along the lines of "following all possible links is not feasible in practise, so different reachability criteria have been introduced to define which subsets of links need to be followed."}
Reachability criteria are boolean functions ($c_i$) restricting the dereferencing of links from the internal data source of the query engine.
They take as parameters an RDF triple $t$ from an internal triple store, a dereferenceable IRI $iri$ from $t$, and the query $B$~\cite{Hartig2012} \rt{Are you sure it's the full query (I don't remember). I thought it was just applied to each triple pattern.}.
If $c_i$ returns $true$, the query engine must dereference $iri$.
More formally
\begin{equation}\label{eq:reachabilityCriteria}
c_i(t, iri, B) \rightarrow \{\mathrm{true}, \mathrm{false}\}
\end{equation}

\subsection{Decentralized Knowledge Graphs}

We define a Decentralized Knowledge Graph (DKG) as a KG $G$ materialized in a network of resources $R$.
A resource $r_i \in R$ contains a KG $g_i \subseteq G$ and is mapped and exposed by an IRI $iri_i$.
The network forms a graph where the resources $r_i$ are the nodes and the $iri_j \in t \in g_i$ are directed edges starting from $r_i$ to $r_j$.
$G$ is formed by the union of all the $g \in r$, such that $G = \bigcup_{i=0}^{n}g_i$ given $n$ resources in the network.
\section{Approach}

\sepfootnotecontent{sf:shapeIndexURL}{
   \ifanonymous
      \url{https://anonymous.4open.science/w/shape-index-specification-707E}
   \else
      \url{https://constraintautomaton.github.io/shape-index-specification/}
   \fi
}

\sepfootnotecontent{sf:recursiveShape}{
    In this paper we ignore "inverse constraints" such as
    inverse triple constraint in ShEx 
    %\url{https://shex.io/shex-primer/index.html\#inverse-properties}
    and SHACL Inverse Paths 
    %\url{https://www.w3.org/TR/shacl\#property-path-inverse} 
    to avoid recurse 
    shape schemas~\cite{Corman2019}.
    We only consider shapes that can be transformed into a single \texttt{SELECT} SPARQL query.
}

\sepfootnotecontent{sf:subwebsep}{
   We assume an implicit conversion between the subweb specification and the formalization of reachability criteria.
   Space constraints prevent us from detailing the conversion.
}

\sepfootnotecontent{sf:ssf_project}{
   The paper \citetitle{delva2023} additional material also proposes a script to convert SCHACL shapes into SPARQL queries.
   \url{https://github.com/MaximeJakubowski/ssf_project}
   \rt{This footnote seems unnecessary, since delva2023 is already cited.}
}

\rt{Introduce your subsections here in a few lines, so the reader knowns what to expect. This should help explain to the reader why the formalization is needed before you introduce your shape index approach.}

\subsection{LTQP Completeness when Pruning}\label{sec:slde}

We propose to focus the completeness of LTQP on results instead of traversal in the context DESP.\rt{Say why}
We formalize our approach as follows.
A query is executed over a DKG $G$ formed by the union of all the $g \in r$ in a network $R$.
The query engine has to build a KG $G^{\prime}$ using a reachability $C^{\prime}$ in its internal data store from the  $g \in r$ by dereferencing resources in $R$ such that
$G^{\prime} \subseteq G$ holds to answer its query.
We are trying to solve an optimization problem where another query engine builds a KG
$G^{\prime\prime} \subseteq G^{\prime}$
potentially smaller than the one produced with a specific traversal completeness policy
by defining a reachability criterion $C^{\prime\prime}$.
We focus on maintaining the same results completeness, so when using $C^{\prime\prime}$ the following equation must hold

\begin{equation}\label{eq:evalQueryStructuralAssumption}
   [\![ Q ]\!]^{G^{\prime\prime}} = [\![ Q ]\!]^{G^{\prime}}
\end{equation}
for any network $R$.
Because the $g \in G^{\prime}$ can only be acquired from the dereferencing of resources $r \in R$, a smaller $G^\prime$ implies that a lesser number of HTTP requests has been performed to answer a query.
Naturally, query execution is faster with a smaller KG instance.\rt{This is not entirely accurate, needs to be rephrased. It mostly is so though. Perf depends on distributions. You could create a smaller KG that is slower to query over than a large KG. For example, when you have many distinct predicates.}
Additionally, HTTP requests can be slow and unpredictable~\cite{hartig2016walking}, making them a significant factor in overall query execution time. 
Thus, the benefit of reducing the number of HTTP requests is twofold.

To define less selective reachabilities to produce $G^{\prime\prime}$, we propose extending the reachability criteria by formalizing a chain of criteria in a concept called \emph{composite reachability criteria}.
In this form, a reachability criterion $cp_i$ is said to \emph{prune} links, and $cd_i$ is said to \emph{discover} links.
A reachability $cp_i$ acts upon all the links that have yet to be dereferenced as well as on the incoming links.\rt{What is the difference? Does it even matter?}
Equation~\ref{eq:cReachabilityCriteria} formalizes a composite reachability criterion $C$.

\begin{equation}\label{eq:cReachabilityCriteria}
   C(t, iri, B) = \bigvee_{cd \in Cd} cd(t, iri, B) \mathrel{\land} \bigwedge_{cp \in Cp} \, cp(t, iri, B)
\end{equation}
where $Cd$ is the set of every $cd_i(t, iri, B)$ and $Cp$ the set of every $cp_i(t, iri, B)$ used by the engine.
This section has introduced our general concept of pruning. 
In the next section, we present our specific shape-based approach using a shape index.

\subsection{Shape Index}

Pruning in LTQP requires information about the data models of dereferenced sources.
However, obtaining complete, up-to-date, and expensive information for each source in a large decentralized network is unrealistic.\rt{But yet, it is what you do with shape indexes, so it is realistic then? Probably needs some rephrasing.}
To address this, we propose the concept of the \emph{shape index}. 
We define a shape index is a mapping between sets of RDF documents and RDF data shapes, which describe subweb controlled by a data provider.
When a set of RDF documents is mapped, the associated KG must conform to the shape.
Unlike statistics on triples, shapes are independent of the size of the KG or updates that remain compliant with the shape, making them a more cost-effective alternative for use cases where the data model remains stable. 
Moreover, shapes support open-world assumptions, offering data providers flexibility at the cost of being unsuitable for pruning (if we do not consider negative statement).\rt{I don't understand this sentence.}

We formalize a shape index as follows:
\begin{equation}\label{eq:shapeIndex}
   SI = \{s_1 \mapsto IRI_1, s_2 \mapsto IRI_2 \cdots, s_n \mapsto IRI_n\}
\end{equation}
\rt{You must define what $s_i$ and $IRI_i$ are. (Probably iri instead of IRI to be aligned with the earlier formalisms?)}
given $n$ entries.
The subweb described by the index is defined by $D = \bigcup_{i=1}^{n} \bigcup_{iri \in IRI_i} iri$.
A shape index \emph{must} map every resource in $D$.
We denote a shape index as \emph{complete} when every shape $s_i \in \text{dom}(SI)$ is closed and \emph{incomplete} otherwise.\rt{When is a shape closed? (may also be part of preliminaries if that is easier)}
A mapping between a shape and a set of IRIs has implications in the distribution of the data in $D$.
When a shape $s$ is mapped to an $IRI$, then the KG targeted by the mapping $G = \{g | g \in r, iri \mapsto r \land iri \in IRI\}$ satisfies $s$.
Given that the shape is closed, then every set of triples in the resource associated with $D$ respecting the shape must be in a resource mapped to an $iri \in IRI$.
We provide a detailed technical description of the shape index in an online specification~\sepfootnote{sf:shapeIndexURL}.

\rt{This feels like a good place to start a new subsection...}

RDF shapes use the concept of \emph{targets} to determine the root node for validation.
In this work, we consider all entities in a KG within a document to conform to the same schema.
\rt{A link is missing here, which says that you use the root of star patterns as validation roots (?)}
We refer to these entities as tree stars (patterns) \rt{This is confusing. Introduce tree stars and tree star patterns as separate concepts}, an extension of the existing star patterns concept in RDF \rt{citation needed}.
Star patterns consist of triples sharing the same subject.
We extend this idea by considering all star patterns linked via the object term of a preceding star pattern, forming a tree-like structure of star patterns.
For example, consider a user that links to his posts with recursive replies.
Tree stars can capture this through a star pattern defining a user, and recursively nested star patterns representing posts.
This concept, formalized in Definition~\ref{def:starPattern}, serves two purposes: defining targets for validation and capturing relationships between the triple patterns of our query and shape entities.\rt{Aha, here it is! This should have come as first sentence :-) Always explain why something is needed before you explain it.}

\begin{definition}[Tree Star Pattern]\label{def:starPattern}
   We define a star pattern $Q_{star}$ as a set of $tp \in Q$~\cite{Karim2020} with the same subject such that 
   given a builder function 
   \begin{equation}
       BQ_{star}(s) = \{tp_i \in Q| S(tp_i) = s\}
   \end{equation}
   with $s \in \mathcal{I} \cup \mathcal{B} \cup \mathcal{V}$ then $Q_{star_s} = BQ_{star}(s)$.
   We define a tree star pattern $Q_{starT}$ as the union between a root star pattern $Q_{star_s}$
   and the star patterns having as subject term an object term of another star pattern in $Q_{starT}$.
   We define a function 
   $O_{star}: q \in Q \rightarrow  \mathcal{I} \cup \mathcal{B} \cup \mathcal{V}$
   that returns every non-literal object terms of a star pattern.

   We then define $Q_{starT}$ given a  set of partial tree star patterns $Q_{starPT}$
   \begin{equation}
       Q_{starT} = \bigcup_{q \in Q_{starPT}} q
   \end{equation}
   where $Q_{starPT}$ is formed with a root $Q_{star_s}$ by

   \begin{equation}
           Q_{starPT_i} =
       \begin{cases}
           \{Q_{star_s}\} & \text{if } i = 1 \\
           \{BQ_{star}(o)| o \in \bigcup\limits_{q \in Q_{starPT_{i-1}}} O_{star}(q)\} & \text{if } i>1
       \end{cases}
   \end{equation}

   We also define a function  
   $S_{star}: q \rightarrow  \mathcal{I} \cup \mathcal{B} \cup \mathcal{V}$
   returning the subjects of the $Q_{starPT_i}$ of a $Q_{starT}$

   We propose a similar definition for the context of KGs where we replace the query $Q$ by a KG $G$ and where the terms 
   cannot be a $v \in \mathcal{V}$. 
   We denote this structure a tree star.
\end{definition}
Thus, the target shapes in the shape index correspond to the subject of each root star pattern when a KG is divided into tree stars with no shared partial tree stars.\rt{This should also have come much sooner.}

\subsection{Link Pruning Using Shape Indexes}\label{sec:sourceSelection}

In this section we make the link between the shape index and link prunning \rt{Make sure to go through the whole paper with a spell-checker. I've seen you make this typo regularly.} in LTQP.
A decentralized dataset \rt{DKG?} exposing a shape index provides a query engine with the opportunity to reduce its search domain by knowing which resources are query-relevant.
More formally, instead of traversing the whole subweb $D$ associated with a shape index, the engine will traverse a subweb $d \subseteq D$, ignoring the knowable query-irrelevant sources.
The concept of composite reachability criteria allows us to ignore certain sources during traversal based on the knowledge acquired during traversal.
Our approach involves dynamically constructing new reachability criteria during traversal by adding pruning critera as we discover and analyze shape indexes.
These criteria are designed so that, they will always produce the same completeness of results as the one that was defined at the beginning of the traversal.\rt{I don't understand this. Do you mean that the order of finding shape indexes does not matter?}

\rt{I still think introducing time here is a really bad idea. I'm almost certain that reviewers will shoot the paper down because of this. In the formalization, let's just assume prior knowledge of all shape indexes and corresponding reachability criteria. Combining them during query execution is an implementation detail.}
More formally, let us introduce time $t$ as a factor for our reachability criteria $C_t$.
The query execution begins with an initial reachability criterion $C_0$.
At any time $t$, Equation~\ref{eq:evalQueryStructuralAssumption} must hold if we consider $C_t = C_{t+1} = C_{t+2} \dots = C_{tf}$ until the end of the execution $tf$, given that $G^{\prime}$ is produced using $C_0$.
The initial criterion $C_0$ must include a discovery reachability criterion $cd_{\text{shape index}}$ that leads to a shape index document.
After dereferencing a shape index $SI_i$ at time $tsi_i$, the query engine creates a set of links $IRI_p$, which contains the links that must be pruned.
The links to prune are determined by evaluating the shape index to identify the set of IRIs that are not query-relevant.
This is achieved by performing a query-shape containment check ($\sqsubseteq_{qs}$), as defined in Section~\ref{sec:containment}.
We define
$IRI_p = \{ \bigcup_{iri \in SI_i(s_j)} iri | B \sqsubseteq_{qs}  s_j = \mathrm{false}, s_j \in \text{dom}(SI_i) \}$.
From this sets of links we define a pruning reachability criteria;
\begin{equation}
       cp_{si}(t, iri, B) = iri \notin IRI_p
\end{equation}
The new reachability $C_{tsi_i + 1}$ is created by taking the $Cd$ and $Cp$ of $C_{ts_i}$ and adding
and $cp_{si}$ to $Cp$.


\subsection{Query Shape Containment}\label{sec:containment}

We consider determining if a source is query-relevant a \emph{query-shape containment} problem.
The understanding of this problem is similar to the classic query containment problem \rt{Citation needed}.
Query containment determines if a query can be answered from the answer of another query, independent to the database or KG.
In our work, we want to determine if a tree star pattern from a query can be answered by any sources respecting a specific shape.
\rt{There is an unnatural jump here in your storyline. Needs another sentence, or separate paragraph?}
Intuitively shapes expressions can be treated as a segment of a query~\cite{delva2023} \rt{What is a segment of a query? Do you mean that shapes can be translated into queries?}.
Furthermore it is a common approach for shape validation over an RDF graph is to convert shapes into SPARQL queries~\cite{labragayo2017validatingdescribinglinkeddata, Corman2019,Prestamo2023, spapeExpressionConvert}.~\sepfootnote{sf:recursiveShape}
We refer to this transformation of a shape $s$ as $T(s)$ producing a query $q_s$.
\rt{Another jump?}
We consider open shapes to always represent queries that retrieve entire KGs (ignoring any negative statements). \rt{You'll need to elaborate on these negative statements} 
This is because, under the open-world assumption, a shape defines the minimum constraints of a graph.
With this transformation the problem \rt{Unclear what \emph{the} problem is here.} becomes similar to query containment problems.
\rt{Another jump?}
Query containment may intuitively seem unsuitable for dynamic reachability due to its NP-complete time complexity~\cite{Spasi2023}.
However, this complexity depends on the size of the queries, which are typically small in practise~\cite{Doan2012}; we do not expect in most cases queries containing thousands or millions of triple patterns~\cite{Bonifati2019}.
Additionally, practical use cases can often leverage polynomial-time algorithms~\cite{Doan2012}.
In our context, the container query \rt{This concept is unknown to the reader} (the shape) adheres to a tree-star pattern template structure, where predicates are always IRIs.
This structure arises because shapes predominantly describe predicate terms and object terms and are isomorphic to the specific KG.
By exploiting this structure, it is possible to design an algorithm with polynomial time complexity.

We consider that a query $Q$ is contained in a shape $S$, denoted as $Q \sqsubseteq_{qs} S$, if a tree star pattern in $Q$ is contained in $Q_s = T(S)$. 
For a tree star pattern to be contained, we need to consider the two parts of $Q = Q_{\text{body}} \cup Q_{\text{unions}}$.\rt{What if bgps and unions are nested? Limitation of your work? (which is fine, but must be mentioned)}
$Q_{\text{body}}$ is the Basic Graph Pattern (BGP) of the query, and $Q_{\text{unions}} = \bigcup Q_u$ represents the Union Graph Patterns (UGP), where $Q_u = q_0 \cup q_1 \cup q_2 \dots \cup q_n$.
A tree star pattern $Q_{\text{starT}_i}$ is contained in $Q_s$ if its segment in $Q_{\text{body}}$ is contained in $Q_s$, and if its segment in at least one $q_i$ in each $Q_u$ is contained in $Q_s$. 
If $Q_{\text{starT}_i}$ is not part of any $Q_u$, then $Q_u$ is ignored.
\rt{I would not mention everything after this, and instead at the start of this paragraph say that for simplicity of this formalization, we focus on just BGPs of triple patterns and unions.}
In our containment problem, we ignore \texttt{GROUP BY} segments.
We make this choice because, in the context of shape queries, \texttt{GROUP BY} is primarily used to set a cardinality, and we are not attempting to identify sources that can fully answer specific segments of a query. 
Instead, our goal is to disregard data sources that are irrelevant to the query, given the constraints imposed by the shape index. 
Discriminating based on cardinalities could potentially affect query results.
Additionally, this article does not consider filter expressions, as they can significantly increase the complexity of the problem.
Moreover, "false" negatives do not impact the correctness of our approach. \rt{But this is important to mention indeed! But needs some more elaboration on the fact that doing more (useless) requests is not a problem.}
However, incorporating filter expressions into future work would be an interesting direction to explore.

% https://en.wikipedia.org/wiki/Master_theorem_(analysis_of_algorithms)
\begin{algorithm}[h]
   \caption{Determine if a tree star pattern is contained ($isContain_{T}$)}\label{alg:containmentTree}\rt{I think I also mentioned this in my previous review, but you probably mean isContained? Also not sure what the T in there means.}
   \begin{algorithmic}
      \REQUIRE  $Q_{star}$, $Q_{starT_i}$, $Q_s = Q_{s\text{body}} \cup Q_{s\text{unions}}$ and $Eval_{star}$
      \ENSURE \TRUE $ $ or \FALSE $ $ whether the tree star \rt{pattern?} is contained in the shape

      \IF{$S_{star}(Q_{star}) \in Eval_{star}$}
         \RETURN \TRUE
      \ENDIF 

      \FORALL{$tp \in Q_{star}$} % O(n_star)
         \IF{\NOT $match(tp, Q_{s\text{body}})$}
            \STATE $hasOnePath \gets $ \FALSE
            \FORALL{$q_{us} \in Q_{s\text{unions}}$}
               \IF{$isContain_{T}(Q_{star}, Q_{starT_i}, q_{us}, Eval_{star})$}
                  \STATE $hasOnePath \gets $ \TRUE
               \ENDIF
            \ENDFOR
            \IF{\NOT $hasOnePath$}
               \RETURN \FALSE
            \ENDIF
         \ELSE
            %eval nested
            \IF{$O(tp) \in  S_{star}(Q_{starT_i})$}
               \IF{\NOT $isContain_{T}(Q_{star_{O(tp)}} \in Q_{starT_i}, Q_{starT_i}, Q_s, Eval_{star})$}
                  \RETURN \FALSE
               \ENDIF
            \ENDIF
            %
         \ENDIF
      \ENDFOR

      \STATE $Eval_{star} \gets Eval_{star} \cup S_{star}(Q_{star})$
      \RETURN \TRUE
   \end{algorithmic}
\end{algorithm}


We define the function $isContain_{T}$ in Algorithm~\ref{alg:containmentTree} to evaluate whether a tree star \rt{pattern?} with a root star pattern $Q_{star_i}$ from $Q_{starT_i}$ is contained in $Q_s$. 
The algorithm also takes a set $Eval_{star}$ to track which partial tree stars have already been evaluated.
The algorithm works by examining each triple pattern in the root star pattern $Q_{star_i}$ and checking if an equivalent triple pattern (ignoring the variable names) can be found in the BGP of $Q_s$ using the $match$ function.
If the triple pattern cannot be found in the BGP, the algorithm then looks into the UGPs of $Q_s$. 
Since we assume that the union statements are not nested (perhaps \rt{This \emph{perhaps} is not very scientific...} through rewriting. Either say it's a limitation, or cite a paper showing you can rewrite. (or prove yourself)), this limits the number of recursive calls.
If an equivalent triple pattern is found, the algorithm checks whether the object of the triple pattern is the subject of a partial tree star pattern in $Q_{starT_i}$.
If it is, the algorithm recursively applies the same procedure to this partial tree star pattern as $Q_{star_i}$.
To avoid cycles and redundant evaluations when processing the object of the star patterns, we maintain a set of evaluated subjects in $Eval_{star}$.
We notice that the complexity of Algorithm~\ref{alg:containmentTree} is $O( n_o \times n_{tp}^2 \times n_{sunion}^2)$
where $n_{tp}$ is the number of triple patterns of $Q_{star_i}$, $n_{sunion}$ the number of UGP in $Q_s$ and $n_o$ the number of object term in $Q_{starT_i}$.
To solve $Q \sqsubseteq_{qs} S$, we need to consider the number of tree stars from the BGP with their number of segments in the UGP and the number of BGPs in the UGPs.
This operation results again in a polynomial time complexity algorithm.
%To solve $Q \sqsubseteq_{qs} S$, we need to consider the $n_{starT}$ tree star from the BGP with their $n_{starTu}$ segment in the UGP and $n_{starTui}$ BGP in the UGP.
%This operation results in a polynomial time complexity of $O(n_o \times n_{tp}^2 \times n_{union}^2 \times n_{starT} \times n_{starTu} \times n_{starTui})$.
In the \nameref{sec:appendix} Algorithm~\ref{alg:containment} present the full resolution.\rt{What do you mean by "the full resolution"? Is it a different algorithm?}

\section{Experiment}

\sepfootnotecontent{sf:implementationComunica}{
\url{https://github.com/constraintAutomaton/comunica-feature-link-traversal/tree/feature/shapeIndex}
}

\sepfootnotecontent{sf:implementationQueryShapeContainment}{
   \url{https://github.com/constraintAutomaton/query-shape-detection}
}

\sepfootnotecontent{sf:shapeIndexGenerator}{
   \url{https://github.com/constraintAutomaton/rdf-dataset-fragmenter.js/tree/feature/shape-index-fragmentation-strategy}
}

\subsection{Implementation}
We implemented our approach using the LTQP version of the Comunica query engine~\cite{taelman_iswc_resources_comunica_2018}. 
We chose Comunica for its modularity~\cite{taelman_swj_componentsjs_2022} and its established use in multiple LTQP studies, 
including our prior experiment with the shape index concept~\cite{Bogaerts2021LinkTW, Taelman2023, eschauzier_quweda_linkqueue_2023, Hanski2024, eschauzier_amw_rcubemetric_2024, tam2024opportunitiesshapebasedoptimizationlink}.
Our implementation of our reachability criteria and the link queue filter is open source~\sepfootnote{sf:implementationComunica}, as well as our query-shape containment solver.~\sepfootnote{sf:implementationQueryShapeContainment}

\subsection{Experimental setup}
We use the social benchmark Solidbench~\cite{Taelman2023} which is based on the LDBC benchmark~\cite{Angles2020} to evaluate our contribution.
We created an open source module~\sepfootnote{sf:shapeIndexGenerator} to generate shape indexes in Solidbench based on user-provided mapping between ShEx shapes and data model objects.
Additionally, our module offers the functionality to generate incomplete shape indexes reproducibly using user-defined probability and a random seed.

We design the following experiments to analyze and evaluate our approach:

\begin{itemize}
   \item Evaluation of the execution time and results of our query-shape containment algorithm with the shapes used in the study.
   \item Evaluation of the state-of-the-art traversal algorithm, using the type-index, the LDP specification and type-index, and the LDP specification~\cite{Taelman2023} in relation to our shape index approach in a network where every dataset provides a complete shape index with the most descriptive shapes.
   \item Evaluation of a network where each dataset exposes a shape index where each data model entries are described with a shape. 
   There are three experiments: one in which the shapes are the most descriptive, one in which the shapes do not refer to other shapes outside of what is hosted by the dataset, and one in which the shapes provide a minimal description of the data mode.
   \item Evaluation of a network where 0\%, 20\%, 50\%, and 80\% of the shape index entries are present in the datasets, meaning that some shape indexes are complete, incomplete, or missing.
   \item Evaluation of the sources needed to evaluate each query independently of the network topology and evaluation of the performance of our engine using a \texttt{CNone}~\cite{Hartig2012}(no links are followed) reachability given that the link queue has been injected with the source the required sources.
 \end{itemize}

To run our experiments we used a machine with the Os Ubuntu 20.04.6 LTS with cpu 2x Hexacore Intel E5645 (2.4GHz) CPU and 24GB RAM.

PROVIDE LINK FOR EXPERIMENTS AND SHAPES
\iffalse
102x pcgen3 nodes
https://doc.ilabt.imec.be/ilabt/virtualwall/hardware.html#virtual-wall-2
\fi

\section{Results}

The empirical evaluation of the query-containment algorithm shows its execution with the shapes of our experiment is negligible, as presented in table~\ref{tab:queryShapeContainmentEval}.
The execution time for the other shape set is also negligible. Because of space restrictions, we do not include those results, but they are available online.
\input{table/table_query_shape_containment_exec}


\section{Conclusion}\label{sec:conclusion}

In this article, we introduced pruning mechanism in LTQP by extending the concept of reachability criteria and using shape indexes.
Using the SolidBench benchmark, we demonstrate that our approach achieves a significant reduction in the number of HTTP requests without degrading performance, provided that the query exploits the structural properties of the dataset.
In the best-case scenario, our approach improved query execution time by 7 times and enabled a query that previously could not complete within the time limit to finish successfully.
On average, performance improved by 1.76 times, while in the worst case, execution time increased by 2.8 times.
Our method effectively handles scenarios with reduced shape indexes information in networks.
These findings are particularly relevant for decentralization efforts enabling third-party clients to efficiently query large, heterogeneous networks.
Our approach imposes minimal overhead on servers, relying solely on them to serve small static shape index files.
Future work could focus on enhancing query planning in LTQP~\cite{taelman2024towards} with RDF data shapes, developing and maintaining shape indexes, and exploring the impact of shape-based LTQP optimization on result arrival times~\cite{Acosta2017}.

%\rt{I still have some open questions after reading the full article: What is the impact on server load (CPU usage)? Does adding shapes lead to an increase of that? Because if not, you could conclude here by saying that adding shapes adds significant benefits to the client, with no overhead to the server, except for a (possibly offline) process for shape creation/derivation. Also, does adding shapes influence query result arrival times negatively or positively?}

\paragraph*{Supplemental Material Statement:}\label{sec:supplementalMaterial} All source code, benchmark queries, datasets, raw results, and supplementary analyses are available online.~\footnote{
    \ifanonymous
       \url{https://anonymous.4open.science/r/documentation-1A65}
    \else
       \url{https://github.com/shapeIndexComunicaExperiment/documentation}
    \fi 
    \label{sf:supplementalMaterial}}


%DO NOT FORGET THE ACKNOWLEDGMENTS!!!

\printbibliography
\addcontentsline{toc}{section}{References}

\section*{Appendix}


\input{analysis/artefact/query_containment_execution_time/fully_bounded/table_query_shape_containment_exec}

\input{analysis/artefact/ratio_useful_resources/table_ratio_useful_ressources}

\begin{algorithm}
    \caption{Determine if a query is contained in a shape ($isContain$)}\label{alg:containment}
    \begin{algorithmic}
       \REQUIRE  $Q = Q_{\text{body}} \cup Q_{\text{unions}}$, $Q_s$
       \ENSURE \TRUE\ or \FALSE\ whether the tree star is contained in the shape
       
       \STATE divide $Q_{\text{body}}$ into $Q_{starT}$
       \STATE $mainBodyContain \gets \FALSE$
 
       \FORALL{$Q_{starT_i} \in Q_{star}$}
          \FORALL{$q \in Q_{starT_i}$}
             \STATE $mainBodyContain \gets mainBodyContain \lor isContain_{T}(q, Q_{\text{starT}_i}, Q_s, \{\})$
          \ENDFOR
       \ENDFOR
 
       \IF{$mainBodyContain$}
          \FORALL{$Q_u \in  Q_{\text{unions}}$}
             \IF{\NOT $isContain(Q_u, Q_s)$}
                \RETURN \FALSE
             \ENDIF
          \ENDFOR
       \ELSE
          \RETURN \FALSE
       \ENDIF
       \RETURN \TRUE
    \end{algorithmic}
 \end{algorithm}
\end{document}
