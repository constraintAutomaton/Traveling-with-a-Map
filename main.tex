
\documentclass[sigconf, anonymous, review, natbib=false]{acmart}

%%
%% \BibTeX command to typeset BibTeX logo in the docs
\AtBeginDocument{%
  \providecommand\BibTeX{{%
    Bib\TeX}}}

%% Rights management information.  This information is sent to you
%% when you complete the rights form.  These commands have SAMPLE
%% values in them; it is your responsibility as an author to replace
%% the commands and values with those provided to you when you
%% complete the rights form.
\setcopyright{acmlicensed}
\copyrightyear{2024}



%%
%% Submission ID.
%% Use this when submitting an article to a sponsored event. You'll
%% receive a unique submission ID from the organizers
%% of the event, and this ID should be used as the parameter to this command.
%%\acmSubmissionID{123-A56-BU3}

%%
%% For managing citations, it is recommended to use bibliography
%% files in BibTeX format.
%%
%% You can then either use BibTeX with the ACM-Reference-Format style,
%% or BibLaTeX with the acmnumeric or acmauthoryear sytles, that include
%% support for advanced citation of software artefact from the
%% biblatex-software package, also separately available on CTAN.
%%
%% Look at the sample-*-biblatex.tex files for templates showcasing
%% the biblatex styles.
%%


%%
%% The majority of ACM publications use numbered citations and
%% references, obtained by selecting the acmnumeric BibLaTeX style.
%% The acmauthoryear BibLaTeX style switches to the "author year" style.
%%
%% If you are preparing content for an event
%% sponsored by ACM SIGGRAPH, you must use the acmauthoryear style of
%% citations and references.
%%
%% Bibliography style
\RequirePackage[
  datamodel=acmdatamodel,
  style=acmnumeric,
  ]{biblatex}

%% Declare bibliography sources (one \addbibresource command per source)
\addbibresource{references.bib}

%%

\usepackage{listings}
%% auto break lines
\lstset{breaklines=true}

\usepackage{svg}
\usepackage{graphicx}
\usepackage{sepfootnotes}

\begin{document}


\title{Optimizing link traversal queries with Shape Indexes: A Simple Data Summary Approach for unindexed linked data networks}

\author{Bryan-Elliott Tam}
\email{bryanelliott.tam@ugent.com}
\affiliation{%
  \institution{Gent Universiteit}
  \city{Ghent}
  \country{Belgium}
}

\author{Ruben Taelman}
\email{ruben.taelman@ugent.be}
\affiliation{%
  \institution{Gent Universiteit}
  \city{Ghent}
  \country{Belgium}
}
\author{Pieter Colpaert}
\email{pieter.colpaert@ugent.be}
\affiliation{%
  \institution{Gent Universiteit}
  \city{Ghent}
  \country{Belgium}
}

\begin{abstract}
% Context  
The centralization of web information raises legal and ethical concerns, particularly in social, healthcare and education applications.
% Need  
Decentralized architectures, offer a promising alternative, but efficient query processing remains a challenge.  
Link Traversal Query Processing (LTQP) enables querying across decentralized networks but suffers from long execution times and high data transfer due to excessive HTTP requests.  
% Task  
We propose a shape-based pruning approach that relies on \emph{shape indexes} and a \emph{query-shape subsumption} algorithm to reduce the search space and consequently, the number of HTTP requests.
% Object  
We formalize this method as a link pruning mechanism for LTQP and evaluate its impact on social media queries using the SolidBench benchmark.  
% Findings  
Our results show that shape-based pruning improves query execution time and reduces network usage up to 7 times compared to the state of the art, in exchange of a minor increase in the number of triples per shape-index instance.
% Conclusion  
This work demonstrates the potential of shape-based metadata for optimizing LTQP queries in decentralized knowledge graphs, going beyond its traditional use on data validation.

\keywords{Linked Data,
Link Traversal Query Processing,
RDF data shapes,
Decentralization,
Data summarization
}

\end{abstract}


%%
%% Keywords. The author(s) should pick words that accurately describe
%% the work being presented. Separate the keywords with commas.
\keywords{Linked data,
Link Traversal Query Processing,
Query containment,
RDF data shapes,
Data summarisation,
Descentralized environments}

\maketitle

\section{Reaseach questions}
In the follow section we are going to formulate our reasearch questions.

\begin{itemize}
    \item Can our method reduce the ratio of non-contributing data source dereferenced?
    \item How does the diminution of HTTP request affect the query execution time?
    
    \item How does the level of detail of the shapes impact the performances?
    \item What is the difference in performance between a\emph{complete} and an \emph{incomplete} shape index?
    \item How does the ratio of subdomains containing an index influence global performances?
    \item How does the fragmentation of the subdomain impact the performance?
    
    \item How does the quantity of non-query contributing resources impact the performances?

    \item What is the ideal query execution time if we only dereferenced contributing data sources?
    \item What is the performance of our method compared to a single endpoint solution?
\end{itemize}

\section{Related Work}

LTQP is a SPARQL querying paradigm that answers queries by exploring the web using the follow-your-nose principle~\cite{hartig2016walking}.
The main challenge of LTQP is the web's open-ended nature leading to large search space.
Completeness in LTQP is defined by the traversal of a well-defined set of links~\cite{Hartig2012}.
The first method employed to define this set was the reachability criteria~\cite{Hartig2012}, which are boolean functions that determine whether a link should be dereferenced.
Building on this, the theoretical query language LDQL~\cite{hartigLDQL} was introduced, separating the traversal definition from the query definition.
Further advancements include the subweb specifications language (SWSL)~\cite{Bogaerts2021LinkTW}, which allows data providers to define how their DKG should be traversed.
These contributions are centered on guiding the engine in selecting links to follow (discovery process).
However, they do not explicitly address mechanisms for restricting certain links (pruning process).
Link pruning can be very useful to reduce the search domain of queries when information about the data model of DKG is found.

RDF data shapes are used for validating, describing, and communicating data structures, as well as generating data and driving user interfaces~\cite{Gayo2018a,Gayo2018}.
The two main RDF data shape formalisms are SHACL and ShEx.
Both describe RDF data but differ in focus: ShEx emphasizes graph structure, while SHACL targets constraints.
For common use cases, they are equally expressive and interchangeable~\cite{Gayo2018c}.
Automatic generation of RDF data shapes based on knowledge graph (KG)~\cite{fernandez2023extracting} and shape-based integration of data~\cite{LabraGayo2023} are also topics that have been studied and can support shape-based summary approaches.
Shape Trees~\cite{shapetreesShapeTrees} is an index structure for validating and organizing decentralized knowledge graphs (DKG).
However, Shape Trees have not been used for query optimization, and the way they are formalized makes this use difficult.
RDF data shapes have also been used in the litterature for querying of centralized KG~\cite{kashif2021}.

Source selection is a crucial challenge in decentralized querying.
Approaches like SPARQL \texttt{SERVICE} clauses, service descriptions, basic statistics on triple counts, and histogram methods have been studied~\cite{hose2012towards, Harth2010}.
However, most of those source selection methods face the limitation of assuming a small number of data sources~\cite{Harth2010}, leaving their suitability for LTQP uncertain.
Bloom filters~\cite{dia2018fast} is also a mecanism that has with success for federated DKG, however in the context of LTQP it has been show that bloom filters have little effect on performances~\cite{Hanski2024}
Schemas-based indexing~\cite{Stuckenschmidt2004} using ontologies has also been explored for source selection of SPARQL queries,
however the proposed approach is sensible to the high reuse of vocabulary terms in RDF~\cite{Harth2010}.
\section{Preliminaries}\label{sec:preliminaries}

\subsection{RDF knowledge graphs and SPARQL queries}

Our work focuses on the union of conjunctive queries over RDF knowledge graphs~(KG) using the SPARQL query language~\cite{w3SPARQLQuery}.
The core components of KGs and SPARQL queries are respectively the triples and the triple patterns as deined in Definition~\ref{def:triple} and ~\ref{def:triplePattern}.

\begin{definition}[Triple]\label{def:triple}
    RDF triple $t = (s,p,o)$ are tuples formed with three terms. A subject where $s \in\mathcal{I} \cup \mathcal{B}$, a predicate $p \in \mathcal{I}$ and an object $o \in \mathcal{I} \cup \mathcal{L} \cup \mathcal{B}$.
    Where $\mathcal{I}$, $\mathcal{B}$, $\mathcal{L}$,  are respectively the set of every possible IRI, blank node, literal.
\end{definition}

\begin{definition}[Triple pattern]\label{def:triplePattern}
    Triple patterns $tp = (s_{tp}, p_{tp}, o_{tp})$ are similar to triples, where $s_{tp} \in \mathcal{I} \cup \mathcal{B} \cup \mathcal{V}$,
    $p_{tp} \in \mathcal{I} \cup \mathcal{V}$ and an object term  $o_{tp} \in \mathcal{I} \cup \mathcal{L} \cup \mathcal{V} \cup \mathcal{B}$.
    Where $\mathcal{V}$ is the set of every possible variable.
    %A triple pattern returns a solution sequence with solution mappings a single or multiple $tp$ forms a Basic Graph Pattern (BGP).
\end{definition}

We also define two access functions to get respectively the subject and object term of a triple pattern or a triple while ignoring literals,
$ S: tp \rightarrow \mathcal{I} \cup \mathcal{B} \cup \mathcal{V}$ and $O: tp \rightarrow \mathcal{I} \cup \mathcal{B} \cup \mathcal{V}$.
We denote $[\![ Q ]\!]^{G}$ as the evaluation of a query $Q$ over a KG $G$~\cite{Angles2008}.

\subsection{Reachability Criteria}

LTQP defines completeness on the traversal of links instead of the query results~\cite{Hartig2012}.
To formalize the completeness of queries, a link discrimination formalism has been developed called \emph{Reachability criteria}~\cite{Hartig2012}.
Reachability criteria are boolean functions ($c_i$) restricting the dereferencing of links from the internal data source of the query engine.
They take as parameters an RDF triple $t$ from an internal triple store, a dereferenceable IRI $iri$ from $t$, and a union of conjective queries $Q$~\cite{Hartig2012}.
If $c_i$ returns $true$, the query engine must dereference $iri$.
More formally
\begin{equation}\label{eq:reachabilityCriteria}
c_i(t, iri, Q) \rightarrow \{\mathrm{true}, \mathrm{false}\}
\end{equation}

\subsection{Decentralized Knowledge Graphs and subweb}\label{sec:dkg}

We define a DKG as a KG $G$ materialized in a network of resources $R$.
A resource $r_i \in R$ is map to a KG $g_i \subseteq G$, which is a set of triples~\cite{w3ConceptsAbstract}.
We denote this mapping $r_i \mapsto_{g} g_i$.
A resource is mapped and exposed by an IRI $iri_i$ denoted by $iri_i \mapsto_{iri} r_i$.
The network forms a graph where the resources $r_i$ are the nodes and the $iri_j \in g_i$ are directed edges starting from $r_i$ to $r_j$.
$G$ is formed by the union of all the $g \in \text{dom}(R)$.%, such that $G = \bigcup_{i=1}^{n}g_i$ given $n$ resources in the network.
A subweb is a (sub)DKG defined by a set of IRIs controlled by a data provider, which can offer a structure that aids query engine optimization.

\printbibliography

\end{document}
\endinput

%%
%% End of file
