\section{Approach}

\sepfootnotecontent{sf:shapeTarget}{
   It has to be noted that the concept of target and map of SHACL and ShEx does not apply, since it is always the whole knowledge graph that has to be valid.
}

\sepfootnotecontent{sf:shapeIndexURL}{
   \url{https://constraintautomaton.github.io/shape-index-specification/}
}

\subsection{Structured linked data environments}
The web from a holistic perspective does not have a structure exploitable by the engine for query optimization.
On the web, any document can be published at any location with no trust mechanism to guide source selection.
From the perspective of small web subsection data publishers can implicitly or explicitly organize their data publication in a way that a query engine could exploit 
its organization for query optimization.
With that intuition \citeauthor{Taelman2023} propose the notion of \emph{structural assumptions}~\cite{Taelman2023} formalized using the subweb specifications(SWS)~\cite{Bogaerts2021LinkTW}
to guide query engines towards relevant data source given that the query engine decides to trust the data from publishers following selected web specifications.
They have shown that using this web model the execution time of queries targetting principally data interacting with the structural assumptions it is possible to 
reduce query execution time to the extent where query planning becomes the bottleneck of some queries as opposed to the execution of HTTP requests like other studies proposes~\cite{Taelman2023, hartig2016walking}.
We propose to extend the formalization of \citeauthor{Taelman2023} by formaly define a \emph{structured data vaults} $SDV$ , a new more abstract definition of \emph{structural assumptions} $sa$ and result completeness 
using the SWS as a foundation.

A structured data vaults is in essence a set of structured assumptions that are affecting a certain domain of IRI of the world wide web.
This concept shares some similarities with specification-annotated world~\cite{Bogaerts2021LinkTW} however its data selectors (Structural assumptions in our model) can target a subset of
its Web of Linked Data~\cite{Bogaerts2021LinkTW}.
It is defines as follow:

\begin{definition}[Structured data vaults]\label{def:structuredDataVaults}
A structured data vaults is defined by the tuple~\newline $SDV = (W_{SDV}, SA = \{\sigma_{sa_1}(W_1, Q), \sigma_{sa_2}(W_2, Q)...\sigma_{s_n}(W_n, Q)\})$ given 
$n$ structural assumption $\sigma_{sa_i}$ where $W_i  \subseteq W_{SDV}$ and $W_{SDV} = \bigcup_{i=0}^{n}W_i$.
\end{definition}

A Structural assumption share similarities to a source selector~\cite{Bogaerts2021LinkTW}.
Structural assumption differs from source selectors as they take a query from the user as a parameter
effectively giving the ability to pushdown queries to the level of source selection.
Additionally, it has the ability to indicate links that surely will not contribute to the query execution hence 
not functioning as a link discovery mechanism but also has link pruning.
We define a structural assumption as follows:

(WE NEED TO DEFINE Web of link data)
\begin{definition}[Structural assumption]\label{def:structuralAssumption}
A Structural assumption is a function $\sigma_{sa}(W,Q) \rightarrow (I_{sa}\in I_D, I_{nsa} \in I_D)$ that considering a Web of Linked Data~\cite{Bogaerts2021LinkTW} $W \subseteq W_{SDV}$ 
where IRI $i_i \in I_D$ are mapped to resources $d_i \in D$ and a query $Q$ returns a set of IRI $I_{sa}$ respecting the a segment of the structure of publication of 
its associated $SDV$ and a set of IRI $I_{nsa}$ denoting IRI not contributing to the query result considering withing the $SDV$.
\end{definition}

From those defintions and we define a results based completeness notion to better evaluate the performance of engine following different reachability criteria.
Reachability criteria already defines completeness but it is from the perspective of the traversal of links in the internal data sources,
they have been defined as such because in the open web is it not possible to guarantee results completness~\cite{Hartig2012}.
This ``limitation''~\cite{Hartig2012} on the notion of completeness can be to some extent alleviated with the SWS formalizum.
Withing a structured data vaults an internal data store 

$R_{sa}$ is complete given that the internal data store to produce the  

\begin{equation}
   R_{sa} = \bigcup_{\sigma_{sa} \in SA} \sigma_{sa}(Q)_1 
\end{equation}
map the iri to the documents to the triple and care for the bag semantic


The objective of this new formalization is to first provide a formal definition of the ``data vault'' from \citeauthor{Taelman2023} and secondly give more liberty to the data publisher about the structural assumption they want to provide.
Some structural assumption are virtually cost-free for the data provider such as publishing its data following the LDP specification.
However, some others might be more costly such as providing statistical information about data published in a vault.
Additionally, some files in the vault might intentionally not be as structured as others in such a way that documenting some structured information about the 
data might make less sense. Hence to encapsulate those practical real-world concerns we have endeavored to extend the previous formalization.

WE NEED TO PUT THE SUBWEB SPECIFICATION IN THE PRELIMINARIES.

\iffalse
We propose to formalize an abstract definition of \emph{structural assumptions} and to create a notion of \emph{Structured linked data environments}(SLDE)
to be able to define them more precisely and propose a related notion of result completeness.
A notion of results completeness is important for the comparison of the execution of queries with different reachability criteria.
Reachability criteria already defines completeness but it is from the perspective of the traversal of links in the internal data sources,
they have been defined as such because in the open web is it not possible to guarantee results completness~\cite{Hartig2012}.
This ``limitation'' on the notion of completeness can be to some extent alleviated with the concept of Structured linked data environments.

A Structured linked data environments 
\begin{equation}
E_{ld} = (SA=\{ sa_1, sa_2...sa_n\}, D)
\end{equation}
is a set of structural assumption $SA$  defined at Definition~\ref{def:structuralAssumption},
with an associated domain $D$ in the form of set of URL defining the control web region of the data publisher and the outer limit of application of the structural assumption in $E_{ld}$.
$SA$ is made as such that given an access policy it is possible to find all the results 

\begin{equation}
R_{sa} = \bigcup_{sa \in SA \land i_j \in D} sa(Q, I_{i} = \{i_1, i_2 ..., i_j\}) 
\end{equation}

by evaluation every $sa$ until no new $I_{sa}$ are provided. 
In most cases an $sa$ only need to be evaluated one time because it will be independent of $I_i$ and it is the only variable element of the equation.



\begin{definition}[Structural assumption]\label{def:structuralAssumption}
   A structural assumption $sa(Q, I_{i}) \rightarrow (I_{sa}, I_{nsa})$ is a function that provided a query $Q$ and an set of IRI $I_{i} \in D$ known by the query engine 
   returns a set of IRI $I_{sa}$ containing IRI in $D$ respecting the condition of the structural assumption and a set of IRI $I_{nsa}$ containing IRI in $D$ not respecting
   the condition of the assumption.
   It has to be noted that a $sa$ can be agnostic to either $Q$ (for example the LDP specification~\cite{Taelman2023}) or $I_i$ (for example the type index~\cite{Taelman2023}) but not both.  
\end{definition}



The completness of results given the traversal of $n$ SLDE for the result bag $R$ is defined by 

\begin{equation}
   R   \sqsubseteq  \biguplus_{i=0}^{n} \bigcup_{sa \in SA_i} sa 
\end{equation}

It has to be noted that it can be difficult in practice to find $n$ because a reachability criteria because different reachability criteria might 


\iffalse
While BQPs are syntactic objects, we shall use them as a represen-
tation of Linked Data queries which have a certain semantics. In the
remainder of this section we define this semantics. Due to the open-
ness and distributed nature of Webs such as the WWW we cannot
guarantee query results that are complete w.r.t. all Linked Data on
a Web. Nonetheless, we aim to provide a well-defined semantics.
Consequently, we have to limit our understanding of completeness.
However, instead of restricting ourselves to data from a fixed set
of sources selected or discovered beforehand, we introduce an ap-
proach that allows a query to make use of previously unknown data
and sources. Our definition of query semantics is based on a two-
phase approach: First, we define the part of a Web of Linked Data
that is reached by traversing links using the identifiers in a query
as a starting point. Then, we formalize the result of such a query
as the set of all valuations that map the query to a subset of all
data in the reachable part of the Web. Notice, while this two-phase
approach provides for a straightforward definition of the query se-
mantics in our model, it does not correspond to the actual query
execution strategy of integrating the traversal of data links into the
query execution process as illustrated in Section 2.
\fi



\fi

\subsection{Shape index}
The concept of a shape index was already describe in our publication \citetitle {tam2024opportunitiesshapebasedoptimizationlink}. 
However, it was not formalized.
A shape index $SI = (M, D)$, is a tuple where $M = \{s_1 \mapsto IRI_1, s_2 \mapsto IRI_2 \cdots, s_n \mapsto IRI_n\}$ is a mapping between RDF shapes and a set of IRI 
and where $D = IRI_0 \cup (\bigcup\limits_{i=1}^{n} IRI_i)$ is the domain of application of the index.
$IRI_0$ is the set of IRI that are inside of the domain of the index but not present in the IRI set of the mapping such that 
$iri \in IRI_0 \iff \nexists iri \in M(s_i)$.
When $IRI_0 = \emptyset$ then the shape index describe each ressources of its domain and is denoted as a \emph{complete shape index}. 

A mapping between a shape and a set of IRI has some implication.
when a shape is mapped to a set of $IRI$ then every triple of the knowledge graph $g$ such as $iri \mapsto g \land iri \in IRI$
are validated by the shape $s(g) = \text{true}$.~\sepfootnote{sf:shapeTarget}
A shape mapping given that the shape is close also design an exclusivity in the presence of the of $g$ inside the mapped $IRI$ such that 
given $G = \bigcup\limits_{j=1}^{n} g_j$  where $g_j$ is the knowledge graph behind every ressource of $IRI$ then $s(g) = true \iff g \mapsto iri \in IRI$.
A web specification is also available online.~\sepfootnote{sf:shapeIndexURL}

\subsection{Expressing RDF Data shapes into SPARQL algebra}
A shape expression $e_i$ can be intuitively treated as a segment of a query~\cite{delva2023} more precisely of a star pattern with dependencies query.
A common approach to do shape validation on RDF graph is to convert shapes into SPARQL queries~\cite{labragayo2017validatingdescribinglinkeddata, Corman2019, spapeExpressionConvert}.
\citeauthor{spapeExpressionConvert} has written an informative document on the transformation of ShEx shapes into
\texttt{SELECT} SPARQL queries for validation purposes. 
In this paper, we propose formalizing a similar approach but converting our shape formalism into SPARQL algebra for use in a query shape containment algorithm.
We are looking into defining a query $Q = q_i \bowtie q_{i+1} ... \bowtie q_n$ that is a translation of every shape expression $e$ in a way that given a subgraph of triple $g$ if the 
subgraph is validated by the $e$ then the response of $q_i \in Q$ is $\mathcal{G}$ and on otherwise the response is $\emptyset$.
Thus given an $S$ validating a graph $\mathcal{G} \in d$ in a database $d$ then $\llbracket Q \rrbracket^{\mathcal{G}} = \mathcal{G} \in d$ otherwise an empty set will be returned.

INSPIRE YOURSELF BY Corman2019 style

!!add having count in the query definition for the cardinalities!!
https://www.w3.org/TR/sparql11-query\#sparqlHavingClause

\begin{prop}\label{prop:triplePattern}
   In a shape expression $e_i$, $p$ with the combination of $r$ can be transformed into a triple pattern $tp$ with a cardinality property path.
   Such as $Q_{e_i}(x,o) = (x, p^r, o)$. 
   We suppose a hypothetical extension of the cardinality property path where a user can define any range cardinality property path where $^r$ is a range.
\end{prop}

\begin{prop}
   In a shape expression $e_i$ the $c$ is transform into a filter expression if $c$ target a RDF type for instance the type of $o$ must be a string or
   a comparison with a litteral for instance $o=3$ such as by combining with Proposition \ref{prop:triplePattern} $Q_{e_i}(x,o) = \sigma(x, p^r, o), c(o)$.
   If $c$ target a shape then $Q_{e_i}(x,o) = \sigma(x, p^{r_{1}}, o_1) \bowtie \sigma(o_1, p_1^{r_2}, o_2)$.
\end{prop}

\begin{prop}
   In a disjunction shape expression $e_i = e_j|e_k$ then the expression can be transform into the union between two queries
   $Q_{e_i}(x,o)= Q_{e_j}(x,o_j) \cup Q_{e_k}(x,o_k)$.
\end{prop}

\begin{prop}
   In a shape expression when $n= \mathrm{true}$ then $\neg Q_{e_i}(x,o)$.
\end{prop}

\begin{prop}
   A shape $S$ is transform into $Q_{E} = Q_{e_1} \bowtie Q_{e_2} ... \bowtie Q_{e_{n_e}}$ given $n_e$ shape expressions and a close shapes $op = \mathrm{false}$.
\end{prop}

\begin{prop}
   A shape $S$ is transform into $Q_{E} = \sigma(x, p, o) -  (Q_{e_1} \bowtie Q_{e_2} ... \bowtie Q_{e_{n_e}}) \forall n \in e_i = \mathrm{true} $ where $Q_{e_i}$ 
   $x,p, o \in \mathcal{V}$ given $n_e$ shape expressions and $S$ is an open shapes $op = \mathrm{true}$.
\end{prop}

\subsection{Query shape containment}
The query shape containment problem is similar to the query containment problem as it 
try to determine if the possible answers of the star pattern with dependencies query are contained in any instance of data respecting an RDF data shape.
We can divide the query shape contaiment problem into two category the \emph{strict} and \emph{loose} containment.
The difference between the problems lies into their handling of the union of queries, optional triple patterns and cardinality of property path.


We consider an RDF data shapes as start pattern with dependencies such as 
\begin{equation}
   Q_{shape} = Q_{body} \bigvee Q_{nested body}
\end{equation}
because the subject of a shape is always a single variable 

CITE FOR PROPERTY PATH PARTICULARLY THE NEGATION \cite{Kostylev2015}



\subsection{Online source selection in environments with a shape index}

The shape index by solving the query containment problem can be expressed as a structural assumption.
... define the algorithm